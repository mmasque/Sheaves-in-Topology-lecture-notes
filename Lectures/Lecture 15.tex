\chapter{Proper maps, proper pushforward, the proper base change theorem}
\label{lecture:15}

We begin with some general theory of proper maps.

\section{Proper maps}

\begin{defn}
	A continuous map $f\colon Y \to X$ between topological spaces is \emph{universally closed} if for any map $X' \to X$ the base change $X' \pullback{X} Y \to Y$ is a closed map.
\end{defn}

\remyquote{The proof I'll give for this lemma is taken from Bourbaki. You'll hate it.}
\begin{lem}
	Let $X \in \catTopologicalSpace$. The map $X \to *$ is universally closed if and only if $X$ is compact.
\end{lem}
\begin{proof}
	If $X$ is compact, we need to show that the projection map $\pi\colon X\times Y \to Y$ is closed for any $Y \in \catTopologicalSpace$. Let $Z \subset X \times Y$ closed and let $y \in Y \setminus \pi(Z)$. We will construct an open set around $y$ that does not intersect $\pi(Z)$.
	Notice that $X \cong X \times \{y\} \subset Z^{c}$, so we can cover $X \times \{y\}$ by finitely many opens contained in $Z^c$. Their intersection is still an open open $U$, and the projection map is open so $\pi(U)$ is open containing $y$ and not intersecting $\pi(Z)$.
	Conversely, assume $X \to *$ is universally closed.
	To prove compactness we will use prove that any collection of closed subsets of $X$ with the finite intersection property has nonempty intersection. So let $Z_i \subset X$ ($i \in I$) be a such a collection. We use the notation $Z_J = \bigcap_{j \in J}Z_j$ for $J \subset I$.
	Define a new space \[
    	X' = X^{\text{disc}} \cup \{\infty\}
    \] where $U \subset X'$ is open if and only if $\infty \in U$ implies $Z_J \subset U$ for some finite subset $J \subset I$.
	One checks this defines a topology on $X'$. Note that the subspace topology on $X'$ is still the discrete topology. Furthermore, any open subset $U$ containing $\infty$ contains some $Z_J$ for $J$ finite, and since $Z_J \neq \emptyset$ we have $U \cap X^\text{disc} \neq \emptyset$: $X^{\text{disc}}$ is dense in $X'$. Let $\overline{\Delta}$ be the closure of $\{(x,x) \mid x \in X\} \subset X \times X'$. The image in $X'$ of $\overline{\Delta}$ is closed by assumption and contains $X$ so it is equal to $X'$ because $X$ is dense in $X'$. Thus there exists a point $(x, \infty) \in \overline{\Delta}$ for some $x \in X$. Now for all opens $U \subset X$ containing $x$ and all $J \subset I$ finite, there exists a point $y \in X$ with $(y, y) \in U \times (Z_j \cup \{\infty\})$ \todo{why? Look at Remy's notes}, and thus $x \in \overline{Z_J} = Z_J$ for all $J \subset I$ finite.
\end{proof}

\begin{prop}
	Let $f\colon Y \to X$ be a continuous map. The following are equivalent:
	\begin{enumerate}
    	\item $f$ is universally closed,
		\item the product $f \times \id_W\colon Y \times W \to X \times W$ is closed for every space \(W\),
		\item $f$ is closed and the fibre $f^{-1}(x)$ is compact for all $x \in X$,
		\item $f$ is closed and the fibre $f^{-1}(Z)$ is compact for all $Z \subset X$ closed.
    \end{enumerate}
\end{prop}
\todo{Add the proof}

\begin{defn}
	A map $f\colon Y \to X$ of topological spaces is called \emph{separated} if the diagonal map $
    	\Delta_{Y/X}\colon Y \to Y\pullback{X} Y
    $ is closed. The map $f$ is called \emph{proper} if it is universally closed and separated.
\end{defn}
\begin{rmk}
	Some authors do not include the separated condition, which is automatic if $X$ and $Y$ are both locally compact Hausdorff spaces.
\end{rmk}

\section{Proper pushforward}

\begin{defn}
	Let $f\colon Y \to X$ be a map of topological spaces. The \emph{proper pushforward} (sometimes called the \emph{pushforward with proper support}) is the functor $f_{!}\colon \catAbelianSheaf(Y) \to \catAbelianSheaf(X)$ defined, given a sheaf $\sh{F} \in \catAbelianSheaf(Y)$, by \[
		(f_{!}\sh{F})(U) = \setpred{s \in \sh{F}(f^{-1}(U))}{\restr{f}{\supp(s)}\colon \supp(s) \to U \text{ is proper}}\text{.}
    \]
\end{defn}
\begin{rmk}
	In Homework~8 you are asked to show this assignment gives a sheaf on $X$.
\end{rmk}
\begin{exmp}
	In Additional exercise~15.3 you are asked to show that for the inclusion of an open subset $j\colon U \hookrightarrow X$ the proper pushforward becomes the extension-by-zero functor.
\end{exmp}
We have a functor $f_!$ between abelian categories. One checks it is left exact.
\begin{defn}
	The $i$th \emph{higher pushforward with proper support} is the \(i\)th right derived functor $\rightderived^if_!\colon \catAbelianSheaf(Y) \to \catAbelianSheaf(X)$ of the \emph{proper pushforward}.
\end{defn}

\begin{exmp}
	If $j\colon U \hookrightarrow X$ is an open subset then $\rightderived^ij_! = 0$ for all $i > 0$ since $j_!$ is exact.
\end{exmp}

\begin{exc}\label{exc:compactly-supported-cohomology-higher-proper-pushforward-to-point}
	If $X$ is Hausdorff then $\cohomology^i_c(X, \sh{F}) = \rightderived^i f_!\sh{F}$ for any sheaf $\sh{F} \in \catAbelianSheaf(X)$ and $f\colon X \to *$.
\end{exc}

\section{The proper base change theorem}
\remyquote{We can't prove this because we're dumb. Which is fine.}
\noindent
We are now ready to state one of the main results of the course. We do so in two versions, one involving the pushforward and one involving the proper pushforward.

\begin{thm}\label{thm:proper-base-change-theorem-1}
	Let $f\colon Y \to X$ be a proper map of topological spaces and assume that $X$ and $Y$ are either both paracompact Hausdorff or both locally compact Hausdorff. For $\sh{F} \in \catAbelianSheaf(Y)$ the natural map \[
    	(\rightderived^i f_* \sh{F})_x \to \cohomology^i(f^{-1}(x), \sh{F})
    \] is an isomorphism for all $x \in X$ and for all $i \geq 0$.
\end{thm}

\begin{rmk}
	\Cref{thm:proper-base-change-theorem-1} is true for $f$ proper with no further hypotheses, but we will not prove it in this course. See \cite[Lemma~\textsc{09v6}]{stacks-project} for a proof of the general case.
\end{rmk}

\begin{thm}\label{thm:proper-base-changed-theorem-2}
	Let $f\colon Y \to X$ be a continuous map of locally compact Hausdorff spaces. For any $\sh{F} \in \catAbelianSheaf(Y)$ the natural map \[
    	(\rightderived^i f_! \sh{F})_x \to \cohomology^i_c(f^{-1}(x), \sh{F})
    \] is an isomorphism for all $x \in X$ and all $i \geq 0$.
\end{thm}

We will first prove \cref{thm:proper-base-change-theorem-1}. We will need the following lemma.

\begin{lem}\label{lem:proper-base-change-1}
	Let $i\colon Z \hookrightarrow X$ be a closed subset. Assume that either \begin{enumerate}
    	\item $X$ is paracompact Hausdorff or
		\item $X$ is locally compact Hausdorff and that $Z$ is compact.
    \end{enumerate}
	Then there is an isomorphism \[
    	\colim_{Z \subset U \text{ open}}\cohomology^i(U, \sh{F}) \xrightarrow{\sim} \cohomology^i(Z, \sh{F})
    \] for all sheaves and all $i \geq 0$.
\end{lem}
\begin{proof}
	We proved in \cref{prop:restriction-map-isomorphism-paracompact} that the statement holds for $i = 0$.
	Note that the pullback $i^*\colon \catAbelianSheaf(X) \to \catAbelianSheaf(Z)$ preserves soft sheaves if $X$ is paracompact Hausdorff (respectively c-soft sheaves if $X$ is locally compact Hausdorff, irrespective of whether $Z$ is compact): if $W \subset Z$ is closed, then any $s \in \sh{F}(W)$ extends to $\sh{F}(X)$ hence also to $\sh{F}(Z)$ \todo{where have we used the assumptions on $X$?} (note that $W$ is compact if $Z$ is).
	Now given an injective resolution $0 \to \sh{F} \to \cochaincomplex{I}$, it is (c)-soft, and $i^*\cochaincomplex{I}$ is a (c)-soft resolution of $i^*\sh{F}$. Unravelling the definitions, we obtain the desired result:
	\begin{align*}
    	\cohomology^i(Z, \sh{F}|_Z) &= \cohomology^i(\Gamma(Z, i^* \cochaincomplex{I})) \\ &\cong \cohomology^i(\filteredcolim_{Z \subset U} \Gamma(U, i^*\cochaincomplex{I}))\\ & \cong \filteredcolim_{Z\subset U} \cohomology^i(\Gamma(U, i^*\cochaincomplex{I}))\\ &= \filteredcolim_{Z \subset U} \cohomology^i(U, \sh{F}).
	\end{align*}
\end{proof}

We are now ready to prove the first version of the Proper Base Change Theorem.
\begin{proof}[name={of \cref{thm:proper-base-change-theorem-1}}]
	Recall that $\rightderived^i f_* \sh{F}$ is the sheafification of the assignment $U \mapsto H^i(f^{-1}(U), \sh{F})$, so that \[
    	(\rightderived^i f_* \sh{F})_x = \colim_{x \in U \text{ open}} \cohomology^i(f^{-1}(U), \sh{F}).
    \] By \cref{lem:proper-base-change-1} we know that \[
    	\cohomology^i(f^{-1}(x), \sh{F}) = \colim_{f^{-1}(x) \subset V \text{ open}} \cohomology^i(V, \sh{F}),
    \] since $f^{-1}$ is compact. By properness, the opens $f^{-1}(U)$ are coinitial amongst the $V \supset f^{-1}(x)$: an open $V$ contains $f^{-1}(f(V^C)^C)$. So the colimits agree. \todo{Clarify argument at the end.}
\end{proof}

\begin{lem}\label{lem:proper-base-change-2}
	Let $i\colon Z \hookrightarrow X$ be a closed subspace of a locally compact Hausdorff space. Let $\sh{F} \in \catAbelianGroup(X)$ be a c-soft sheaf of abelian groups. Then $i^*\sh{F}$ is c-soft and the map \[
    	\cohomology^0_c(X, \sh{F}) \to \cohomology^0_c(Z, \sh{F})
    \] is surjective.
\end{lem}
\begin{proof}
	We already proved in \cref{lem:proper-base-change-1} that $i^*\sh{F}$ is c-soft. Given $s \in\cohomology^0_c(Z, \sh{F})$, choose a compact neighborhood $K \subset X$ of $\sup(s)$. The section $s|_{Z \cap K}$ extends to a section \[
    	s_1 \in \cohomology^0((Z\cap K) \cup \partial K, \sh{F})
    \] that is $0$ on the boundary $\partial K$. This section further extends to $s_2 \in \cohomology^0(K, \sh{F})$ by c-softness. We can extend by $0$ to a section in $\cohomology_c^0(X, \sh{F})$, since $s_2|_{\partial K} = 0$. \todo{Add details and verify.}
\end{proof}

\begin{proof}[name={of \cref{thm:proper-base-changed-theorem-2}}]
	We have a map \[
    	(f_!\sh{F})_x = \filteredcolim_{x \in U \text{ open}}(f_!\sh{F})(U) \xrightarrow{\phi} \cohomology_c^0(f^{-1}(x), \sh{F})
    \] given by $s \mapsto s|_{f^{-1}(x)}$ which we claim is injective.
	If $s \in (f_!\sh{F})(U)$ maps to zero under this map, then \[
    	\sup(s) \cap f^{-1}(x) = \emptyset.
    \] Since the map $\sup(s) \to U$ is proper, the set $V = U \setminus f(\sup(s))$ is an open neighborhood of $x$ with the property that $\sup(s) \cap f^{-1}(V) = \emptyset$.
	Thus $s$ maps to zero in $(f_!\sh{F})(V)$, proving injectivity of $\phi$. If $\sh{F}$ is c-soft, then the composition \[
    	\cohomology^0_c(Y, \sh{F}) \hookrightarrow (f_! \sh{F})(X) \to (f_!\sh{F})_x \xrightarrow{\phi} \cohomology^0_c(f^{-1}(x), \sh{F})
    \] is surjective by \cref{lem:proper-base-change-2} so the map $\phi$ is an isomorphism.

	Now take an injective resolution \[
    	0 \to \sh{F} \to I^0 \to I^1 \to \cdots
    \] Then $i^*\cochaincomplex{I}$ is a c-soft resolution of $i^*\sh{F}$ by \cref{lem:proper-base-change-2}, so we have \[
    	\cohomology^i_c(f^{-1}(x), \sh{F}) = \cohomology^i(\cohomology^0_c(f^{-1}(x), \cochaincomplex{I})) = \cohomology^i(\filteredcolim_{x \in U \text{ open}}(f_!\cochaincomplex{I}(U))) = \filteredcolim_{x \in U} \cohomology^i((f_!\cochaincomplex{I})(U)) = (\rightderived^i f_!\sh{F})_x,
    \] and we are done.\todo{expand details, verify}
\end{proof}

The next corollary explains the use of \emph{base change} in the namings of \cref{thm:proper-base-change-theorem-1} and \cref{thm:proper-base-changed-theorem-2}.

\begin{cor}\label{cor:proper-base-change}
Let the diagram
\[\begin{tikzcd}
	{Y'} \ar[rd, pullback] & {X'} \\
	Y & X
	\arrow["{f'}", from=1-1, to=1-2]
	\arrow["{g'}"', from=1-1, to=2-1]
	\arrow["g", from=1-2, to=2-2]
	\arrow["f"', from=2-1, to=2-2]
\end{tikzcd}\]
be a pullback square in $\catTopologicalSpace$.
\begin{enumerate}
	\item If $f$ is a proper map and $X$ and $X'$ are either both paracompact Hausdorff or both locally compact Hausdorff, then there are canonical isomorphisms \[
    	g^* \rightderived^i f_* \xrightarrow{\cong} (\rightderived^i f'_*)g'^{*}\colon \catAbelianSheaf(Y) \to \catAbelianSheaf(X).
    \]
	\item If $X,X'$ and $Y'$ are locally compact Hausdorff, then there are canonical isomorphisms \[
    	g^*\rightderived^i f_! \xrightarrow{\cong} (\rightderived^i f'_!)g'^{*}\colon \catAbelianSheaf(Y) \to \catAbelianSheaf(X').
    \]
\end{enumerate}
\end{cor}
\begin{proof}[sketch]
	\todo{write}
\end{proof}

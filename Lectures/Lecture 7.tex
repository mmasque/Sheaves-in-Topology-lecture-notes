\documentclass[../main.tex]{subfiles}

\begin{document}

\chapter{Limits and colimits of (pre)sheaves}
\begin{defn}
    A \emph{diagram} in a category $\cat C$ is a functor $D: \cat J \to \cat C$ from a small category $\cat J$. 
\end{defn}

In this lecture, we will show that every diagram $D: \cat J \to \catPresheaf(X)$ or $D: \cat J \to \catAbelianPresheaf(X)$ has a limit and colimit, and we will describe them explicitly.

\section{Limits and colimits in $\catSet$}

We will begin with treating limits and colimits in $\catSet$. For a more complete treatment of this topic, we refer the reader to \cite[Chapter~3.2]{riehlCategoryTheoryContext2016}.
\begin{prop}
    The limit of $D: \cat J \to \catSet$ exists and is given by \[
        \lim_{i\in \cat J}D(i) = \{(a_i)_{i \in \cat J} \in \prod_{i \in \cat J}D(i) \mid D(\phi)(a_i) = a_j \text{ for all } \phi: i \to j \text{ in } \cat J\}
    \]
\end{prop}
\begin{proof}
	Denote our candidate limit set $S$. There are maps $S \xrightarrow{\pi_i} D(i)$ for all $i \in \cat J$. For any $\phi: i \to j$ in $\cat J$ the diagram 
    \[
      \begin{tikzcd}
                & S \arrow[ld, "\pi_i"'] \arrow[rd, "\pi_j"] &      \\
D(i) \arrow[rr] &                                            & D(j)
\end{tikzcd}  
    \]
commutes. This commutativity is directly by construction of our set $S$. Given any cone $(C \xrightarrow{\pi_i} D(i))_{i \in \cat J}$ there exists a unique map of cones to $(S \xrightarrow{\pi_i} D(i))_{i \in \cat J}$ making the diagram 
\[
\begin{tikzcd}
                & C \arrow[ldd, bend right] \arrow[rdd, bend left] \arrow[d, dashed] &      \\
                & S \arrow[ld, "\pi_i"'] \arrow[rd, "\pi_j"]                         &      \\
D(i) \arrow[rr] &                                                                    & D(j)
\end{tikzcd}\]

commute. It is given by $c \mapsto (\phi_i(c))_{i \in \cat J}$.

\end{proof}

\begin{rmk}\label{rmk:limits-are-products-and-equalisers-in-set}
	The limit of a diagram $D: \cat J \to \catSet$ is the equalizer of 

	\[
		\prod_{i \in \cat J} D(i) \rightrightarrows \prod_{\phi: i \to j} D(j)
	\]
	with maps 
	\begin{align*}
		(a_i)_{i \in \cat J} & \mapsto (a_j)_{\phi: i \to j} \\
		(a_i)_{i \in \cat J} & \mapsto (D(\phi)(a_i))_{\phi: i \to j}.
	\end{align*} 
\end{rmk}


\begin{cor}\label{cor:hom-limits-pass-to-set}
	For any locally small category $\cat C$ we have 
    \[
        \Hom_{\cat C}(X, \lim_{i \in \cat J}D(i)) \cong \lim_{i \in \cat J} \Hom_{\cat C}(X, D(i))
    \] and 
    \[
         \Hom_{\cat C}(\colim_{i \in J} D(i), Y) \cong \lim_{i \in \cat J} \Hom_{\cat C}(D(i), Y).   
    \]
\end{cor}
\begin{proof}(Sketch of the first isomorphism)
To give a map $X \to \lim D(i)$ is to give a cone $$(X \to D(i))_{i \in \cat J};$$ this means giving maps $\pi_i \in \Hom_{\cat C}(X, D(i))$ such that for all $\phi: i \to j$ in $\cat J$ the diagram 
\[\begin{tikzcd}
                            & X \arrow[ld, "\pi_i"'] \arrow[rd, "\pi_j"] &      \\
D(i) \arrow[rr, "D(\phi)"'] &                                            & D(j)
\end{tikzcd}\] commutes. That is, it is to give a collection $(\pi_i)_{i \in \cat J} \in \prod_{i \in \cat J} \Hom(X, D(i))$ such that $\Hom_{\cat C}(X, D(\phi)) = \pi_j$ for all $\pi: i \to j$. 
	
\end{proof}

\begin{cor}
 A locally small category $\cat C$ has all small limits if and only if it has all small products and equalisers (the dual statement, requiring small coproducts and coequalisers, holds for colimits). 
\end{cor}
\begin{proof}
	Small products and equalisers are small limits. We thus need only prove small limits exist if small products and equalisers exist. To this end, let $D: \cat J \to \cat C$ be a diagram. 
	Consider the equalizer 
	\[ E = \eq(\prod_{i \in \cat J}D(i) \rightrightarrows \prod_{\phi: i \to j}D(j)).\] By \cref{cor:hom-limits-pass-to-set}, the set of morphisms into $E$ can be computed as a limit in set: 
	\[
		\Hom_{\cat C}(X, E) \cong \eq(\Hom_{\cat C}(X, \prod_{i \in \cat J}D(i)) \rightrightarrows \Hom_{\cat C}(X, \prod_{\phi: i \to j}D(j))).
	\] Applying \cref{cor:hom-limits-pass-to-set} again to the objects of this equalizer yields and using \cref{rmk:limits-are-products-and-equalisers-in-set} yields
	\[
		\Hom_\cat{C}(X, E) \cong \eq(\prod_{i \in J}\Hom_{\cat C}(X, D(i)) \rightrightarrows \prod_{\phi: i \to j} \Hom_{\cat C}(X, D(j))) \cong \lim_{i \in \cat J}\Hom_{\cat C}(X, D(i)).
	\] 
\end{proof}





\end{document}

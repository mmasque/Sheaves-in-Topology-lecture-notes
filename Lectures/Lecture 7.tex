\documentclass[../main.tex]{subfiles}

\begin{document}

\chapter{Limits and colimits of (pre)sheaves}
\begin{defn}
    A \emph{diagram} in a category $\cat C$ is a functor $D: \cat J \to \cat C$ from a small category $\cat J$.
\end{defn}

In this lecture, we will show that every diagram $D: \cat J \to \catPresheaf(X)$ or $D: \cat J \to \catAbelianPresheaf(X)$ has a limit and colimit, and we will describe them explicitly.

\section{Limits and colimits of sets}

We will begin by treating limits and colimits in $\catSet$. For a more complete treatment of this topic, we refer the reader to \cite[\S~3.2]{riehlCategoryTheoryContext2016}.
\begin{prop}
    The limit of $D: \cat J \to \catSet$ exists and is given by \[
        \lim_{i\in \cat J}D(i) = \{(a_i)_{i \in \cat J} \in \prod_{i \in \cat J}D(i) \mid D(\phi)(a_i) = a_j \text{ for all } \phi: i \to j \text{ in } \cat J\}
    \]
\end{prop}
\begin{proof}
	Denote our candidate limit set $S$. There are maps $S \xrightarrow{\pi_i} D(i)$ for all $i \in \cat J$. For any $\phi: i \to j$ in $\cat J$ the diagram
    \[
      \begin{tikzcd}
                & S \arrow[ld, "\pi_i"'] \arrow[rd, "\pi_j"] &      \\
D(i) \arrow[rr] &                                            & D(j)
\end{tikzcd}
    \]
commutes. This commutativity is directly by construction of our set $S$. Given any cone $(C \xrightarrow{\pi_i} D(i))_{i \in \cat J}$ there exists a unique map of cones to $(S \xrightarrow{\pi_i} D(i))_{i \in \cat J}$ making the diagram
\[
\begin{tikzcd}
                & C \arrow[ldd, bend right] \arrow[rdd, bend left] \arrow[d, dashed] &      \\
                & S \arrow[ld, "\pi_i"'] \arrow[rd, "\pi_j"]                         &      \\
D(i) \arrow[rr] &                                                                    & D(j)
\end{tikzcd}
\]

commute. It is given by $c \mapsto (\phi_i(c))_{i \in \cat J}$.

\end{proof}

\begin{rmk}\label{rmk:limits-are-products-and-equalisers-in-set}
	The limit of a diagram $D: \cat J \to \catSet$ is the equalizer of

	\[
		\prod_{i \in \cat J} D(i) \rightrightarrows \prod_{\phi: i \to j} D(j)
	\]
	with maps
	\begin{align*}
		(a_i)_{i \in \cat J} & \mapsto (a_j)_{\phi: i \to j} \\
		(a_i)_{i \in \cat J} & \mapsto (D(\phi)(a_i))_{\phi: i \to j}.
	\end{align*}
\end{rmk}


\begin{cor}\label{cor:hom-limits-pass-to-set}
	For any locally small category $\cat C$ we have
    \[
        \Hom_{\cat C}(X, \lim_{i \in \cat J}D(i)) \cong \lim_{i \in \cat J} \Hom_{\cat C}(X, D(i)),
    \]
    \[
         \Hom_{\cat C}(\colim_{i \in J} D(i), Y) \cong \lim_{i \in \cat J} \Hom_{\cat C}(D(i), Y).
    \]
\end{cor}
\begin{proof}(Sketch of the first isomorphism)
To give a map $X \to \lim D(i)$ is to give a cone $$(X \to D(i))_{i \in \cat J};$$ this means giving maps $\pi_i \in \Hom_{\cat C}(X, D(i))$ such that for all $\phi: i \to j$ in $\cat J$ the diagram
\[\begin{tikzcd}
                            & X \arrow[ld, "\pi_i"'] \arrow[rd, "\pi_j"] &      \\
D(i) \arrow[rr, "D(\phi)"'] &                                            & D(j)
\end{tikzcd}\] commutes. That is, it is to give a collection $(\pi_i)_{i \in \cat J} \in \prod_{i \in \cat J} \Hom(X, D(i))$ such that $\Hom_{\cat C}(X, D(\phi)) = \pi_j$ for all $\pi: i \to j$.

\end{proof}

\begin{cor}\label{cor:small-limits-from-eq-prod}
 A locally small category $\cat C$ has all small limits if and only if it has all small products and equalisers (the dual statement, requiring small coproducts and coequalisers, holds for colimits).
\end{cor}
\begin{proof}
	Small products and equalisers are small limits. We thus need only prove small limits exist if small products and equalisers exist. To this end, let $D: \cat J \to \cat C$ be a diagram.
	Consider the equalizer
	\[ E = \eq(\prod_{i \in \cat J}D(i) \rightrightarrows \prod_{\phi: i \to j}D(j)).\] By \cref{cor:hom-limits-pass-to-set}, the set of morphisms into $E$ can be computed as a limit in set:
	\[
		\Hom_{\cat C}(X, E) \cong \eq(\Hom_{\cat C}(X, \prod_{i \in \cat J}D(i)) \rightrightarrows \Hom_{\cat C}(X, \prod_{\phi: i \to j}D(j))).
	\] Applying \cref{cor:hom-limits-pass-to-set} again to the objects of this equalizer yields and using \cref{rmk:limits-are-products-and-equalisers-in-set} yields
	\[
		\Hom_\cat{C}(X, E) \cong \eq(\prod_{i \in J}\Hom_{\cat C}(X, D(i)) \rightrightarrows \prod_{\phi: i \to j} \Hom_{\cat C}(X, D(j))) \cong \lim_{i \in \cat J}\Hom_{\cat C}(X, D(i)).
	\]
\end{proof}
\begin{cor}
	The colimit of any diagram in $\catSet$ exists, and it is given by \[
    	\coeq(\coprod_{\phi: i\to j} D(i) \rightrightarrows \coprod_{i \in j} D(i))
    \]
\end{cor}

\begin{cor}
	Let $D: \cat J \to \cat C$ be a diagram with $\cat J$ filtered. Then \[
    	\filteredcolim_{i \in \cat J} D(i) = (\coprod_{i \in \cat J} D(i))/_\sim
    \] where $a \sim b$ for $a \in D(i)$ and $b \in D(j)$ if and only if there exist maps $\phi: i \to k$ and $\psi: j \to k$ such that $D(\phi)(a) = D(\psi)(b)$.
\end{cor}\todo{proof of this, some prose.}

\section{Limits and colimits of abelian groups}
\remyquote{Bootstrap, bootstrap, bootstrap.}
\begin{lem}
	Every diagram in $\catAbelianGroup$ has a limit and a colimit.
\end{lem}
\begin{proof}
	We saw (co)products and (co)equalisers in the previous lecture. Products of abelian groups are products on the underlying sets with the usual group structure. Coproducts are direct sums. Equalisers are kernels, coequalisers are cokernels: \begin{align*}
	  	\eq(A \mathrel{\mathop{\rightrightarrows}^{\mathrm{f}}_{\mathrm{g}}} B) & = \ker(f - g)\\
		\coeq(A \mathrel{\mathop{\rightrightarrows}^{\mathrm{f}}_{\mathrm{g}}}B) & = \coker(f - g).
    \end{align*} Thus we can use \cref{cor:small-limits-from-eq-prod} to compute limits and colimits from (co)products and (co)equalizers via kernels and cokernels.
\end{proof}
\begin{rmk}
If $\cat J$ is filtered, the colimit is easier: the filtered colimit \[
	\filteredcolim D(i) = (\coprod(i))/_\sim
\] in $\catSet$ gets an abelian group structure as follows: if $a \in D(i)$ and $b \in D(j)$, then choose maps $\phi: i \to k$ and $\psi: j \to k$ (using that $\cat J$ is filtered) and set \[
	a + b =  D(\phi)(a) + D(\psi)(b).
\] One needs to check this is well defined and that it indeed turns the colimit in $\catSet$ to the colimit in $\catAbelianGroup$. This result actually follows from a more general fact: forgetful functors create filtered colimits \cite[Theorem~5.6.5]{riehlCategoryTheoryContext2016} \todo{this reference deals with monadic functors, not immediately obvious it proves our claim. A small discussion of this would be nice.}
\end{rmk}

\section{Limits and colimits of presheaves}
% \remyquote{Maybe let me write down what I just said because it's nice.}

Recall that a presheaf to a category $\cat D$ is a contravariant functor $F \in \catFunctor(\cat C\opp, \cat D)$. In this course the category $\cat D$ has been $\catSet$ or $\catAbelianGroup$, for which we know small limits and colimits exist. The following lemma appears as \cite[Proposition~3.3.9]{riehlCategoryTheoryContext2016} stripped of the context of presheaves. It tells us that if $\cat C$ is small, the work we've done so far for $\catSet$ and $\catAbelianGroup$ suffices to show presheaves into these categories have small limits and colimits. Furthermore, we can compute (co)limits of such presheaves objectwise. We have seen these claims before in \cref{lem:psh-category-set}.

\begin{lem}
	Let $\cat C$ be a small category and $\cat D$ be a category with small limits and colimits.
	Then any diagram $D: \cat J \to \catFunctor(\cat C\opp, \cat D)$ has a limit and a colimit, computed objectwise:
	\begin{align*}
    	(\lim_{i\in \cat J}D(i))(U) &= \lim_{i \in \cat J}(D(i)(U))\\
    	(\colim_{i\in \cat J}D(i))(U) &= \colim_{i \in \cat J}(D(i)(U))
    \end{align*} for all $U \in \cat C$.

\end{lem}
\begin{proof}
    We do limits; colimits follow dually. We will turn $U \mapsto \lim_{i \in \cat J}(D(i)(U))$ into a functor $\cat C^op \to \cat D$. Given $f: U \to V$ in $\cat C$, we get morphisms 
    \[
        \lim_{i \in \cat J} D(i)(V) \xrightarrow{\pi_i} D(i)(V) \xrightarrow{D(i)(f)} D(i)(U)
    \] turning $\lim_{i \in \cat J} D(i)(V)$ into a cone over $D(-)(U)$ by naturality of $D(-)(f)$. This gives a unique map \[
        \lim_{i \in \cat J} D(i)(V) \to \lim_{i \in \cat J} D(i)(U)
    \] which turns $U \mapsto \lim_{i \in J} D(i)(U)$ into a functor. It is then clearly a limit of $D$. For more details, see \cite[Chapter~5.3]{maclane:71}
\end{proof}

\begin{cor}
	The presheaf categories $\catPresheaf(X)$ and $\catAbelianPresheaf(X)$ have all small limits and colimits.
\end{cor}

\section{Limits and colimits of sheaves}

\begin{thm}\label{thm:limits-colimits-sheaves}
	Any diagram $D: \cat J \to \catSheaf(X)$ (or $D: \cat J \to \catAbelianSheaf(X)$) has a limit, computed in $\catPresheaf(X)$ ($\catAbelianPresheaf(X)$).
	Such a diagram also has a colimit in $\catSheaf(X)$ ($\catAbelianSheaf(X)$), obtained by sheafification of the colimit in $\catPresheaf(X)$ ($\catAbelianPresheaf(X)$).
	% In other words, the forgetful functor $\catSheaf(X) \to \catPresheaf(X)$ reflects limits.
\end{thm}
\begin{proof}
   For limits, it suffices to show that the limit in presheaves is a sheaf: we saw that limits in $\catPresheaf(X)$ are computed objectwise. If $U = \cup_{i \in I}U_i$ is an open cover, then \[
    \lim_{k \in \cat J}D(k)(U) \to \prod_{i \in I} \lim_{k \in \cat J}D(k)(U_i) \rightrightarrows \prod_{i, j \in I} \lim_{k \in \cat J} D(k)(U_i \cap U_j)
   \] is an equaliser: either check by hand, or use that limits commute with limits.

   For colimits, we get a cocone $(D(j) \to (\colim D(i))^\#)_{j \in \cat J}$. Given any other cocone $(D(j) \to \mathcal{F})_{j \in \cat J}$, we get a unique morphism of co-cones in presheaves:
   \[
    (D(j) \to \colim_{i \in \cat J} D(i))_{j \in \cat J} \to (D(j) \to \mathcal{F})_{j \in \cat J}.
   \] The universal property of sheafification gives a unique factorisation 
   
  \[\begin{tikzcd}
\colim D(i) \arrow[rr] \arrow[rd] &                                           & \mathcal{F} \\
                                  & (\colim D(i))^\# \arrow[ru, "\exists !"'] &            
\end{tikzcd}\] which shows that the sheafified presheaf is the colimit. 
\end{proof}
\begin{exmp}
    The terminal object in $\catSheaf(X)$ is the constant sheaf $*=h_{X/X}$ given by $U\mapsto *$ for all opens $U \subset X$. The coproduct $\coprod_{s \in S} *$ in $\catPresheaf(X)$ is the constant presheaf $U \mapsto S$, whose sheafification is the constant sheaf $\underline{S}$. Thus $\underline{S} \cong \coprod_{s \in S} *$ in $\catSheaf(X)$, which is also clear in $\catLocalHomeomorphism{/X}$. \todo{Give more detail in this and the last proof.}
\end{exmp}
\end{document}

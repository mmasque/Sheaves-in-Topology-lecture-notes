\documentclass[../main.tex]{subfiles}

\begin{document}

\chapter{Limits and colimits of (pre)sheaves}
\begin{defn}
    A \emph{diagram} in a category $\cat C$ is a functor $D: \cat J \to \cat C$ from a small category $\cat J$.
\end{defn}

In this lecture, we will show that every diagram $D: \cat J \to \catPresheaf(X)$ or $D: \cat J \to \catAbelianPresheaf(X)$ has a limit and colimit, and we will describe them explicitly.

\section{Limits and colimits of sets}

We will begin by treating limits and colimits in $\catSet$. For a more complete treatment of this topic, we refer the reader to \cite[Chapter~3.2]{riehlCategoryTheoryContext2016}.
\begin{prop}
    The limit of $D: \cat J \to \catSet$ exists and is given by \[
        \lim_{i\in \cat J}D(i) = \{(a_i)_{i \in \cat J} \in \prod_{i \in \cat J}D(i) \mid D(\phi)(a_i) = a_j \text{ for all } \phi: i \to j \text{ in } \cat J\}
    \]
\end{prop}
\begin{proof}
	Denote our candidate limit set $S$. There are maps $S \xrightarrow{\pi_i} D(i)$ for all $i \in \cat J$. For any $\phi: i \to j$ in $\cat J$ the diagram
    \[
      \begin{tikzcd}
                & S \arrow[ld, "\pi_i"'] \arrow[rd, "\pi_j"] &      \\
D(i) \arrow[rr] &                                            & D(j)
\end{tikzcd}
    \]
commutes. This commutativity is directly by construction of our set $S$. Given any cone $(C \xrightarrow{\pi_i} D(i))_{i \in \cat J}$ there exists a unique map of cones to $(S \xrightarrow{\pi_i} D(i))_{i \in \cat J}$ making the diagram
\[
\begin{tikzcd}
                & C \arrow[ldd, bend right] \arrow[rdd, bend left] \arrow[d, dashed] &      \\
                & S \arrow[ld, "\pi_i"'] \arrow[rd, "\pi_j"]                         &      \\
D(i) \arrow[rr] &                                                                    & D(j)
\end{tikzcd}
\]

commute. It is given by $c \mapsto (\phi_i(c))_{i \in \cat J}$.

\end{proof}

\begin{rmk}\label{rmk:limits-are-products-and-equalisers-in-set}
	The limit of a diagram $D: \cat J \to \catSet$ is the equalizer of

	\[
		\prod_{i \in \cat J} D(i) \rightrightarrows \prod_{\phi: i \to j} D(j)
	\]
	with maps
	\begin{align*}
		(a_i)_{i \in \cat J} & \mapsto (a_j)_{\phi: i \to j} \\
		(a_i)_{i \in \cat J} & \mapsto (D(\phi)(a_i))_{\phi: i \to j}.
	\end{align*}
\end{rmk}


\begin{cor}\label{cor:hom-limits-pass-to-set}
	For any locally small category $\cat C$ we have
    \[
        \Hom_{\cat C}(X, \lim_{i \in \cat J}D(i)) \cong \lim_{i \in \cat J} \Hom_{\cat C}(X, D(i)),
    \]
    \[
         \Hom_{\cat C}(\colim_{i \in J} D(i), Y) \cong \lim_{i \in \cat J} \Hom_{\cat C}(D(i), Y).
    \]
\end{cor}
\begin{proof}(Sketch of the first isomorphism)
To give a map $X \to \lim D(i)$ is to give a cone $$(X \to D(i))_{i \in \cat J};$$ this means giving maps $\pi_i \in \Hom_{\cat C}(X, D(i))$ such that for all $\phi: i \to j$ in $\cat J$ the diagram
\[\begin{tikzcd}
                            & X \arrow[ld, "\pi_i"'] \arrow[rd, "\pi_j"] &      \\
D(i) \arrow[rr, "D(\phi)"'] &                                            & D(j)
\end{tikzcd}\] commutes. That is, it is to give a collection $(\pi_i)_{i \in \cat J} \in \prod_{i \in \cat J} \Hom(X, D(i))$ such that $\Hom_{\cat C}(X, D(\phi)) = \pi_j$ for all $\pi: i \to j$.

\end{proof}

\begin{cor}\label{cor:small-limits-from-eq-prod}
 A locally small category $\cat C$ has all small limits if and only if it has all small products and equalisers (the dual statement, requiring small coproducts and coequalisers, holds for colimits).
\end{cor}
\begin{proof}
	Small products and equalisers are small limits. We thus need only prove small limits exist if small products and equalisers exist. To this end, let $D: \cat J \to \cat C$ be a diagram.
	Consider the equalizer
	\[ E = \eq(\prod_{i \in \cat J}D(i) \rightrightarrows \prod_{\phi: i \to j}D(j)).\] By \cref{cor:hom-limits-pass-to-set}, the set of morphisms into $E$ can be computed as a limit in set:
	\[
		\Hom_{\cat C}(X, E) \cong \eq(\Hom_{\cat C}(X, \prod_{i \in \cat J}D(i)) \rightrightarrows \Hom_{\cat C}(X, \prod_{\phi: i \to j}D(j))).
	\] Applying \cref{cor:hom-limits-pass-to-set} again to the objects of this equalizer yields and using \cref{rmk:limits-are-products-and-equalisers-in-set} yields
	\[
		\Hom_\cat{C}(X, E) \cong \eq(\prod_{i \in J}\Hom_{\cat C}(X, D(i)) \rightrightarrows \prod_{\phi: i \to j} \Hom_{\cat C}(X, D(j))) \cong \lim_{i \in \cat J}\Hom_{\cat C}(X, D(i)).
	\]
\end{proof}
\begin{cor}
	The colimit of any diagram in $\catSet$ exists, and it is given by \[
    	\coeq(\coprod_{\phi: i\to j} D(i) \rightrightarrows \coprod_{i \in j} D(i))
    \]
\end{cor}

\begin{cor}
	Let $D: \cat J \to \cat C$ be a diagram with $\cat J$ filtered. Then \[
    	\filteredcolim_{i \in \cat J} D(i) = (\coprod_{i \in \cat J} D(i))/_\sim
    \] where $a \sim b$ for $a \in D(i)$ and $b \in D(j)$ if and only if there exist maps $\phi: i \to k$ and $\psi: j \to k$ such that $D(\phi)(a) = D(\psi)(b)$.
\end{cor}\todo{proof of this, some prose.}

\section{Abelian groups}
\begin{lem}
	Every diagram in $\catAbelianGroup$ has a limit and a colimit.
\end{lem}
\begin{proof}
	We saw (co)products and (co)equalisers in the previous lecture. Products of abelian groups are products on the underlying sets with the usual group structure. Coproducts are direct sums. Equalisers are kernels, coequalisers are cokernels: \begin{align*}
	  	\eq(A \mathrel{\mathop{\rightrightarrows}^{\mathrm{f}}_{\mathrm{g}}} B) & = \ker(f - g)\\
		\coeq(A \mathrel{\mathop{\rightrightarrows}^{\mathrm{f}}_{\mathrm{g}}}B) & = \coker(f - g).
    \end{align*} Thus we can use \cref{cor:small-limits-from-eq-prod} to compute limits and colimits from (co)products and (co)equalizers via kernels and cokernels.
\end{proof}
\begin{rmk}
If $\cat J$ is filtered, the colimit is easier: the filtered colimit \[
	\filteredcolim D(i) = (\coprod(i))/_\sim
\] in $\catSet$ gets an abelian group structure as follows: if $a \in D(i)$ and $b \in D(j)$, then choose maps $\phi: i \to k$ and $\psi: j \to k$ (using that $\cat J$ is filtered) and set \[
	a + b =  D(\phi)(a) + D(\psi)(b).
\] One needs to check this is well defined and that it indeed turns the colimit in $\catSet$ to the colimit in $\catAbelianGroup$. This result actually follows from a more general fact: forgetful functors create filtered colimits \cite[Theorem~5.6.5]{riehlCategoryTheoryContext2016} \todo{this reference deals with monadic functors, not immediately obvious it proves our claim. A small discussion of this would be nice.}
\end{rmk}

\section{Limits and colimits of presheaves}

Recall that a presheaf to a category $\cat D$ is a contravariant functor $F \in \catFunctor(\cat C\opp, \cat D)$. In this course the category $\cat D$ has been $\catSet$ or $\catAbelianGroup$, for which we know small limits and colimits exist. The following lemma appears as \cite[Proposition~3.3.9]{riehlCategoryTheoryContext2016} stripped of the context of presheaves. It tells us that if $\cat C$ is small, the work we've done so far for $\catSet$ and $\catAbelianGroup$ suffices to show presheaves into these categories have small limits and colimits. Furthermore, we can compute (co)limits of such presheaves objectwise. We have seen these claims before in \cref{lem:psh-category-set}.

\begin{lem}
	Let $\cat C$ be a small category and $\cat D$ be a category with small limits and colimits.
	Then any diagram $D: \cat J \to \catFunctor(\cat C\opp, \cat D)$ has a limit and a colimit, computed objectwise:
	\begin{align*}
    	(\lim_{i\in \cat J}D(i))(U) &= \lim_{i \in \cat J}(D(i)(U))\\
    	(\colim_{i\in \cat J}D(i))(U) &= \colim_{i \in \cat J}(D(i)(U))
    \end{align*} for all $U \in \cat C$.

\end{lem}
\begin{proof}
	We reproduce the proof given in \cite{riehlCategoryTheoryContext2016}. The idea is to show that $\catFunctor(\cat{C}\opp, \cat D) \to \catFunctor(\cat{C}^\text{disc}, \cat D)$ \todo{Write the details}
\end{proof}

\begin{cor}
	The presheaf categories $\catPresheaf(X)$ and $\catAbelianPresheaf(X)$ have all small limits and colimits.
\end{cor}

\section{Limits and colimits in sheaves}

\begin{thm}
	Any diagram $D: \cat J \to \catSheaf(X)$ (or $D: \cat J \to \catAbelianSheaf(X)$) has a limit, computed in $\catPresheaf(X)$ ($\catAbelianPresheaf(X)$).
	Such a diagram also has a colimit in $\catSheaf(X)$ ($\catAbelianSheaf(X)$), obtained by sheafification of the colimit in $\catPresheaf(X)$ ($\catAbelianPresheaf(X)$).
	% In other words, the forgetful functor $\catSheaf(X) \to \catPresheaf(X)$ reflects limits.
\end{thm}
\todo{Write the rest of the lecture}


\chapter{Additive and abelian categories}
	\remyquote{We can only write down diagrams, prove they commute, and run away as fast as we can.}
\begin{defn}
	A \emph{pre-additive} category $\cat C$ is a locally small category where each hom-set $\Hom_{\cat C}(X, Y)$ are endowed with an abelian group structure.
\end{defn}
\begin{lem}
\begin{enumerate}
	\item If $X \in \cat C$, then the following are equivalent:
	\begin{enumerate}
    	\item $X$ is initial
		\item $X$ is terminal
		\item $\id_X = 0 \in \Hom_{\cat C}(X,X)$
    \end{enumerate}
	\item If $X, Y, Z \in \cat C$ then the following are equivalent:
	\begin{enumerate}
    	\item There are maps $X \xrightarrow{i} Z$ and $Y \xrightarrow{j} Z$ making $Z$ the coproduct.
    	\item There are maps $Z \xrightarrow{p} X$ and $X \xrightarrow{q} Y$ making $Z$ the product.
		\item There are maps $X \rightleftarrows{i}[p] Z \leftrightarrows{j}[q] Y$ such that $p \circ i = \id_X$ and $q \circ j = \id_Y$ and $i\circ p + j \circ q = \id_Z$.
    \end{enumerate}

\end{enumerate}

\end{lem}

\begin{defn}
	A biproduct in a pre-additive category of $X, Y \in \cat C$ is an object $Z$ with maps $X \rightleftarrows{i}[p] Z \leftrightarrows{j}[q] Y$ with $pi = \id_X$, $qj = \id_Y$ and $ip + jq = \id_Z$.
\end{defn}
\begin{rmk}
	Note that these force $q \circ i$ and $p \circ j$ to be zero. To see this it suffices to show that $j \circ q \circ i = 0$, because $j$ is monic.
	This holds because $jqi = (\id_Z - ip)i = i - ipi = i - i = 0$.
\end{rmk}

\begin{proof}(Of the lemma)
	\begin{enumerate}
    	\item If $x$ is initial, then $\Hom_{\cat C}(X, X) = 0$ so $\id_X = 0$.
		Conversely, if $\id_x = 0$ then every map $f: X \to Y$ i n $\cat C$ satisfies $f = f \circ \id_X = f \circ 0 = 0$. So $X$ is initial. This proves (i) is equivalent to (ii) and that (ii) is equivalent to (iii) follows dually.
		\item If $X \rightleftarrows{i}[p] Z \leftrightarrows{j}[q] Y$ is a biproduct we saw $qi = 0$ and $pj = 0$. Then $X \xrightarrow{i} Z$ and $Y \xrightarrow{j} Z$ is a coproduct. If $X \xrightarrow{f} W$ and $Y \xrightarrow{g} W$ is any co cone, set $h = fp + gq: Z \to W$. Then $hi = (fp + gq)i = fpi + gqi = f \circ \id_X + g \circ 0 = f$, and likewise $hj = g$. If $h': Z \to W$ satisfies $h'i = f$ and $h'j = g$ then $h' = h' \circ \id_Z = h' \circ (ip + jq) = h'ip + h'jq = fp + gq = h$. So $X \xrightarrow{i} Z$ and $Y \xrightarrow{j} Z$ form a coproduct. Conversely, if these maps form a coproduct, then the cocones given by ($X \xrightarrow{\id} X$ and $Y \xrightarrow{0} X$) and by ($X \xrightarrow{0} Y$ and $Y \xrightarrow{\id} Y$) define maps $Z \xrightarrow{p} X$, $Z \xrightarrow{q} Y$ with $pi = \id_X, pj = 0, qi = 0, qj = \id_Y$. Then $ip + jq: Z \to Z$ satisfies $(ip + jq)i = ipi + jqi = i \circ \id_X + j \circ 0 = i$, and likewise $(ip + jq)\circ j = \cdots = j$. So $ip + jq = \id_Z$ by the universal property of the coproduct. This proves $(i) \Leftrightarrow (iii)$ and $(ii) \Leftrightarrow$ follows dually.
    \end{enumerate}
\end{proof}
\begin{defn}
	An \emph{additive} category is a pre-additive category with finite products.
\end{defn}
\begin{rmk}
	By the lemma \todo{cite} if $\cat C$ is additive then
	\begin{itemize}
    	\item the terminal object is also initial, hence a zero object $0$.
		\item IF $X, Y \in \cat C$ then the biproduct $X \rightleftarrows{i}[p] X \oplus Y \leftrightarrows{j}[q] Y$ exists, so $\cat C$ has finite coproducts.
		\item If $X_1, \cdots, X_n \in \cat C$ then the map $X_1 \coprod \cdots \coprod X_n \to X_1 \times \cdots \times X_n$ such that $X_i \to X_1 \coprod \cdots \coprod X_n \to X_1 \times \cdots \times X_n \to X_j$ is $\id_X$ if $i = j$ and $0$ if $i \neq j$ is an isomorphism. We write $X_1 \oplus \cdots \oplus X_n$ for the n-ary biproduct.
    \end{itemize}

\end{rmk}
\begin{exc}
	\todo{something about turning map into functor, additional exercises}.
\end{exc}

\begin{defn}
	A \emph{semi-additive} category is a category $\cat C$ with finite products and coproducts such that
	\begin{enumerate}
    	\item the natural map $ \emptyset \to *$ is an isomorphism (so $\cat C$ has a zero object $0$),
		\item for $X_1, ..., X_n$ in $\cat C$ the map $X_1 \coprod \cdots \coprod X_n \to X_1 \times \cdots \times X_n$ is an isomorphism.
    \end{enumerate}
	(Note that we use (i) to define the map in (ii)).
\end{defn}

\begin{rmk}
	While (pre)-additive is a \emph{structure}, being semi-additive is a \emph{property} of a category $\cat C$.
\end{rmk}

\begin{lem}
	If $\cat C$ is semi-additive, then it is canonically enriched in commutative monoids. If $\cat C$ was additive, then this agrees with the given enrichment in abelian groups (under the inclusion $\catAbelianGroup \hookrightarrow \catCommutativeMonoid$).
\end{lem}
\begin{proof}
	For $f, g: X \rightrightarrows Y$ in $\cat C$, define $f + g$ to be
	\[
		X \xrightarrow{\Delta} X \times X = X \oplus X \xrightarrow{f \oplus g} Y \oplus Y = Y \coprod Y \xrightarrow{\triangledown} Y
	\] where $\Delta$ is the diagonal map and $\triangledown$ is the codiagonal.
	For $f, g, h$ maps from $X$ to $Y$, commutativity of \todo{D1 manual notes} shows that $(f+g) + h = f + (g + h)$. For $f: X \to Y$ commutativity of \todo{D2 manual notes} shows $f + 0 = f$ and likewise $ 0 + f = f$.
	 For $f,g: X \rightrightarrows Y$ commutativity of \todo{D3} gives $f + g = g + f$ so have $\Hom_{\cat C}(X,Y)$ is a commutative monoid. If $f,f': X \rightrightarrows Y$ and $g,g':Y  \rightrightarrows Z$ then commutativity of \todo{D4} shows that $g(f+ f') = gf + gf'$ and likewise $(g + g')f = gf + g'f$, proving the first statement.
	If $\cat C$ is additive and $f,g: X \rightrightarrows Y$ then let
	\[
    	X \rightleftarrows{i_1}[p_1] X \oplus X \leftrightarrows{i_2}[p_2] X
    \] be the biproduct, and likewise for $Y$. Then commutativity of \todo{D5} and \todo{D6} gives $f \oplus g = (f \oplus g) \circ \id_{X \oplus X} = (f \oplus g) \circ (i_1 p_1 + i_2 p_2) = i_1 f p_1 + i_2 g p_2$, and likewise $f \oplus 0 = i_1 f p_1$, $0 \oplus g = i_2 g p_2$, so \[
    	f \oplus g = (f \oplus 0) + (0 \oplus g).
    \] Then bilinearity of composition shows that the composite
	\[
    	X \xrightarrow{\Delta} X \oplus X \xrightarrow{f \oplus g} Y \oplus Y \triangledown Y
    \] is $f + g$ (in the enriched sense).
\end{proof}
% What we have learnt from this long proof is that being an additive category is a \emph{property}.
\begin{cor}
	Being additive (i.e. admitting an additive structure) is a property.
\end{cor}
\begin{proof}
	The category $\cat C$ is additive if and only if if it is semi-additive and all $\Hom$ monoids are groups.
\end{proof}

\begin{exmp}
	$\catCommutativeMonoid$ is semi-additive (omitted), but not additive ($\Hom(\bb N, \bb N) = \bb N$ does not have inverses).
\end{exmp}

\section{Abelian categories}
\begin{exc}
	If $\cat C$ is pre-additive, show that \[
    	\eq(X \rightrightarrows{f}[g] = \eq(X \rightrightarrows{(f-g)}[0] Y)) =: \ker(f-g)
    \] (if one of them exists).
\end{exc}
So an additive category $\cat C$ has finite limits (respectively colimits) if and only if it has kernels (respectively cokernels).
\begin{defn}
	A \emph{pre-abelian} category is an additive category with kernels and cokernels.
\end{defn}
\begin{defn}
	Let $\cat C$ be a pre-abelian category and $f: X \to Y$ in $\cat C$. The \emph{image} of $f$ is $\im(f) = \ker(Y \to \coker(f))$. The \emph{coimage} of $f$ is $\coim(f) = \coker(\ker(f) \to X)$.
\end{defn}
Warning: the zoo of names will get worse.
\begin{lem}
	Any $f: X \to Y$ in a pre-abelian category factors uniquely via \[
    	X \to \coim(f) \to \im(f) \to Y.
    \]
\end{lem}
\begin{proof}
	The composition $\ker(f) \to X \to Y$ is $0$. So there is a unique factorisation \[
    	X \to \coim(f) \to Y
    \] of $f$. The composition $\coim(f) \to Y \to \coker(f)$ is $0$ (since this holds after precomposition with $X \to \coim(f)$). So there is a unique factorisation \[
    	\coim(f) \to \im(f) \to Y.
    \]
\end{proof}
And now the moment we have all been waiting for: the definition of an abelian category.

\begin{defn}
	An \emph{abelian} category is a pre-abelian category $A$ such that the $\coim(f) \to \im(f)$ is an isomorphism for all $f: X \to Y$ in $\cat A$.
\end{defn}
\begin{exmp}
	In $\catAbelianGroup$, $\catModule_R$ this is the first isomorphism theorem: $X/\ker(f) \xrightarrow{\sim} \im(f)$.
\end{exmp}
\begin{lem}
	If $\cat A$ is abelian and $\cat C$ is small, then $\catFunctor(\cat C\opp, \cat A)$ is abelian.
\end{lem}
This lemma implies that the category of presheaves of abelian groups on a topological space is an abelian category.

\begin{proof}
	If $\alpha, \beta: F \implies G$ is a natural transformation, then define the sum $\alpha + \beta: F \implies G$ by $(\alpha + \beta)_U = \alpha_U + \beta_U$. This is natural since the maps $F(V) \to F(U)$ and $G(V) \to G(U)$ are group homorphisms for any map $U \to V$ in $\cat C$. This turns $\catFunctor(\cat C\opp, \cat A)$ into a pre-additive category. All the other questions (existence of finite, limits and colimits, the first isomorphism theroem) are checked objectwise.
\end{proof}

\section{Limits and colimits of abelian groups}
\remyquote{Bootstrap, bootstrap, bootstrap.}

\section{Limits and colimits of presheaves}
\remyquote{Maybe let me write down what I just said because it's nice.}

\section{Limits and colimits of sheaves}

\end{document}

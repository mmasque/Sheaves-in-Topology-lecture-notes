\documentclass[../main.tex]{subfiles}

\begin{document}

\chapter{Injective and projective objects, resolutions}

\remyquote{08N5 sounds like a bird flu variant - maybe a my generation joke.}

%Last week, using Iverson's approach, we saw how several diagram lemmas (five and snake lemma) can be concisely proven in abelian categories by first proving the four lemma. The proof of this lemma was somewhat cumbersome however. One work around 

Some more remarks on the four and five lemma which we discussed last week:
\begin{itemize}
    \item We followed the idea in~\cite{IversenCohomologyOfSheaves} to check exactness only \emph{once}.
    \item \emph{Warning}: Checking exactness in an abelian category might be cumbersome. To check $A\to B\to C$ is exact we \emph{cannot} use $\Hom[\cat A](X, -)$ since
    \[\Hom[\cat A](X,A) \to \Hom[\cat A](X, B) \to \Hom[\cat A](X,C)\]
    need not be exact.
    \item An alternative method~\cite[Tag~08N5]{stacks-project}: $A\xrightarrow{f} B\xrightarrow{g} C$ is exact if and only if for any $h\colon X\to B$ with $g\circ h= 0$, there exists an epimorphism $k\colon Y\twoheadrightarrow X$ such that $hk\colon Y\to B$ lifts to $A$: 
    % https://q.uiver.app//#q=WzAsNSxbMCwwLCJZIl0sWzEsMCwiWCJdLFsyLDEsIkMiXSxbMCwxLCJBIl0sWzEsMSwiQiJdLFswLDEsImsiLDAseyJzdHlsZSI6eyJoZWFkIjp7Im5hbWUiOiJlcGkifX19XSxbMSwyLCIwIl0sWzMsNCwiZiJdLFswLDNdLFsxLDRdLFs0LDIsImciXV0=
    \[\begin{tikzcd}
    	Y & X \\
    	A & B & C
    	\arrow["k", two heads, from=1-1, to=1-2]
    	\arrow[from=1-1, to=2-1, dashed]
    	\arrow[from=1-2, to=2-2]
    	\arrow["0", from=1-2, to=2-3]
    	\arrow["f"', from=2-1, to=2-2]
    	\arrow["g"', from=2-2, to=2-3]
    \end{tikzcd}\]
    (Idea: take $Y\twoheadrightarrow X$ to be $A\pullback{\ker g} X$ -- see Additional exercise~10.3 for surjectivity of $Y\twoheadrightarrow X$ since $A\twoheadrightarrow \ker g$.)
    This reduces diagram lemmas to point-set-like statements, but is not easier.
\end{itemize}

The snake lemma was not proven last week, nor will it be this week, for the proof see \cite[\S~\RN{1}.1]{IversenCohomologyOfSheaves}.


\section{Injective and projective objects}
\remyquote{We do what we always do in commutative algebra: you Zorn it.}

\begin{defn}
    An object \(X\) of a category \(\cat C\) is \emph{injective}\index{to categorize!injective objects} (resp.~\emph{projective}) if $\Hom[\cat C](-, X)$ (resp.~$\Hom[\cat C](X,-)$) preserves epimorphisms. Explicitly, given a map $f\colon A\to X$ and a monomorphism $A\hookrightarrow B$, $f$ extends to $B$:
       % https://q.uiver.app//#q=WzAsMyxbMCwwLCJBIl0sWzEsMCwiQiJdLFswLDEsIlgiXSxbMCwyLCJmIiwyXSxbMSwyLCJcXGV4aXN0cyIsMCx7InN0eWxlIjp7ImJvZHkiOnsibmFtZSI6ImRhc2hlZCJ9fX1dLFswLDEsIiIsMCx7InN0eWxlIjp7InRhaWwiOnsibmFtZSI6Imhvb2siLCJzaWRlIjoidG9wIn19fV1d
    \[\begin{tikzcd}
    	A & B \\
    	X
    	\arrow[hook, from=1-1, to=1-2]
    	\arrow["f"', from=1-1, to=2-1]
    	\arrow["\exists", dashed, from=1-2, to=2-1]
    \end{tikzcd}\] if $X$ is injective.
    Dually, $X$ is projective whenever, given an $f\colon X\to B$ and an epimorphism $A\hookrightarrow B$ $f$ factors through $A$:
     % https://q.uiver.app//#q=WzAsMyxbMSwwLCJBIl0sWzEsMSwiQiJdLFswLDEsIlAiXSxbMCwxLCIiLDAseyJzdHlsZSI6eyJoZWFkIjp7Im5hbWUiOiJlcGkifX19XSxbMiwxXSxbMiwwLCIiLDAseyJzdHlsZSI6eyJib2R5Ijp7Im5hbWUiOiJkYXNoZWQifX19XV0=
    \[\begin{tikzcd}
    	& A \\
    	P & B
    	\arrow[two heads, from=1-2, to=2-2]
    	\arrow[dashed, from=2-1, to=1-2]
    	\arrow[from=2-1, to=2-2]
    \end{tikzcd}\]
\end{defn}

Recall that in Additional exercise~10.1(h), (i), we showed that the functors
\[\Hom[\cat A](-, X)\colon \cat A\opp \to \catAbelianGroup\text{,} \quad \Hom[\cat A](X, -)\colon \cat A\to \catAbelianGroup\]
are left exact.
Thus, if $\cat C = \cat A$ is abelian, one way to characterise injective (resp.~projective) objects is via exactness of the corresponding (co)representable functor. Another characterisation of projective objects in abelian categories is the following:
\begin{exc}\todo{reference}
    Any short exact sequence $0\to A \to B \to C \to 0$ splits if and only if $C$ is projective.
\end{exc}

\begin{exmp}[Projective module is summand of free module]
    In $\catLeftModule[R]$ or $\catRightModule[R]$, the free $R$-module $R^{\oplus I}$ is projective:
    \[\leftmoduleHom{R}(R^{\oplus I}, M) \cong M^I, \quad \phi \mapsto (\phi(e_i))_{i\in I}.\]
    If $P$ is projective, there exists $Q$ (which is automatically projective) such that $P\oplus Q \cong R^{\oplus I}$: choose a surjection \(R^{\oplus I}\twoheadrightarrow P\) from a free module (such as \(R^{\oplus P}\twoheadrightarrow P\)).
    Since $P$ is projective, by split exactness of the sequence
    \[\begin{tikzcd}[column sep=scriptsize]
        0 \ar[r] & \ker\pi \ar[r, mono] & R^{\oplus I} \ar[r, "\pi", epi, shift left=1] & P \ar[r] \ar[l, "\sigma", dashed, shift left=1] & 0\text{,}
    \end{tikzcd}\]
    $\pi$ has a section $\sigma$. So $P\oplus Q\cong R^{\oplus I}$ where $Q = \ker \pi = \coker \sigma$.
\end{exmp}

\begin{exmp}
Every projective abelian group is free. The idea is to first prove this for finitely generated abelian groups and then by induction using Zorn's lemma.\todo{source} (This probably also holds for \(R\)-modules over a PID $R$.)
The key step in the proof is: a subgroup of a free abelian group is free. \todo{cite Lang Appendix 2 section 2} 
\end{exmp}

\begin{lem}
    Let $R$ be a ring and $\cat A =  {}_R\catModule$ or $\catModule[R]$. Then $M\in \cat A$ is injective if and only if for every ideal $I\subseteq R$ every $I\to M$ extends to $R\to M$:
    % https://q.uiver.app//#q=WzAsMyxbMCwwLCJJIl0sWzEsMCwiUiJdLFswLDEsIk0iXSxbMCwyXSxbMSwyLCJcXGV4aXN0cyIsMCx7InN0eWxlIjp7ImJvZHkiOnsibmFtZSI6ImRhc2hlZCJ9fX1dLFswLDEsIiIsMCx7InN0eWxlIjp7InRhaWwiOnsibmFtZSI6Imhvb2siLCJzaWRlIjoidG9wIn19fV1d
\[\begin{tikzcd}
	I & R \\
	M
	\arrow[hook, from=1-1, to=1-2]
	\arrow[from=1-1, to=2-1]
	\arrow["\exists", dashed, from=1-2, to=2-1]
\end{tikzcd}\]
\end{lem}
\begin{proof}[Sketch]
    The forwards implication is immediate. Conversely, suppose the condition holds and let $A\hookrightarrow B$ and $f\colon A\to M$ in $\cat A$. Consider the poset,
    \[\{(A', f')\colon A\subseteq A' \subseteq B' \text{ and } f'\colon A' \to M \text{ extends } f\}\]
    ordered by inclusion. This poset satisfies the conditions of Zorn's lemma, so it has a maximal element $(A', f')$. We claim that $A' = B$. Suppose $A' \neq B$ and choose $x\in B\backslash A'$. Consider $\phi\colon R \to B, r\mapsto rx$ and let $I = \phi^{-1}(A')$. Then we obtain a pullback
    % https://q.uiver.app/#q=WzAsNSxbMCwwLCJJIl0sWzEsMCwiUiJdLFsxLDEsIkIiXSxbMCwxLCJBJyJdLFswLDIsIk0iXSxbMCwxLCIiLDAseyJzdHlsZSI6eyJ0YWlsIjp7Im5hbWUiOiJob29rIiwic2lkZSI6InRvcCJ9fX1dLFsxLDIsIlxcdmFycGhpIl0sWzAsMywiXFxwc2kiLDJdLFszLDIsIiIsMix7InN0eWxlIjp7InRhaWwiOnsibmFtZSI6Imhvb2siLCJzaWRlIjoidG9wIn19fV0sWzMsNCwiZiciLDJdLFswLDIsIiIsMSx7InN0eWxlIjp7Im5hbWUiOiJjb3JuZXIifX1dLFsxLDQsImciLDAseyJsYWJlbF9wb3NpdGlvbiI6NzAsInN0eWxlIjp7ImJvZHkiOnsibmFtZSI6ImRvdHRlZCJ9fX1dXQ==
    \[\begin{tikzcd}
    	I & R \\
    	{A'} & B \\
    	M.
    	\arrow[hook, from=1-1, to=1-2]
    	\arrow["\psi"', from=1-1, to=2-1]
    	\arrow["\lrcorner"{anchor=center, pos=0.125}, draw=none, from=1-1, to=2-2]
    	\arrow["\phi", from=1-2, to=2-2]
    	\arrow["g"{pos=0.7}, dotted, from=1-2, to=3-1]
    	\arrow[hook, from=2-1, to=2-2]
    	\arrow["{f'}"', from=2-1, to=3-1]
    \end{tikzcd}\]
    where the map $g$ is the unique extension of $f'\circ \phi$ that we obtain by assumption. But since the sequence
    \[\begin{tikzcd}[column sep=scriptsize]
        0 \ar[r] & I \ar[r, "\begin{pmatrix} \phi \\ -\id\end{pmatrix}"] & A'\oplus R \ar[r, "{(\id, \phi)}"] &A' + Rx \ar[r] & 0
    \end{tikzcd}\]
    is exact, the square
    % https://q.uiver.app/#q=WzAsNCxbMCwwLCJJIl0sWzEsMCwiUiJdLFswLDEsIkEnIl0sWzEsMSwiQScgKyBSeCJdLFswLDEsIiIsMCx7InN0eWxlIjp7InRhaWwiOnsibmFtZSI6Imhvb2siLCJzaWRlIjoidG9wIn19fV0sWzAsMl0sWzIsM10sWzEsM10sWzMsMCwiIiwxLHsic3R5bGUiOnsibmFtZSI6ImNvcm5lciJ9fV1d
\[\begin{tikzcd}
	I & R \\
	{A'} & {A' + Rx}
	\arrow[hook, from=1-1, to=1-2]
	\arrow[from=1-1, to=2-1]
	\arrow[from=1-2, to=2-2]
	\arrow[from=2-1, to=2-2]
	\arrow["\lrcorner"{anchor=center, pos=0.125, rotate=180}, draw=none, from=2-2, to=1-1]
\end{tikzcd}\] is cocartesian. Yet then the morphism $A' + Rx \to M$ induced by $f'$ and $g$ extends $f'$, contradicting maximality.
\end{proof}

\begin{exc}[Homework 6]
An abelian group \(A\) is injective if and only if it is \emph{divisible}: the map $A\to A\text{,}\ a\mapsto na$ is surjective for all $n\in \bb Z_{>0}$.
\end{exc}

\begin{rmk}
If $\cat A$ is a (locally) presentable abelian category, generated by a set $S$ of objects, then an object \(X\) of \(\cat A\) is injective if and only if for every $B\in S$, any extension problem
    % https://q.uiver.app/#q=WzAsMyxbMCwwLCJBIl0sWzEsMCwiQiJdLFswLDEsIlgiXSxbMCwxLCIiLDAseyJzdHlsZSI6eyJ0YWlsIjp7Im5hbWUiOiJob29rIiwic2lkZSI6InRvcCJ9fX1dLFswLDJdLFsxLDIsIiIsMCx7InN0eWxlIjp7ImJvZHkiOnsibmFtZSI6ImRhc2hlZCJ9fX1dXQ==
\[\begin{tikzcd}
	A & B \\
	X
	\arrow[hook, from=1-1, to=1-2]
	\arrow[from=1-1, to=2-1]
	\arrow[dashed, from=1-2, to=2-1]
\end{tikzcd}\] has a solution.
For example, the categories $\catLeftModule[R]$ and $\catRightModule[R]$ of left and right \(R\)-modules are generated by $S = \{R\}$, and the category $\catAbelianSheaf(X)$ of abelian sheaves on a space \(X\) is generated by $\underline{\bb Z}_U = j_{!}\underline{\bb Z}$ for an open subset $j\colon U\hookrightarrow X$.
(But what are the subobjects of $\underline{\bb Z}_U$?)
\end{rmk}

\begin{defn}
    An abelian sheaf \(\mathcal F\) on a space \(X\) is \emph{flasque}\index{sheaf!flasque -} or \emph{flabby} if every diagram
    % https://q.uiver.app/#q=WzAsMyxbMCwwLCJcXHVuZGVybGluZXtcXGJiIFp9X1UiXSxbMSwwLCJcXHVuZGVybGluZXtcXGJiIFp9X1YiXSxbMCwxLCJcXHNoIEYiXSxbMCwxLCIiLDAseyJzdHlsZSI6eyJ0YWlsIjp7Im5hbWUiOiJob29rIiwic2lkZSI6InRvcCJ9fX1dLFsxLDIsIiIsMCx7InN0eWxlIjp7ImJvZHkiOnsibmFtZSI6ImRhc2hlZCJ9fX1dLFswLDJdXQ==
    \[\begin{tikzcd}
    	{\underline{\bb Z}_U} & {\underline{\bb Z}_V} \\
    	{\sh F}
    	\arrow[hook, from=1-1, to=1-2]
    	\arrow[from=1-1, to=2-1]
    	\arrow[dashed, from=1-2, to=2-1]
    \end{tikzcd}\]
    with $U\subset V\subset X$ opens has an extension.
    That is, the restriction $\sh F(V) \twoheadrightarrow \mathcal F(U)$ is surjective for all opens $U\subset V\subseteq X$. 
\end{defn}

\begin{rmk}
    These names are used differently by some authors e.g. in SGA 4$_{\RN{2}}$.
\end{rmk}

\begin{exmp}\label{exmp:injective-sheaves-are-flasque}
    Injective sheaves are flasque.
\end{exmp}

\section{Injective resolutions}

\remyquote{This is why you teach courses, to shade on the literature.}

\begin{defn}
    An abelian category $\cat A$ has \emph{enough injectives}\index{to categorize!has enough injectives} (resp. \emph{enough projectives}) if every object \(X\) of \(\cat A\) admits a monomorphism $X\hookrightarrow I$ with $I$ injective (resp.~an epimorphism $P\twoheadrightarrow X$ with $P$ projective).
\end{defn}

\begin{defn}
    Let $\cat A$ be an abelian category and let \(X\) be an object of \(\cat A\). An \emph{injective resolution} \index{to categorize!injective resolution} (resp. \emph{projective resolution}) of $X$ is an exact sequence
    \[\begin{tikzcd}[column sep=scriptsize]
        0 \ar[r] & X \ar[r] & I^0 \ar[r] & I^1\ar[r] & I^2 \ar[r] & \dots
    \end{tikzcd}\]
    in which every $I^i$ is injective (resp.
    \[\begin{tikzcd}
        \dots \ar[r] & P_2 \ar[r] & P_1 \ar[r] & P_0 \ar[r] & X \ar[r] & 0
    \end{tikzcd}\]
    with all $P_i$ projective).
\end{defn}

In particular as we will prove soon, every object admits an injective (resp. projective) resolution in an abelian category with enough injectives (resp. projectives).

\begin{exmps}
The short exact sequence
    \[\begin{tikzcd}[column sep=scriptsize]
        0 \ar[r] & \bb Z \ar[r, "n\cdot"] & \bb Z \ar[r] & \bb Z/ n\bb Z \ar[r] & 0
    \end{tikzcd}\]
is a projective resolution of the abelian group \(\bb Z/n\bb Z\).

The short exact sequence
    \[\begin{tikzcd}[column sep=scriptsize]
        0\ar[r] & \bb Z \ar[r] & \bb Q \ar[r] & \bb Q / \bb Z \ar[r] & 0
    \end{tikzcd}\]
is an injective resolution of the abelian group \(\bb Z\).
\end{exmps}

\begin{lem}
    If $\cat A$ has enough injective (resp. projectives) then every object \(X\) of \(\cat A\) has an injective (resp. projective) resolution.
\end{lem}
\begin{proof}
    We treat the injective case, the projective case follows dually. Since $\cat A$ has enough injectives, we can find a monomorphism \(X\to I^0\) into an injective object, that is, making the sequence
    \[\begin{tikzcd}[column sep=scriptsize]
        0\ar[r] & X \ar[r] & I^0
    \end{tikzcd}\]
    is exact, covering the base case.
    In the inductive step, assume
    \[\begin{tikzcd}[column sep=scriptsize]
        0 \ar[r] & X \ar[r] & I^0 \ar[r] & I^1 \ar[r] & \dots \ar[r] & I^n
    \end{tikzcd}\]
    is exact. Take $Y \coloneq \coker(I^{n-1} \to I^n)$.
    Since $\cat A$ has enough injectives, there exists an $I^{n+1}$ injective and a monomorphism $Y\hookrightarrow I^{n+1}$. Then the sequence
    % https://q.uiver.app/#q=WzAsNyxbMCwwLCIwIl0sWzEsMCwiWCJdLFsyLDAsIkleMCJdLFszLDAsIlxcZG90cyJdLFs0LDAsIklebiJdLFs2LDAsIklee24rMX0iXSxbNSwxLCJZIl0sWzAsMV0sWzEsMl0sWzIsM10sWzMsNF0sWzQsNV0sWzQsNiwiIiwwLHsic3R5bGUiOnsiaGVhZCI6eyJuYW1lIjoiZXBpIn19fV0sWzYsNSwiIiwxLHsic3R5bGUiOnsidGFpbCI6eyJuYW1lIjoiaG9vayIsInNpZGUiOiJ0b3AifX19XV0=
\[\begin{tikzcd}
	0 & X & {I^0} & \dots & {I^n} && {I^{n+1}} \\
	&&&&& Y
	\arrow[from=1-1, to=1-2]
	\arrow[from=1-2, to=1-3]
	\arrow[from=1-3, to=1-4]
	\arrow[from=1-4, to=1-5]
	\arrow[from=1-5, to=1-7]
	\arrow[two heads, from=1-5, to=2-6]
	\arrow[hook, from=2-6, to=1-7]
\end{tikzcd}\]
is exact since
\begin{align*}
    \im(I^{n-1} \to I^n) &= \ker(I^n \to \coker(I^{n-1} \to I^n)) = \ker(I^n \to Y) = \ker(I^n \to I^{n+1})
\end{align*}
where the third equality follows since postcomposition with a monomorphism does not affect the kernel.
This completes the induction.
\end{proof}

\begin{lem}\label{lem:right-adjoint-to-exact-functor-preserves-injectives}
    Let $\cat A$ and $\cat B$ be abelian categories and let
    \begin{equation*}
      \begin{tikzcd}
        \cat A \ar[r, "F"{name=leftadj}, shift left=2] & \cat B \ar[l, "U"{name=rightadj}, shift left=2]
        \ar[from=leftadj, to=rightadj, phantom, "\leftadj" marking]
      \end{tikzcd}
    \end{equation*}
  be an adjunction where $F$ is exact. Then $U$ preserves injective objects.
\end{lem}
\begin{proof}
    Let $I$ be injective object of \(\cat B\) and let $A \rightarrowtail B$ be a monomorphism in $\cat A$. Then $FA \rightarrowtail FB$ is a monomorphism by exactness of $F$, so the diagram
    % https://q.uiver.app/#q=WzAsNCxbMCwwLCJcXEhvbVtcXGNhdCBBXShCLCBHKEkpKSJdLFsxLDAsIlxcSG9tW1xcY2F0IEFdKEEsIEcoSSkpIl0sWzAsMSwiXFxIb21bXFxjYXQgQl0oRkIsIEkpIl0sWzEsMSwiXFxIb21bXFxjYXQgQl0oRkEsIEkpIl0sWzAsMV0sWzAsMiwiXFxjb25nIiwyXSxbMiwzLCIiLDIseyJzdHlsZSI6eyJoZWFkIjp7Im5hbWUiOiJlcGkifX19XSxbMSwzLCJcXGNvbmciXV0=
\[\begin{tikzcd}
	{\Hom[\cat A](B, GI)} & {\Hom[\cat A](A, GI)} \\
	{\Hom[\cat B](FB, I)} & {\Hom[\cat B](FA, I)}
	\arrow[from=1-1, to=1-2]
	\arrow["\cong"', from=1-1, to=2-1]
	\arrow["\cong", from=1-2, to=2-2]
	\arrow[two heads, from=2-1, to=2-2]
\end{tikzcd}\]
commutes, whence the top map is surjective.
\end{proof}

\begin{defn}
    An abelian category $\cat A$ has \emph{functorial injectives}\index{to categorize!functorial injectives} if there is a functor $F\colon \cat A \to \catFunctor([1], \cat A)$ such that\todo{wording suggests this is property but it is structure!}:
    \begin{enumerate}
        \item $F_0 = \id_{\cat A}$,
        \item each $F_1(X)$ is injective, and
        \item $F_0(X)\hookrightarrow F_1(X)$ for all $X\in \cat A$.
    \end{enumerate}
    We write $X\to I(X)$ for this functor. Dually, one defines the notion of \emph{functorial projectives}. 
\end{defn}

\begin{rmk}
    If $\cat A$ has functorial injectives, then it has functorial injective resolutions. \todo{interpretation?}
\end{rmk}

\begin{thm}
    \begin{enumerate}
        \item\label{thm:modules-functorial-injectives-projectives} The categories $\catLeftModule[R]$ and $\catRightModule[R]$ of left and right \(R\)-modules have functorial injectives and projectives.
        \item\label{thm:presheaf-functorial-injectives-projectives} The category $\catAbelianPresheaf(\cat C)$ of abelian presheaves on a category \(\cat C\) has functorial injectives and projectives.
        \item\label{thm:sheaf-functorial-injectives} The category $\catAbelianSheaf(X)$ of abelian sheaves on a space \(X\) has functorial injectives.
    \end{enumerate}
\end{thm}
\begin{proof}[sketch]
    \begin{enumerate}
        \item Free modules give functorial projectives: $F(M) := R^{\oplus M} \twoheadrightarrow M$. For the injective case, consider the map
        \[(-)^\vee\colon \catLeftModule[R]\opp \to \catRightModule[R],\quad M\mapsto \Hom[\bb Z](M, \bb Q/ \bb Z)\] where $M^\vee$ is a right module via the action,
        \[M^\vee \times R\to M^\vee,\quad (\phi, e)\mapsto (m\mapsto \phi(r\cdot m)).\]
        Note that the functor $(-)^\vee$ is exact since $\bb Q /\bb Z$ is a divisible abelian group. Moreover \[\ev*\colon M\to (M^\vee)^\vee, \quad m\mapsto (\phi \mapsto \phi(m))\] is injective. For suppose $m\neq 0$, if we take $\psi\colon\bb Zm \to \bb Q/ \bb Z$ with $\psi(m)\neq 0$ and extend this to $\phi\colon M\mapsto \bb Q/\bb Z$, then $\ev(m) \neq 0$ since $\phi(m)\neq 0$. Now $F(M^\vee)\twoheadrightarrow M^\vee$ (recall $F$ is a functorial projective) gives
        \[\begin{tikzcd}[column sep=scriptsize]
            M \ar[r, "\ev*", mono] & (M^\vee)^\vee \ar[r, mono] & F(M^\vee)^\vee
        \end{tikzcd}\]
        and we claim that $F(M^\vee)^\vee$ is injective. Indeed, if $S$ is a set, then $F(S)^\vee$ is injective since
        \begin{align*}
            \leftmoduleHom{R}(N, F(S)^\vee) &= \leftmoduleHom{R}(N, \Hom[\bb Z](F(S), \bb Q / \bb Z)) \\
            &\cong \Hom[\bb Z](F(S)\otimes_R N, \bb Q / \bb Z) \tag{\text{\cref{exe:homtensoradjunction}}} \\
            &= \Hom[\bb Z](N^{\oplus S}, \bb Q / \bb Z) \\ 
            &= (N^\vee)^S\text{,}
        \end{align*}
        which we saw was exact.
        \item Let $F$ be an abelian presheaf on a category \(\cat C\). For projectives, use $\bigoplus_{U\in \ob\cat C} \underline{\bb Z}_U^{\oplus F(U)} \twoheadrightarrow F$ (supposedly easy to prove; this not longer works in $\catAbelianSheaf(X)$ since $\underline{\bb Z}_U$ is no longer projective as we will see in Homework~6). For injectives, this is certainly true for $\catAbelianPresheaf(\cat C^{\disc}) \cong \prod_{X\in \ob\cat C} \catAbelianGroup$ by part~\cref{thm:modules-functorial-injectives-projectives}. The inclusion of categories $i\colon\cat C^{\disc} \hookrightarrow \cat C$ gives rise to
        \[i^*\colon \catAbelianPresheaf(\cat C) \to \catAbelianPresheaf(\cat C^{\disc}),\quad F\mapsto F\circ i\] with right adjoint $i_* = \Ran_i$. \todo{reference pullback/pushforward adjunction} Then $i^*$ is exact, so $i_*$ preserves injectives. Thus the functorial injective $i^* F\hookrightarrow I(i^* F)$ gives
        \[\begin{tikzcd}[column sep=scriptsize]
            F \ar[r] & i_*i^* F \ar[r] & i_* I(i^* F).
        \end{tikzcd}\]
        where the first map is the unit of the adjunction \(i^*\leftadj i_*\).
        The unit is monic since $i^*$ is faithful and the second map is since $i_*$ is left exact.
        \item This again holds in $\catAbelianGroup(\cat C^{\disc}) \cong \prod_{x\in X} \catAbelianGroup$ by~\cref{thm:modules-functorial-injectives-projectives}. The inclusion $i\colon X^{\disc}\hookrightarrow X$ gives adjunction
        \begin{equation*}
          \begin{tikzcd}
            \catAbelianGroup(X) \ar[r, "i^*"{name=leftadj}, shift left=2] & \catAbelianGroup(X^{\disc}) \ar[l, "i_*"{name=rightadj}, shift left=2]
            \ar[from=leftadj, to=rightadj, phantom, "\leftadj" marking]
          \end{tikzcd}
        \end{equation*}
        with $i^*$ exact and faithful (look at stalks). Now run the same argument as in part~\cref{thm:presheaf-functorial-injectives-projectives}.\qedhere
    \end{enumerate}
\end{proof}

\begin{exmp}
    If $R = \bb Z$ then $F(S)^\vee = (\bb Z^{\oplus S})^\vee = (\bb Z^\vee)^S = (\bb Q/ \bb Z)^S$.
\end{exmp}

\section*{Exercises}

\todo{if we decide to do this, cite properly in intro w/ permission Remy}

\begin{exe}[Products of injectives are injective]
    Let $\cat A$ be an abelian category. Show that the subcategory of injective objects (resp. projective objects) is closed under products (resp. direct sums) that exist in $\cat A$.
\end{exe}

\begin{exe}[$\Hom$ is right adjoint to the tensor product]\label{exe:homtensoradjunction}
    For rings $A$ and $B$, a $(A,B)$-\emph{bimodule} is an abelian group $M$ which is a left $A$-module and a right $B$-module such that $a(mb) = (am)b$ for all $a\in A, m\in M$ and $b\in B$. Let $M$ be an $(A, B)$-bimodule, let $N$ be a $(B,C)$-bimodule and let $P$ be an $(A,C)$-bimodule. Construct isomorphisms
    \begin{align*}
        \leftmoduleHom{A}(M\otimes_B N, P) &\cong \leftmoduleHom{B}(N, \leftmoduleHom{A}(M, K)) \\ 
        \rightmoduleHom{C}(M\otimes_B N, P) &\cong \rightmoduleHom{B}(M, \rightmoduleHom{C}(N, K))
    \end{align*}
    of $(C, C)$-bimodules and $(A, A)$-bimodules respectively. Restricting to elements on which the two $C$-actions (resp. $A$-actions) agree, deduce isomorphisms of abelian groups
    \[\bimoduleHom{B}{C}(N, \leftmoduleHom{A}(M,K)) \cong \leftmoduleHom{C}(M\otimes_B N, P) \cong \bimoduleHom{A}{B}(M, \rightmoduleHom{C}(N,K))\text{.}\]
\end{exe}

\end{document}
\chapter{Motivation, sheaves and presheaves}

\section{Introduction}
\todo{Not sure if we want to add the examples etc.}


\section{Sheaves and presheaves}

\begin{defn}
    Let $X$ be a topological space. Write \indexterm{$\open(X)$} for the partially ordered set of opens of $X$.  A \indexdefnemph{presheaf} of sets on $X$ is a functor
    \[
        F: \open(X)\opp \to \catSet. 
    \]  
\end{defn}
By changing the codomain we can obtain, for example, presheaves of abelian groups. In this course, we will focus almost entirely on presheaves of sets and of abelian groups.
Let $U \subset V $ be open sets of $X$. The inclusion under $F$ gives a \indexdefnemph{restriction} map $F(V) \to F(U)$. The naming comes from the following example: 

\begin{exmp}\label{exmp:ct-maps-presheaf}
    Let $X$ and $Z$ be topological spaces. The assignment $h_Z: \open(X)\opp \to \catSet$ by $U \mapsto \cont(U,Z)$ can be turned into a presheaf: given $U \subseteq V$ opens of $X$, define $$r_{UV}: \cont(V,Z) \to \cont(U,Z)$$ by $f \mapsto f|_U$.
\end{exmp}
We generalise the notation of function restriction. For $F(V) \to F(U)$, we denote the map pointwise by $s \mapsto s|_U$.

\begin{defn}\label{defn:sheaf}
    Let $X$ be a topological space and $\mathcal{F}$ a presheaf \[\mathcal{F}: \open(X)\opp \to \catSet. \] We call $\mathcal{F}$ a \indexdefnemph{sheaf} if it satisfies the \emph{sheaf condition}\index{sheaf!sheaf condition}:
    for every open $U \subset X$ and every open cover $(U_i)_{i \in I}$ of $U$ with $\bigcup_i U_i = U$, the map 
        \[
            \mathcal{F}(U) \to \prod_{i\in I}\mathcal{F}(U_i)\text{,} \quad s \mapsto (s|_{U_i})_{i \in I}
          \]
    has the following two properties:
    \begin{enumerate}
        \item\label{defn:sheaf:injective} it is injective, and
        \item\label{defn:sheaf:gluing} its image satisfies a \indexdefnemph{gluing condition}: it is given by $\setpred{(s_i)_{i\in I}}{s_i |_{U_i \cap U_j} = s_j | _ {U_i \cap U_j} \forall i,j \in I}$.
    \end{enumerate}
\end{defn}

\begin{rmk}\label{rmk:sheafs-as-equalizers}
    One checks that the sheaf condition equivalent to requiring that
\[\begin{tikzcd}
	{\mathcal{F}(U)} & {\prod_i\mathcal{F}(U_i)} & {\prod_{i,j}\mathcal{F}(U_i \cap U_j)}
	\arrow[from=1-1, to=1-2]
	\arrow["\alpha", shift left, from=1-2, to=1-3]
	\arrow["\beta"', shift right, from=1-2, to=1-3]
\end{tikzcd}\]
    is an \indexterm{equaliser} diagram for all $U$ open in $X$ and for all $(U_i)_{i \in I}$ open covers of $U$ , where $\alpha: (s_i)_{i \in I} \mapsto s_i|_{U_i \cap U_j}$ and $\beta: (s_i)_{i \in I} \mapsto s_j|_{U_i \cap U_j}$.
\end{rmk}

\begin{lem} \label{lem:ct-maps-sheaf}
    The presheaf $h_Z$ from \cref{exmp:ct-maps-presheaf} is a sheaf.
\end{lem}
\begin{proof}
    If two functions agree on every open of a cover of $U$ they agree on $U$, this gives \cref{defn:sheaf}\cref{defn:sheaf:injective}. For~\cref{defn:sheaf:gluing}, we use the pasting lemma.
\end{proof}

\begin{exmp}
    Let $Z$ be a discrete topological space, let $X$ be a topological space. Given an open subset $U$ of $X$, a map $f: U \to Z$ is continuous if and only if it is locally constant. The sheaf $h_Z$ is called the \indexdefnemph{constant sheaf} on the set $Z$, labelled $\constSheaf{Z}$ or $\constSheaf{Z}_X$.
    Explicitly, \(\constSheaf{Z}\) is given by
    \[ \constSheaf{Z} \colon U \mapsto \cont(U,Z)\text{,} \]
    where \(Z\) is endowed with the discrete topology.
\end{exmp}

\begin{exmp}
    If $X$ is a \indexterm{manifold}, then the assignment $U \mapsto C^\infty(U, \mathbb{R})$ is a sheaf of $\mathbb{R}$ vector spaces. One can show that the assignment $U \mapsto \Omega^k(U)$ (smooth differential $k$-forms) is a sheaf.
\end{exmp}

\begin{lem}[name=sheaf of sections]\label{exmp:sheaf-of-sections}
    Let $f: Y \to X$ be a continuous map of topological spaces. The assignment on opens of $X$ given by 
    \[
        h_{Y/X}: U \mapsto \setpred{s: U \to f^{-1}(U) }{ f \circ s = \id_U} =: \cont[X](U, Y)
    \]
    is a sheaf.
\end{lem}
\begin{proof}
    One can prove the above lemma in a similar way we proved \cref{lem:ct-maps-sheaf}. 
    Alternatively, consider the diagram 
    \[\begin{tikzcd}
	{\cont[X](U,Y)} & {\cont(U,Y)} \\
	{\{U \to X\}} & {\cont(U,X)}
	\arrow["{f \circ -}", from=1-2, to=2-2]
	\arrow[from=1-1, to=2-1]
	\arrow[from=2-1, to=2-2]
	\arrow[from=1-1, to=1-2]
	\arrow["\lrcorner"{anchor=center, pos=0.125}, draw=none, from=1-1, to=2-2]
\end{tikzcd}\]
    and check that it is a pullback. We will come back to this in more detail in later lectures. 
\end{proof}

The sheaf \(h_{Y/X}\) of the lemma is called the \emph{sheaf of sections}\index{sheaf!of sections}.
The example in the lemma above is why the elements of $\mathcal{F}(U)$ for an arbitrary sheaf \(\mathcal F\) are more generally also called \emph{sections}\index{section}. 

\begin{exmp}
    Let $Y \xrightarrow{f} X$ be the two-to-one cover of the circle\index{circle covering}: $f: z \mapsto z^2$, with $X := S^1$ and $Y:= S^1$. 
    On small intervals $U \subset X$ we get $f^{-1}(U)  \cong U \times \{1,2\}$. We thus have two sections: $U \mapsto (U, 1)$ and $U \mapsto (U, 2)$. So $h_{Y/X}(U)$ has two elements
    On $V$ a union of small intervals, we get $2^{|\pi_0(V)|}$ elements, where $|\pi_0(V)|$ is the number of path components of $V$. 
    On $W = X$, we get no sections. If $s: X \to Y$ is a section, then the induced map $s_*: \pi_1(X) \to \pi_1(Y)$ is a section to the map $f_*: \pi_1(Y) \to \pi_1(X)$. But this induced map is multiplication by $2$, and it does not have a section. 
\end{exmp}
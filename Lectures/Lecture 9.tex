\chapter{Monomorphisms and epimorphisms, chain complexes, exact sequences}

In the last lecture, we defined abelian categories as \(\catAbelianGroup\)-enriched categories \(\cat A\) with finite limits and colimits in which the first isomorphism theorem holds (the natural map \(\coim f\to\im f\) is an isomorphism for all \(f\colon A\to B\) in \(\cat A\)).
We also showed that the category \(\catAbelianPresheaf(X)\) of abelian presheaves on a space \(X\) is an abelian category.

Today we will show that also the category \(\catAbelianSheaf(X)\) of abelian sheaves on \(X\) is an abelian category, and we will give a description of monomorphisms and epimorphisms.

\section{Abelian category of abelian sheaves}

\begin{lem}\label{lem:pullback-preserves-colimits-and-finite-limits}
Let \(f\colon Y\to X\) be a continuous map.
Then the pullback functors \(f^*\colon\catSheaf(X)\to\catSheaf(Y)\) and \(f^*\colon\catAbelianSheaf(X)\to\catAbelianSheaf(Y)\) preserve finite limits and all colimits.
\end{lem}
\begin{proof}
Since the pullback functor \(f^*\) is left adjoint to the pushforward functor \(f_*\) (\cref{lem:pushforward-pullback-adjunction-sheaves}), it preserves colimits.

For finite limits, there are two methods:
\begin{enumerate}
\item Use that \(f^*\) is given by
  \[\catLocalHomeomorphism_{/X}\to\catLocalHomeomorphism_{/Y}\text{,} \quad (Z\to X) \mapsto \bigl(Z\pullback{X}Y\to Y\bigr)\text{,} \]
  which preserves finite limits, and finite limits in \(\catLocalHomeomorphism_{/-}\) are computed as in \(\catTopologicalSpace_{/-}\).\footnote{This does not hold for all limits; it fails for instance for infinite products, already when \(X\) is a point.}
\item Use that
  \[(f^\circledast\mathcal F)(U) = \colim_{f(U)\subseteq V}\mathcal F(V) \]
  is a filtered colimit, and filtered colimits in \(\catSet\) and \(\catAbelianGroup\) commute with finite limits (Additional exercise~9.1).
  Check that sheafification preserves finite limits (Homework~5).
  \qedhere
\end{enumerate}
\end{proof}

\begin{prop}
If \(X\) is a topological space, then the category \(\catAbelianSheaf(X)\) of abelian sheaves on \(X\) is an abelian category.
\end{prop}
\begin{proof}
Being a full subcategory of the abelian category \(\catAbelianPresheaf(X)\), the category \(\catAbelianSheaf(X)\) is pre-additive, and we have already shown in \cref{thm:limits-colimits-sheaves} that it has all limits and colimits.

It remains to check that the first isomorphism theorem holds in \(\catAbelianSheaf(X)\).
Let \(f\colon\mathcal F\To\mathcal G\) be a map in \(\catAbelianSheaf(X)\).
We can factor \(f\) as
\begin{equation}\label{eq:factorisation-coimage-image}
  \mathcal F \To \coim f \To \im f \To \mathcal G\text{.}
\end{equation}
By \cref{lem:pullback-preserves-colimits-and-finite-limits}, the formation of the kernel, cokernel, image and coimage commutes with the stalk functor \(i_x^*\) for all \(i_x\colon\{x\}\to X\) (see \cref{defn:stalk}).
So \cref{eq:factorisation-coimage-image} induces the factorisation
\[ \mathcal F_x \to (\coim f)_x = \coim f_x \to (\im f)_x = \im f_x \to \mathcal G_x \]
of the map \(f_x\) induced on stalks.
The map \(\coim f_x\to\im f_x\) is an isomorphism since \(\catAbelianGroup\) is an abelian category.
Since this holds for all stalks, the map \(\coim f\to\im f\) is also an isomorphism by \cref{lem:stalkwise-check-isomorphism}.
\end{proof}

\section{Monomorphisms and epimorphisms}

\begin{lem}
Let \(\cat A\) be an abelian category and \(f\colon A\to B\) a map in \(\cat A\).
Then:
\begin{enumerate}
\item \(f\) is monic if and only if \(\ker f = 0\);
\item \(f\) is epic if and only if \(\coker f = 0\).
\end{enumerate}
\end{lem}

The recipe of the proof is: Yoneda + the same statement in \(\catAbelianGroup\).

\begin{proof}
\begin{enumerate}
\item
  By definition, \(\ker f\) represents \(\ker(\Hom[\cat A](-,A)\To\Hom[\cat A](-,B))\).
  So \(\ker f = 0\) if and only if \(\Hom[\cat A](-,A)\To\Hom[\cat A](-,B)\) is an injective map of presheaves, that is, \(f\colon A\to B\) is monic.
\item Follows dually. \qedhere
\end{enumerate}
\end{proof}

For sheaves, let's make it quite concrete.

\begin{lem}\label{lem:monomorphisms-sheaves}
Let \(f\colon\mathcal F\To\mathcal G\) be a map in \(\catSheaf(X)\) (resp.~\(\catAbelianSheaf(X)\)).
Then the following are equivalent:
\begin{enumerate}
\item\label{lem:monomorphisms-sheaves:monic-sheaves} \(f\) is monic (in \(\catSheaf(X)\) resp.~\(\catAbelianSheaf(X)\));
\item\label{lem:monomorphisms-sheaves:monic-presheaves} \(f\) is monic in \(\catPresheaf(X)\) (resp.~\(\catAbelianPresheaf(X)\));
\item\label{lem:monomorphisms-sheaves:injective-open} \(f_U\) is injective for all open subsets \(U\subseteq X\);
\item\label{lem:monomorphisms-sheaves:injective-stalk} \(f_x\) is injective for all points \(x\in X\);
\item\label{lem:monomorphisms-sheaves:injective-etale-space} \(\etalespace(f)\to X\) is injective.
\end{enumerate}
\end{lem}
\begin{proof}
By Additional exercise~7.3, we know that \(f\) is monic if and only if the diagram
\begin{equation*}
  \begin{tikzcd}[arrows=Rightarrow]
    \mathcal F \ar[r, "\id"] \ar[d, "\id"'] & \mathcal F \ar[d, "f"] \\
    \mathcal F \ar[r, "f"'] & \mathcal G
  \end{tikzcd}
\end{equation*}
is a pullback square.
So the equivalence of \cref{lem:monomorphisms-sheaves:monic-sheaves} and \cref{lem:monomorphisms-sheaves:monic-presheaves} follows since the inclusion \(\catSheaf(X)\hookrightarrow\catPresheaf(X)\) (resp.~\(\catAbelianSheaf(X)\hookrightarrow\catAbelianPresheaf(X)\)) creates limits (\cref{thm:limits-colimits-sheaves}), and the equivalence of \cref{lem:monomorphisms-sheaves:monic-presheaves} and \cref{lem:monomorphisms-sheaves:injective-open} since limits in \(\catPresheaf(X)\) (resp.~\(\catAbelianPresheaf(X)\)) are computed objectwise.

The functors
\begin{equation*}
  \begin{tikzcd}
    \catSheaf(X) \ar[r, "\simeq"] & \catLocalHomeomorphism_{/X} \ar[r, inclusion] & \catTopologicalSpace_{/X} \ar[r, inclusion] & \catSet_{/X}
  \end{tikzcd}
\end{equation*}
(and the corresponding functors for abelian sheaves) create fibre products, so they preserve and reflect monomorphisms.
Since monomorphisms in \(\catSet_{/X}\) (resp.~\(\catAbelianGroup(\catSet_{/X})\)) are injective maps, this proves that \cref{lem:monomorphisms-sheaves:monic-sheaves} is equivalent to \cref{lem:monomorphisms-sheaves:injective-etale-space}.

Finally, \cref{lem:monomorphisms-sheaves:injective-stalk} is equivalent to \cref{lem:monomorphisms-sheaves:injective-etale-space} since the fibres of \(\etalespace(\mathcal F)\to\etalespace(\mathcal G)\) over \(x\in X\) are the stalks \(f_x\).
\end{proof}

The epimorphisms of sheaves are more subtle.
Although the epimorphisms of presheaves are just objectwise epimorphisms (characterisation~\cref{lem:monomorphisms-sheaves:injective-open} of the last lemma), this is not true for epimorphisms of sheaves.
This is perhaps to be expected: monomorphisms are related to limits (as discussed in the last proof), and the forgetful functor from sheaves to presheaves creates limits, but epimorphisms are related to colimits (in a dual way, made precise in the following proof), whose relation to colimits of presheaves is not so straightforward.

\begin{lem}\label{lem:epimorphisms-sheaves}
Let \(f\colon\mathcal F\To\mathcal G\) be a map in \(\catSheaf(X)\) (resp.~\(\catAbelianSheaf(X)\)).
Then the following are equivalent:
\begin{enumerate}
\item\label{lem:epimorphisms-sheaves:epic-sheaves} \(f\) is epic (in \(\catSheaf(X)\) resp.~\(\catAbelianSheaf(X)\));
\item\label{lem:epimorphisms-sheaves:locally-lift-section} for every open subset \(U\subseteq X\) and every section \(t\in\mathcal G(U)\), there exists an open cover \(U=\bigcup_{i\in I}U_i\) and sections \(s_i\in\mathcal F(U_i)\) such that \(f(s_i) = t|_{U_i}\);
\item\label{lem:epimorphisms-sheaves:surjective-stalk} \(f_x\) is surjective for all points \(x\in X\);
\item\label{lem:epimorphisms-sheaves:surjective-etale-space} \(\etalespace(f)\to X\) is surjective.
\end{enumerate}
In \(\catAbelianSheaf(X)\), these statements are also equivalent to:
\begin{enumerate}[resume]
\item\label{lem:epimorphisms-sheaves:surjective-etale-space} the sheafification of the presheaf cokernel is zero.
\end{enumerate}
\end{lem}
\begin{proof}
Again, \(f\) is epic if and only if the diagram
\begin{equation*}
  \begin{tikzcd}[arrows=Rightarrow]
    \mathcal F \ar[r, "f"] \ar[d, "f"'] & \mathcal G \ar[d, "\id"] \\
    \mathcal G \ar[r, "\id"'] & \mathcal G
  \end{tikzcd}
\end{equation*}
is a pushout square.
If this holds, it holds for all stalks \(f_x\colon\mathcal F_x\to\mathcal G_x\) since \(i_x^*\) preserves colimits by \cref{lem:pullback-preserves-colimits-and-finite-limits}.
This proves that \cref{lem:epimorphisms-sheaves:epic-sheaves} implies \cref{lem:epimorphisms-sheaves:surjective-stalk}, and the equivalence of \cref{lem:epimorphisms-sheaves:surjective-stalk} and \cref{lem:epimorphisms-sheaves:surjective-etale-space} is clear since surjectivity in \(\catTopologicalSpace_{/X}\) is checked fibrewise.
Conversely, that \cref{lem:epimorphisms-sheaves:surjective-etale-space} implies \cref{lem:epimorphisms-sheaves:epic-sheaves} follows from the equivalence \(\catSheaf(X)\simeq\catLocalHomeomorphism_{/X}\) (resp.~\(\catAbelianSheaf(X)\simeq\catAbelianGroup(\catLocalHomeomorphism_{/X})\)) since a surjection in \(\catLocalHomeomorphism_{/X}\) is surely an epimorphism\footnote{This holds in any \emph{concrete category} \(\cat C\), a category with a faithful functor \(\cat C\to\catSet\), since faithful functors reflect epimorphisms.}.

The equivalence of \cref{lem:epimorphisms-sheaves:locally-lift-section} and \cref{lem:epimorphisms-sheaves:surjective-stalk} is an unwinding of the definitions.
Statement~\cref{lem:epimorphisms-sheaves:locally-lift-section} means that for all \(x\in U\) and all \(t\in\mathcal G(U)\) there exists \(x\in U'\subseteq U\) and \(s\in\mathcal F(U)\) such that \(f(s) = t|_{U'}\), which is \cref{lem:epimorphisms-sheaves:surjective-stalk}.

In the abelian case, the equivalence of \cref{lem:epimorphisms-sheaves:epic-sheaves} and \cref{lem:epimorphisms-sheaves:surjective-etale-space} follows from the construction of the sheaf cokernel as the sheafification of the presheaf cokernel.
\end{proof}

\begin{defn}
A map in an abelian category is called \emph{injective} if it is a monomorphism and \emph{surjective} if it is an epimorphism.
\end{defn}

\todo{warning sign?}

\begin{rmk}
If \(f\colon\mathcal F\To\mathcal G\) is a surjective map of abelian sheaves, then the component \(\mathcal F(U)\to\mathcal G(U)\) need not be surjective.

In Homework~3, we constructed a surjection \(f\colon\mathcal F\To i_*i^*\mathcal F\) for any closed inclusion \(i\colon Z\hookrightarrow X\) and any abelian sheaf \(\mathcal F\) on \(X\).
Take \(X=\bb R\), \(Z = \{0,1\}\subseteq\bb R\) and \(F=\constSheaf{\bb Z}\).
Then \(i^*\mathcal F = \constSheaf{\bb Z}\) (pullbacks of constant sheaves are constant sheaves) and \(\constSheaf{\bb Z}\to i_*\constSheaf{\bb Z}\) is surjective.
But plugging in \(\bb R\) gives the map
\[\constSheaf{\bb Z}(\bb R) = \bb Z \to (i_*\constSheaf{\bb Z})(\bb R) = \constSheaf{\bb Z}(\bb R\cap\{0,1\}) = \bb Z\oplus\bb Z\text{,} \quad a \mapsto (a,a)\text{.} \]
which is not surjective.
However, any \((a,b)\in\bb Z\oplus\bb Z\) can be lifted: the open sets \(U_0\coloneq(-\infty,1)\), \(U_1\coloneq(0,\infty)\) cover \(\bb R\) and \((a,b)\) locally lifts to \(s_0\coloneq a\) and \(s_1\coloneq b\), but they do not glue.
\todo{Picture}
(Another example is on Homework~5.)
\end{rmk}

\section{Exact sequences}

\begin{lem}\label{lem:mono-epi-abelian-category}
Let \(\cat A\) be an abelian category and \(f\colon A\to B\) a map in \(\cat A\).
Then:
\begin{enumerate}
\item\label{lem:mono-epi-abelian-category:i} The map \(\ker f\to A\) (resp.~\(B\to\coker f\)) is monic (resp.~epic), and an isomorphism if \(f=0\).
\item\label{lem:mono-epi-abelian-category:ii} The map \(\im f\to B\) (resp.~\(A\to\coim f\)) is monic (resp.~epic), and an isomorphism if \(f\) is epic (resp.~monic).
\item\label{lem:mono-epi-abelian-category:iii} \(\ker f\) is canonically isomorphic to \(\ker(A\to\coim f)\), and \(\coker f\) is canonically isomorphic to \(\coker(\im f\to B)\).
\item\label{lem:mono-epi-abelian-category:iv} If \(f\) is monic (resp.~epic), then \(A\cong\ker(B\to\coker f)\) (resp.~\(B\cong\coker(\ker f\to A)\)).
  (In particular, any monomorphism (resp.~epimorphism) is an equaliser (resp.~coequaliser), hence a \emph{regular} monomorphism (resp.~epimorphism).)
\item\label{lem:mono-epi-abelian-category:v} If \(f\) is both monic and epic, then it is an isomorphism.
  (\(\cat A\) is a \emph{balanced} category.)
\end{enumerate}
\end{lem}
\begin{proof}
\begin{enumerate}
\item
  The universal property of \(\ker f\) is
  \[\Hom[\cat A](C,\ker f) \cong \setpred{\phi\in\Hom[\cat A](C,A)}{f\phi = 0}\text{,} \]
  so \(\Hom[\cat A](C,\ker f)\to\Hom[\cat A](C,A)\) is injective for all \(C\), hence \(\ker f\to A\) is monic.
  If \(f=0\), the condition \(f\phi=0\) on \(\phi\) is vacuous, so \(\Hom[\cat A](C,\ker f)\cong\Hom[\cat A](C,A)\).
  The statements about \(B\to\coker f\) follow dually.
\item
  Apply \cref{lem:mono-epi-abelian-category:i} to \(B\to\coker f\) (resp.~\(\ker f\to A\)), which we saw is zero if \(f\) is epic (resp.~monic).
\item
  We have a factorisation
  \begin{equation*}
    \begin{tikzcd}[column sep=scriptsize]
      A \ar[r, "\pi", epi] & \coim f \ar[r, iso'] & \im f \ar[r, mono] & B
    \end{tikzcd}
  \end{equation*}
  of \(f\) by \cref{lem:mono-epi-abelian-category:ii}, so the map \(\coim f\to B\) is monic.
  Then for any \(\phi\colon C\to A\), the composition \(\pi\phi=0\) if and only if \(f\phi=0\), so \(\ker\pi=\ker f\).
  The other statement follows dually.
\item
  If \(f\) is monic, then \(A\cong\coim f\cong\im f\) by \cref{lem:mono-epi-abelian-category:ii}.
  The other statement follows dually.
\item
  By \cref{lem:mono-epi-abelian-category:ii}, all maps
  \begin{equation*}
    \begin{tikzcd}[column sep=scriptsize]
      A \ar[r] & \coim f \ar[r, iso'] & \im f \ar[r] & B
    \end{tikzcd}
  \end{equation*}
  are isomorphisms.
  \qedhere
\end{enumerate}
\end{proof}

\begin{defn}
A \emph{cochain complex} in an additive category \(\cat C\) is a diagram
\begin{equation*}
  \begin{tikzcd}
    \ldots \ar[r] & C^{i-1} \ar[r, "d^{i-1}"] & C^i \ar[r, "d^i"] & C^{i+1} \ar[r] & \ldots
  \end{tikzcd}
\end{equation*}
in \(\cat C\) such that \(d^{i+1}d^i=0\) for all \(i\in\bb Z\).
(If it is only defined on a subset \(I\subseteq\bb Z\), we set \(C^i\coloneq 0\) for \(i\notin I\).)

A cochain complex \(C^\bullet\) is \emph{exact} at \(C^i\) if the canonical map \(\im d^{i-1}\to\ker d^i\) is an isomorphism.

A \emph{short exact sequence} is an exact complex \(0\to A\to B\to C\to 0\).

A \emph{cochain map} \(f^\bullet\colon C^\bullet\to D^\bullet\) of cochain complexes is a natural transformation in \(\catFunctor(\bb Z,\cat C)\):
\begin{equation*}
  \begin{tikzcd}
    \ldots \ar[r] & C^{i-1} \ar[r, "d^{i-1}"] \ar[d, "f^{i-1}"'] & C^i \ar[r, "d^i"] \ar[d, "f^i"] & C^{i+1} \ar[r] \ar[d, "f^{i+1}"] & \ldots \\
    \ldots \ar[r] & D^{i-1} \ar[r, "d^{i-1}"'] & D^i \ar[r, "d^i"'] & D^{i+1} \ar[r] & \ldots
  \end{tikzcd}
\end{equation*}
\end{defn}

The following lemma shows that our previous \emph{ad hoc} notion of exactness of \cref{defn:short-exact-sequence-sheaves} agrees with the categorical notion.

\begin{lem}
Let \(\mathcal F^\bullet\) be a cochain complex in \(\catAbelianSheaf(X)\).
Then \(\mathcal F^\bullet\) is exact at \(\mathcal F^i\) if and only if \(\mathcal F^\bullet_x\) is exact at \(\mathcal F^i_x\) is exact for all \(x\in X\).
\end{lem}

\begin{proof}
We saw in \cref{lem:pullback-preserves-colimits-and-finite-limits} that \(i_x^*\colon\catAbelianSheaf(X)\to\catAbelianGroup\) preserves all finite limits and colimits, so in particular the kernel, cokernel, image and coimage.
Thus, we conclude since \(\im d^{i-1}\To\ker d^i\) is an isomorphism if and only if \((\im d^{i-1})_x\to(\ker d^i)_x\) is an isomorphism for all \(x\in X\) by \cref{lem:stalkwise-check-isomorphism}.
\end{proof}

%%% Local Variables:
%%% mode: latex
%%% TeX-master: "../main"
%%% End:

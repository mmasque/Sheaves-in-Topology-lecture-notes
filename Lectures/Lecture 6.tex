\documentclass[../main.tex]{subfiles}

\begin{document}

\chapter{Sheaves of abelian groups}

\section{Sheaves of abelian groups}
\remyquote{Why is he erasing an empty board, you might ask.}
\noindent
Almost everything we have done so far works for sheaves with algebraic structure, for example:
\begin{itemize}
    \item Sheaves of abelian groups ($\rightsquigarrow$ homological algebra)
    \item Sheaves of commutative rings (algebraic geometry)
    \item Sheaves of commutative monoids (useful in logarithmic geometry (?))
\end{itemize}
You can also do more general things:
\begin{itemize}[resume]
    \item Sheaves of topological rings
    \item Sheaves of Banach algebras
    \item Sheaves of categories (\emph{stacks in categories})
    \item Sheaves of groupoids (\emph{stacks})
    \item Sheaves of homotopy types/spaces/anima/\(\infty\)-groupoids (\emph{$\infty$-stacks})
\end{itemize}
We should warn that the sheaf condition should be modified for higher-categorical sheaves.

\begin{defn}
    Let $\cat C$ be a category. A \emph{presheaf on $X\in\catTopologicalSpace$ with values in $\cat C$} is a functor $F\colon \open(X)\opp \to \cat C$. If $\cat C$ is complete (or at least the products of the sheaf condition), then a sheaf on $X$ with values in $\cat C$ is a presheaf such that the sheaf condition holds in $\cat C$. 
    
    We denote the categories of such objects $\catPresheaf(X, \cat C)$ and $\catSheaf(X, \cat C)$ respectively. In the case $\cat C$ is $\catAbelianGroup$, we may use the notation $\catAbelianPresheaf(X) := \catPresheaf(X, \catAbelianGroup)$ and $\catAbelianSheaf(X) := \catSheaf(X, \catAbelianGroup)$.
\end{defn}

We will need a description of limits in $\catAbelianGroup$. 

\begin{lem}[name={\cite[Section~\RN{5}.1]{maclane:71}}]
    The forgetful functor $\catAbelianGroup\to \catSet$ creates limits.
\end{lem}

\begin{exmp}
The product of a family \((A_i)_{i\in I}\) of abelian groups is given by the set $\prod_{i\in I} A_i$ with coordinatewise addition.
Equalisers in $\catAbelianGroup$ (and in fact in any additive category with all kernels) are given by
\[\eq(f,g) = \setpred{a\in A}{f(a) = g(a)} = \ker(f-g)\text{.}\] 

However, colimits are not the same as in $\catSet$.
The coproduct of a family \((A_i)_{i\in I}\) of abelian groups is the direct sum
\[\bigoplus_{i\in I} A_i = \setpred{(a_i)_{i\in I} \in \prod_{i\in I} A_i}{\#\setpred{i\in I}{a_i\neq 0}<\infty}\text{,}\]
and coequalisers in $\catAbelianGroup$ are given by $\coeq(f,g) = \coker(f-g)$.
\end{exmp}

\begin{cor}\label{cor:sheaf-condition-forgetful-functor}
A presheaf $F\colon \open(X)\opp \to \catAbelianGroup$ is a sheaf if and only if the composite
\[ \open(X)\opp\xrightarrow{F} \catAbelianGroup \xrightarrow{U}\catSet \]
is a sheaf, where $U\colon \catAbelianGroup\to\catSet$ is the forgetful functor.
\end{cor}

\begin{defn}\label{defn:filtered-category}
    A small category $\cat I$ is \emph{filtered} if $\cat I\neq \varnothing$ and the following conditions hold:
    \begin{enumerate}
        \item\label{defn:filtered-category:upperbound} For all $i,j\in \cat I$ there exists a $k\in\cat I$ and arrows $i\to k \leftarrow j$
        \item\label{defn:filtered-category-equaliser} For $u,v\colon i\to j$ in $\cat I$ there exists a $k\in \cat I$ and $w\colon j\to k$ such that $wu = wv$.
    \end{enumerate}
    Dually, there is a notion of \emph{cofiltered category}.
\end{defn}

\begin{exmp}
    A poset $P$ always satisfies \cref{defn:filtered-category-equaliser}, so it is filtered if and only if it is nonempty and every pair of elements has an upper bound.
    If $I$ is a set, then the poset $\mathcal P_{\textup{fin}}(I) := \setpred{J\subseteq I}{I\text{ finite}}$ of finite subsets of \(I\) ordered by inclusion is filtered, since $\mathcal P_{\textup{fin}}(I)$ contains the empty set and two finite subsets $J, J'\subseteq I$ have an upper bound $J\cup J'$. 
\end{exmp}

\begin{exc}
    Show that a small category $\cat I$ is filtered if and only if every finite diagram $D\colon \cat J \to \cat I$ in $\cat I$ has a cocone.
\end{exc}

\begin{defn}[Only the notation was introduced in the lecture]
    A \emph{filtered colimit}, denoted $\filteredcolim$, is the colimit of a functor $F\colon \cat I\to \cat C$ where $\cat I$ is a filtered category. A \emph{cofiltered limit}, denoted $\cofilteredlim$, is the limit of a functor $F\colon \cat I\to \cat C$ where $\cat I$ is a cofiltered category.
\end{defn}

\begin{lem}[name={\cite[Section~\RN{9}.1]{maclane:71}}]
    The forgetful functor $U\colon \catAbelianGroup\to\catSet$ creates filtered colimits.
\end{lem}

\begin{exmp} The following are examples of filtered limits and colimits, in some cases the indexing category is omitted in the notation.
\begin{itemize}
    \item We saw that \[\bigoplus_{i\in I} A_i \cong \filteredcolim_{J\in \cat P_{\textup{fin}}(I)} \oplus_{j\in J} A_j\text{.}\]
    \item The filtered colimit of the rings \(\frac{1}{n}\bb Z\) where the diagram is indexed by the divisibility poset is
      \[\filteredcolim_{n}\frac{1}{n} \bb Z = \bb Q\text{.}\]
    \item The filtered colimit of the field extensions of \(\bb Q\) is
      \[\filteredcolim_{\bb Q\to K \textup{ finite}} K = \overline{\bb Q}\text{.}\]
    \item The filtered limit of the rings \(\bb Z/p^n\bb Z\) for \(p\) a prime is the ring
      \[\cofilteredlim_{n} \bb Z / p^n\bb Z = \bb Z_p\]
      of the \(p\)-adic integers.
    \item The stalk\index{stalk} of a sheaf \(\mathcal F\) at a point \(x\) is a filtered colimit:
      \[ \cat F_x = \filteredcolim_{U\ni x} \cat F(U)\text{.} \]
      Indeed the indexing category $\setpred{U\in\open(X)\opp}{x\in U}$ is filtered: if $x\in U, V \subseteq X$ are open then $x\in U\cap V$.
    \item Likewise, the pullback of a presheaf \(\mathcal F\) is given by
      \[ f^{\circledast} \mathcal F(V) = \filteredcolim_{f(V) \subseteq U} \cat F(U)\text{,} \]
      a filtered colimit.
\end{itemize}
    
\end{exmp}

\section{Abelian group objects}
\remyquote{I'm only familiar with this for \(\infty\)-categories, not for \(1\)-categories.}
\noindent
\begin{defn}
    Let $\cat C$ be a category with finite products. An \emph{abelian group object} in $\cat C$ is a quadruple $(A, m, i, 0)$ of an object $A\in \cat C$ and morphisms
    \begin{align*}
        m\colon A\times A\to A\text{,} \quad i\colon A\to A\text{,} \quad 0\colon * \to A
    \end{align*}
    such that the following diagrams commute:
    \begin{enumerate}
        \item Associativity:
            \[\begin{tikzcd}
            A\times A\times A \arrow[r, "m\times \id"] \arrow[d, "\id\times m"'] & A\times A \arrow[d, "m"] \\
            A\times A \arrow[r, "m"']                                            & A                       
            \end{tikzcd}\]
        \item Commutativity:
            \[\begin{tikzcd}
            A\times A \arrow[rd, "m"'] \arrow[rr, "\operatorname{swap}"] &   & A\times A \arrow[ld, "m"] \\ 
            & A & 
            \end{tikzcd}\]
        \item Identity:
            \[\begin{tikzcd}
            A\times * \arrow[r, "\id\times 0"] \arrow[d, "\pr*_1"'] & A\times A \arrow[d, "m"] & *\times A \arrow[l, "0\times\id"'] \arrow[d, "\pr*_2"] \\
            A \arrow[r, "\id"']                                          & A                        & A \arrow[l, "\id"]                                        
            \end{tikzcd}\]
        \item Inverse:
            \[\begin{tikzcd}
            * \arrow[d, "0"'] &                           & A \arrow[d, "\Delta"] \arrow[rr] \arrow[ll]                           &                          & * \arrow[d, "0"] \\
            A                & A\times A \arrow[l, "m"] & A\times A \arrow[r, "\id \times i"'] \arrow[l, "i\times \id"] & A\times A \arrow[r, "m"'] & A               
            \end{tikzcd}\]
    \end{enumerate}
    Write $\catAbelianGroup(\cat C)$ for the \emph{category of abelian group objects} in $\cat C$.
    A map \(f\colon(A,m,i,0)\to (B,m',i',0')\) in \(\catAbelianGroup(\cat C)\) is a map \(f\colon A\to B\) in \(\cat C\) such that \(f\circ m=m'\circ(f\times f)\), that is, such that \(f\) commutes with the group operation \(m\).
\end{defn}

Using the diagrams above, one can show that a map of abelian group objects in \(\cat C\) also commutes with inverses and the unit.

\begin{rmk}
The category \(\catAbelianGroup\) of abelian groups is precisely the category \(\catAbelianGroup(\catSet)\) of abelian group objects in \(\catSet\).
\end{rmk}

\begin{lem} We have the following equivalences of categories:
    \begin{enumerate}
        \item $\catAbelianGroup(\catPresheaf(X)) \cong \catAbelianPresheaf(X)$
        \item $\catAbelianGroup(\catSheaf(X)) \cong \catAbelianSheaf(X)$
    \end{enumerate}
\end{lem}

\begin{proof}[sketch]
    \begin{enumerate}
        \item A presheaf $F\colon \open(X)\opp \to \catAbelianGroup$ consists of $F(U)\in\catAbelianGroup$ for all $U\in \open(X)\opp$ with restriction maps $r_{UV}\colon F(V)\to F(U)$ in $\catAbelianGroup$. This means that the group operation $F(U)\times F(U)\to F(U)$ is natural, that is, if $U\subseteq V$ then
        \[\begin{tikzcd}
        F(V)\times F(V) \arrow[r, "m_V"] \arrow[d, "r_{UV}\times r_{UV}"'] & F(V) \arrow[d, "r_{UV}"] \\
        F(U)\times F(U) \arrow[r, "m_U"']                                  & F(U)                    
        \end{tikzcd}\] So $m\colon F\times F\to F$ is natural. Likewise, $0$ and \(i\) are natural transformations (follow your nose). Conversely, an abelian object in $\catPresheaf(X)$ is a presheaf of abelian groups (why?).
        \item By \cref{cor:sheaf-condition-forgetful-functor}, a presheaf $F\colon \open(X)\opp\to \catAbelianGroup$ is a sheaf if and only if the composite $U\circ F\colon \open(X)\opp \to \catSet$ is a sheaf, where $U\colon \catAbelianGroup\to \catSet$ is the forgetful functor. Thus for any sheaf $\cat F\in \catAbelianSheaf(X)$, $U\circ \cat F$ is a sheaf of sets which has abelian group object structure by (i). Conversely, $(\mathcal F, m,i, 0)\in \catAbelianGroup(\catSheaf)$ is the data of a sheaf of abelian groups, since it is a presheaf of abelian groups by (i) and thus a sheaf since it is a sheaf on sets.  \qedhere
    \end{enumerate}\end{proof}

\begin{rmk}
There is a `tensor product' \(\otimes\) on presentable categories for which we (probably) have \(\catAbelianSheaf(X)\simeq\catSheaf(X)\otimes\catAbelianGroup\).
\end{rmk}

\begin{rmk}
    We can do everything we know for sheaves of abelian groups as well:
    \begin{itemize}
        \item We have an adjunction:
            \[\begin{tikzcd}
            \catAbelianPresheaf(X) \ar[r, "{\etalespace*}"{name=leftadj}, shift left=2] & \catAbelianGroup(\catTopologicalSpace_{/X}) \ar[l, "h_{-/X}"{name=rightadj}, shift left=2]
            \ar[from=leftadj, to=rightadj, phantom, "\leftadj" rotate=-90]
          \end{tikzcd}\]
         which restricts to an equivalence:
            \[\begin{tikzcd}
            \catAbelianSheaf(X) \ar[r, "{\etalespace*}"{name=leftadj}, shift left=2] & \catAbelianGroup(\catLocalHomeomorphism_{/X}) \ar[l, "h_{-/X}"{name=rightadj}, shift left=2]
            \ar[from=leftadj, to=rightadj, phantom, "\simeq"]
          \end{tikzcd}\]
          \item Monodromy:
            \[\catAbelianGroup(\catLocallyConstantSheaf(X)) \simeq \catName{Rep}_{\bb Z}(\homotopy[1](X,x))\text{,} \]
            where $\catName{Rep}_{\bb Z}(G)$ denotes the category of \(G\)-representations in $\catModule_{\bb Z}$.
            We may also denote $\catAbelianGroup(\catLocallyConstantSheaf(X))$ by $\catLocallyConstantAbelianSheaf(X) \simeq \catModule_{\bb Z[\homotopy[1](X,x)]}$, where \(\bb Z[\homotopy[1](X,x)]\) is the group algebra of \(\homotopy[1](X,x)\).
          \item Pushforward and pullback: for $f\colon Y\to X$ we have adjunctions
              \[\begin{tikzcd}
            \catAbelianPresheaf(Y) \ar[r, "f_*"'{name=rightadj}, shift right=2] &
            \catAbelianPresheaf(X) \ar[l, "f^\circledast"'{name=leftadj}, shift right=2]
            \ar[from=leftadj, to=rightadj, phantom, "\leftadj" rotate=-90]
          \end{tikzcd}\]
          and 
              \[\begin{tikzcd}
                \catAbelianSheaf(Y) \ar[r, "f_*"'{name=rightadj}, shift right=2] &
                \catAbelianSheaf(X) \ar[l, "f^*"'{name=leftadj}, shift right=2]
                \ar[from=leftadj, to=rightadj, phantom, "\leftadj" rotate=-90]
              \end{tikzcd}\]
        \item All the functors $f_*, f^{\circledast}, f^*, (-)^\sharp, h_{-/X}$ commute with the forgetful functors (see \cite[\href{https://stacks.math.columbia.edu/tag/0085}{Lemma~0085}]{stacks-project} for a hands-on proof that $(-)^\sharp$ does).
        \item A map $f\colon \cat F\to \cat G$ in $\catAbelianGroup(X)$ is an isomorphism if and only if the induced map $f_x\colon\mathcal F_x\to\mathcal G_x$ on stalks is an isomorphism for all $x\in X$. 
    \end{itemize}
\end{rmk}

In order to do homological algebra for sheaves of abelian groups, we need a notion of \emph{exactness} for sequences of sheaves of abelian groups.
Since the image presheaf of map of sheaves is in general not a sheaf itself, we define exactness in $\catAbelianGroup(X)$ via stalks\index{stalk}.

\begin{defn}\label{defn:short-exact-sequence-sheaves}
A sequence
\[\mathcal F\xrightarrow{f}\mathcal G\xrightarrow{g}\mathcal H\]
of sheaves of abelian groups on a space \(X\) is \emph{exact at $\cat G$} if the induced sequence
\[ \mathcal F_x\xrightarrow{f_x} \mathcal G_x \xrightarrow{g_x} \mathcal H_x \]
of stalks is exact at \(\mathcal G_x\) for all points $x\in X$.
\end{defn}

\begin{rmk}
Incidentally, the reason to define exactness via the stalks -- that the image presheaf is not a sheaf -- is also the reason the global sections functor (evaluation in the global sections) is generally only left exact, and thus gives rise to a right derived functor.
Right deriving the global sections functor on a sheaf  $\mathcal F\in \catAbelianGroup(X)$ allows us to compute sheaf cohomology, denoted $H^i(X;\mathcal F)$.
\end{rmk}

\begin{exmp}
    Let $0\to A\to B\to C\to 0$ be a short exact sequence in $\catAbelianGroup$. Then the sequence
    \[0\to \underline{A}\to \underline{B}\to \underline{C}\to 0\]
of constant sheaves is exact. Indeed, $\left(\underline{Z}\right)_x\cong Z$ for any $x\in X$ and any abelian group or set $Z$, since this is the fibre of $Z\times X\xrightarrow{\pr*_X} X$ at $x\in X$. 
\end{exmp}

\begin{exmp}
    Let $\iota\colon \{0\} \xhookrightarrow{} \bb R$, $A\in\catAbelianGroup$. Then $\iota_* A$ has etale space,
    \todo{draw line with $\#A$ origins}
\end{exmp}

\end{document}
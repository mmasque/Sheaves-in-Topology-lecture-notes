\chapter{Computing sheaf cohomology and higher pushforwards}

In this appendix, we summarise some strategies for computing sheaf cohomology and higher pushforwards.

\section{Sheaf cohomology}

\subsection{Compact spaces}
\begin{lem}[name={\cref{lem:compactly-supported-cohomology-is-ordinary-cohomology-on-compact-space}}]
Let \(\mathcal F\) be an abelian sheaf on a compact space \(X\).
Then \(H^i(X,\mathcal F)\cong H^i_c(X,\mathcal F)\) for all \(i\geq 0\).
\end{lem}

This result has the following consequences for computing the sheaf cohomology of a space \(X\):
\begin{itemize}
\item If \(X\) is compact, we can use both strategies for ordinary cohomology and strategies for compactly supported cohomology to compute either one.
\item If it is possible to relate \(X\) to a compact space \(K\) (one example we encountered was the embedding \(\bb R^n\cong(0,1)^n\hookrightarrow [0,1]^n\)), it might be possible to use strategies for compactly supported cohomology of \(K\) to say something about ordinary cohomology of \(X\).
\end{itemize}

\subsection{Open and closed subsets}
\begin{lem}
Let \(i\colon Z\hookrightarrow X\) be a closed subset of a topological space \(X\) and let \(\mathcal F\) be a sheaf on \(Z\).
Then we have
\[ H^i(X,i_*\mathcal F) \cong H^i(Z,\mathcal F) \]
for all \(i\geq 0\).
\end{lem}

\begin{lem}[name={Additional exercise~13.1(d)}]
Let \(j\colon U\hookrightarrow X\) be an open subset of a locally compact Hausdorff space \(X\) and let \(\mathcal F\) be a sheaf on \(U\).
Then we have
\[ H^i_c(X,j_!\mathcal F) \cong H^i_c(U,\mathcal F) \]
for all \(i\geq 0\).
\end{lem}

\begin{lem}[name={Homework~3, Exercise~4(d)}]
Let \(\mathcal F\) be an abelian sheaf on a locally compact Hausdorff space \(X\) and let \(j\colon U\hookrightarrow X\) be an open subset with closed complement \(i\colon Z\hookrightarrow X\).
Then there is a short exact sequence of sheaves of the form
\[ 0 \to j_!j^*\mathcal F \to \mathcal F \to i_*i^*\mathcal F \to 0\text{.} \]
\end{lem}

Note that \(j^*\mathcal F=\mathcal F|_U\) and \(i^*\mathcal F=\mathcal F|_Z\).

\begin{cor}[name={open--closed sequence, \cref{lem:open-closed-sequence}}]
Let \(\mathcal F\) be an abelian sheaf on a locally compact Hausdorff space \(X\) and let \(U\subseteq X\) be an open subset with closed complement \(Z\).
Then there is a long exact sequence of the form
\[ \dots \to H^i_c(U,\mathcal F|_U) \to H^i_c(X,\mathcal F) \to H^i_c(Z,\mathcal F|_Z) \to H^1(U,\mathcal F|_U) \to \dots\text{.} \]
\end{cor}

\subsection{Mayer--Vietoris}
The Mayer--Vietoris sequence for ordinary sheaf cohomology:
\begin{prop}[name={Mayer--Vietoris sequence, Homework~7, Exercise~1(c)}]
Let \(\mathcal F\) be an abelian sheaf on a topological space \(X\) and let \(U\) and \(V\) be open subsets of \(X\).
Then there is a long exact sequence of the form
\[ \dots \to H^i(U\cup V,\mathcal F) \to H^i(U,\mathcal F)\oplus H^i(V,\mathcal F) \to H^i(U\cap V,\mathcal F) \to H^{i+1}(U\cup V,\mathcal F)\to\dots\text{.} \]
\end{prop}

There is a variant for compactly supported sheaf cohomology.
Note that the order is flipped:
\begin{prop}[name={compactly supported Mayer--Vietoris sequence, Additional exercise~13.2}]
Let \(\mathcal F\) be an abelian sheaf on a topological space \(X\) and let \(U\) and \(V\) be open subsets of \(X\).
Then there is a long exact sequence of the form
\[ \dots \to H^i_c(U\cap V,\mathcal F) \to H^i_c(U,\mathcal F)\oplus H^i_c(V,\mathcal F) \to H^i_c(U\cup V,\mathcal F) \to H^{i+1}_c(U\cap V,\mathcal F)\to\dots\text{.} \]
\end{prop}

\subsection{Coefficients}

\begin{lem}[name={\cref{cor:coefficients-cohomology-integers}}]
If \(X\) is a locally compact Hausdorff space with compactly supported cohomology with \(\underline{\bb Z}\) coefficients given by
\[ H^i_c(X,\underline{\bb Z}) =
  \begin{cases}
    \bb Z & \text{if } i=0\text{,} \\
    0 & \text{if } i>0\text{,}
  \end{cases}
\]
then we have
\[ H^i_c(X,\underline{A}) =
  \begin{cases}
    A & \text{if } i=0\text{,} \\
    0 & \text{if } i>0\text{,}
  \end{cases}
\]
for any abelian group \(A\).
\end{lem}

\subsection{Homotopy invariance}

\begin{thm}[name={Vietoris--Begle mapping theorem, \cref{thm:Vietoris-Begle-mapping-theorem}}]
Let \(f\colon Y\to X\) be a proper map such that
\[ H^i(f\inv(x),\underline{\bb Z}) =
  \begin{cases}
    \bb Z & \text{if } i = 0\text{,} \\
    0 & \text{if } i>0
  \end{cases}
\]
for all \(x\in X\).
Then for all abelian sheaves \(\mathcal F\) on \(X\), the maps
\[ H^i(X,\mathcal F)\to H^i(Y,f^*\mathcal F) \]
are isomorphisms for all \(i\in\bb N\).
\end{thm}

\begin{cor}[name={\cref{cor:sheaf-cohomology-homotopy-invariant}}]
If \(f,g\colon Y\to X\) are homotopic maps, then the maps
\[ f^*,g^*\colon H^i(X,\underline{A}) \to H^i(Y,\underline{A}) \]
agree for all \(i\geq 0\) and all abelian groups \(A\).
\end{cor}

\begin{cor}[name={\cref{cor:sheaf-cohomology-homotopy-equivalence}}]
If \(f\colon X\to Y\) is a homotopy equivalence (of paracompact Hausdorff or locally compact Hausdorff spaces), then
\[ f^*\colon H^i(X,\underline{A})\xrightarrow{\cong} H^i(Y,\underline{A}) \]
is an isomorphism for any abelian group \(A\).
In particular, if \(X\) and \(Y\) are homotopy equivalent, then \(H^i(X,\underline{A})\cong H^i(Y,\underline{A})\) for any \(A\).

In particular, if \(X\) is contractible (and paracompact Hausdorff or locally compact Hausdorff), then
\[ H^i(X,\underline{A}) =
  \begin{cases}
    A & \text{if } i = 0\text{,} \\
    0 & \text{if } i > 0
  \end{cases}
\]
for any abelian group \(A\).
\end{cor}

\subsection{Čech cohomology}

\begin{prop}[name={\cref{cor:Čech-cohomology-vanishing-intersections}}]
Let \(X\) be a topological space with an open cover \(\mathcal U=\{U_i\hookrightarrow X\}_{i\in I}\) and let \(\mathcal F\) be an abelian sheaf on \(X\).
If \(H^i(U_{i_0,\ldots,i_n},\mathcal F)=0\) for all \(i>0\), all \(n\geq 0\) and all \(i_0,\ldots,i_n\in I\), then the map
\[ \check{H}^i(\mathcal U,\mathcal F) \to H^i(X,\mathcal F) \]
is an isomorphism.
\end{prop}

\begin{cor}[name={\cref{exmp:Čech-cohomology-intersections-disjoint-unions-of-contractibles}}]
If all intersections \(U_{i_0,\ldots,i_n}\) are disjoint unions of contractibles, then \(H^i(U_{i_0,\ldots,i_n},\underline{\bb Z})=0\) for all \(i>0\), so \(\check{H}^i(\mathcal U,\underline{\bb Z})=H^i(X,\underline{\bb Z})\).
\end{cor}

\begin{prop}[name={\cref{thm:Čech-cohomology-paracompact-Hausdorff-space}}]
Let \(\mathcal F\) be an abelian sheaf on a paracompact Hausdorff space \(X\).
Then there is an isomorphism
\[ \colim_{\mathcal U\in h_0\catOpenCover{X}}\check{H}^i(\mathcal U,\mathcal F) \xrightarrow{\cong} H^i(X,\mathcal F)\text{.} \]
\end{prop}

\section{Higher pushforwards}

The following two lemmas express that the (proper) pushforward is the relative version of (compactly supported) cohomology.

\begin{lem}[name={\cref{exc:cohomology-higher-pushforward-to-point}}]
Let \(\sh F\) be an abelian sheaf on \(X\) and let \(f\colon X\to *\) denote the unique map.
Then \(\cohomology^i(X,\sh F) = \rightderived^i f_*\sh F\).
\end{lem}

\begin{lem}[name={\cref{exc:compactly-supported-cohomology-higher-proper-pushforward-to-point}}]
Let \(\sh F\) be an abelian sheaf on a Hausdorff space \(X\) and let \(f\colon X\to *\) denote the unique map.
Then \(\cohomology^i_c(X,\sh F) = \rightderived^i f_!\sh F\).
\end{lem}

\begin{thm}[name={\cref{thm:proper-base-change-theorem-1}}]
Let $f\colon Y \to X$ be a proper map and assume that $X$ and $Y$ are either both paracompact Hausdorff or both locally compact Hausdorff.
For an abelian sheaf \(\sh F\) on \(Y\), the natural map \[
  (\rightderived^i f_* \sh{F})_x \to \cohomology^i(f^{-1}(x), \sh{F})
\] is an isomorphism for all $x \in X$ and for all $i \geq 0$.
\end{thm}

\begin{thm}[name={\cref{thm:proper-base-changed-theorem-2}}]
Let $\colon: Y \to X$ be a map of locally compact Hausdorff spaces. For an abelian sheaf \(\sh{F}\) on \(Y\), the natural map \[
  (\rightderived^i f_! \sh{F})_x \to \cohomology^i_c(f^{-1}(x), \sh{F})
\] is an isomorphism for all $x \in X$ and all $i \geq 0$.
\end{thm}

\begin{cor}[name={proper base change theorem, \cref{cor:proper-base-change}}]
Let the diagram
\[\begin{tikzcd}
	{Y'} \ar[pullback]{rd} & {X'} \\
	Y & X
	\arrow["{\hat f}", from=1-1, to=1-2]
	\arrow["{\hat g}"', from=1-1, to=2-1]
	\arrow["g", from=1-2, to=2-2]
	\arrow["f"', from=2-1, to=2-2]
\end{tikzcd}\]
be a pullback square in $\catTopologicalSpace$.
Then:
\begin{enumerate}
	\item If $f$ is a proper map and $X$ and $X'$ are either both paracompact Hausdorff or both locally compact Hausdorff, then there are canonical isomorphisms \[
    	g^* \circ \rightderived^i f_* \xrightarrow{\cong} \rightderived^i \hat f_* \circ \hat g^*\colon \catAbelianSheaf(Y) \to \catAbelianSheaf(X).
    \]
	\item If $X,X'$ and $Y'$ are locally compact Hausdorff, then there are canonical isomorphisms \[
    	g^* \circ \rightderived^i f_! \xrightarrow{\cong} \rightderived^i \hat f_! \circ \hat g^*\colon \catAbelianSheaf(Y) \to \catAbelianSheaf(X').
    \]
\end{enumerate}
\end{cor}

\section{Cohomology of some spaces}
The following results were computed in earlier lectures or in the homework exercises.

\begin{cor}[name=\cref{cor:computation-cohomology-interval-real-line}]~  
\begin{enumerate}
        \item If $X$ is either $\bb R$, $[0,1]$ or $[0,1)$ then the cohomology of \(X\) with coefficients in the constant sheaf \(\underline{\bb Z}\) is $$\cohomology^i(X, \underline{\bb Z}) = \begin{cases}
            \bb Z & \text{if } i = 0\text{,} \\ 
            0 & \text{if } i>0\text{.}
        \end{cases}$$
        \item The compactly supported cohomology of \(\bb R\), \([0,1]\) and \([0,1)\) with coefficients in the constant sheaf \(\underline{\bb Z}\) is:
        \begin{align*}
            \cohomology^i_c(\bb R, \underline{\bb Z}) &= \begin{cases}
                \bb Z & \text{if } i = 1\text{,} \\
                0 & \text{else,}
            \end{cases} \\
            \cohomology^i_c([0,1], \underline{\bb Z}) &= \begin{cases}
                \bb Z & \text{if } i = 0\text{,} \\
                0 & \text{else,}
            \end{cases} \\
            \cohomology^i_c([0,1), \underline{\bb Z}) &= 0 \qquad \text{for all } i\text{.}
        \end{align*}
    \end{enumerate}
\end{cor}

\begin{prop}[name=Cohomology of the sphere]
    	The (compactly supported) cohomology of the n-sphere $\sphere^n$ with coefficients in the constant sheaf $\underline{\bb Z}$ is \[
        	\cohomology^i(\sphere^n, \underline{\bb Z}) = \cohomology_c^i(\sphere^n, \underline{\bb Z}) = \begin{cases}
            	\bb Z, & \text{if } i = 0,n\\
                0,&\text{else}.
            \end{cases}
        \] 
Note that the cohomology and compactly supported cohomology of any sheaf on $\sphere^n$ agree, by \cref{lem:compactly-supported-cohomology-is-ordinary-cohomology-on-compact-space}.
\end{prop}

\begin{prop}[name=Cohomology of Euclidean space]~
	\begin{enumerate}
    	\item The cohomology of $\bb R^n$ with coefficients in the constant sheaf $\bb Z$ is \[
        	\cohomology^i(\bb R^n, \underline{\bb Z}) = \begin{cases}
            	\bb Z, & \text{if } i = 0,\\
                0, &\text{else},
            \end{cases}
        \] for all $n \in \bb Z_{\geq 0}$.
        \item The compactly supported cohomology of $\bb R^n$ with coefficients in the constant sheaf $\bb Z$ is \[
        	\cohomology_c^i(\bb R^n, \underline{\bb Z}) = \begin{cases}
            	\bb Z, & \text{if } i = n,\\
                0, & \text{else},
            \end{cases}
        \] for all $n \in \bb Z_{\geq 0}$. 
    \end{enumerate}

\end{prop}

The following proposition genealises the cohomology of $\bb R^n$. It shows that the (compactly supported) cohomology with coefficients in the constant sheaf $\bb Z$ tracks the number of holes in $\bb R^n\setminus \{x_1, \ldots, x_m\}$. We treat the case $n = 1$ separately: the case $n > 1$ has nonvanishing cohomology in two distinct degrees, but these are the same degree for $i = 0$.

\begin{prop}[name=Cohomology of Euclidean space with missing points]
Let $m,n \in \bb Z_{\geq 1}$, and let $x_1, \ldots x_m$ be distinct points in $\bb R^n$. 
	\begin{enumerate}
        \item The cohomology of $\bb R\setminus\{x_1, \ldots, x_m\}$
        with coefficients in the constant sheaf $\bb Z$ is \[
    	     	\cohomology^i(\bb R\setminus\{x_1, \ldots, x_m\}, \underline{\bb Z}) = \begin{cases}
                \bb Z^{m+1}, & \text{if } i = 0,\\
                0, & \text{else}.
            \end{cases}
        \]
        \item The cohomology of $\bb R^n\setminus\{x_1, \ldots, x_m\}$ with coefficients in the constant sheaf $\bb Z$ is \[
        	\cohomology^i(\bb R^n\setminus\{x_1, \ldots, x_m\}, \underline{\bb Z}) = \begin{cases}
                \bb Z^m, & \text{if } i = n-1,\\
                \bb Z, & \text{if } i = 0,\\
                0, & \text{else},
            \end{cases}
        \] for all $n \in \bb Z_{> 1}$. 
        \item The compactly supported cohomology of $\bb R\setminus\{x_1, \ldots, x_m\}$
        with coefficients in the constant sheaf $\bb Z$ is \[
    	     	\cohomology^i_c(\bb R\setminus\{x_1, \ldots, x_m\}, \underline{\bb Z}) = \begin{cases}
                \bb Z^{m+1}, & \text{if } i = 1,\\
                0, & \text{else}.
            \end{cases}
        \]
    	\item The compactly supported cohomology of $\bb R^n\setminus \{x_1, \ldots, x_m\}$ with coefficients in the constant sheaf $\bb Z$ is \[
        	\cohomology^i_c(\bb R^n\setminus \{x_1, \ldots, x_m\}, \underline{\bb Z}) = \begin{cases}
                \bb Z^{m}, & \text{if } i = 1\\
                \bb Z, & \text{if } i = n \\
                0, &\text{else},
            \end{cases}
        \] for all $n \in \bb Z_{>1}$.
    \end{enumerate}
\end{prop}
%%% Local Variables:
%%% mode: LaTeX
%%% TeX-master: "../main"
%%% End:

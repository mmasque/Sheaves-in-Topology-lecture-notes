\documentclass[../main.tex]{subfiles}

\begin{document}
\chapter{Additive and abelian categories}
\section{Additive categories}
\remyquote{We can only write down diagrams, prove they commute, and run away as fast as we can.}
\begin{defn}
	A \emph{pre-additive} category $\cat C$ is a locally small category where each hom-set $\Hom_{\cat C}(X, Y)$ are endowed with an abelian group structure.
\end{defn}
\begin{lem}\label{lem:pre-additive-categories-are-nice}
Let $\cat C$ be a pre-additive category.
\begin{enumerate}
	\item If $X \in \cat C$, then the following are equivalent:
	\begin{enumerate}
    	\item $X$ is initial;
        \item $X$ is terminal;
        \item $\id_X = 0 \in \Hom_{\cat C}(X,X)$;
    \end{enumerate}
	\item If $X, Y, Z \in \cat C$ then the following are equivalent:
	\begin{enumerate}
    	\item There are maps \[X \xrightarrow{i} Z \xleftarrow{j} Y\] making $Z$ the coproduct.
    	\item There are maps \[Z \xrightarrow{p} X \xleftarrow Y\] making $Z$ the product.
		\item There are maps \[X \xrightleftarrows{i}[p] Z \xleftrightarrows{j}[q] Y\] such that $p \circ i = \id_X$ and $q \circ j = \id_Y$ and $i\circ p + j \circ q = \id_Z$.
    \end{enumerate}

\end{enumerate}

\end{lem}

\begin{defn}
	A biproduct in a pre-additive category of $X, Y \in \cat C$ is an object $Z$ with maps \[X \xrightleftarrows{i}[p] Z \xleftrightarrows{j}[q] Y\] with $pi = \id_X$, $qj = \id_Y$ and $ip + jq = \id_Z$.
\end{defn}
\begin{rmk}
	Note that these force $q \circ i$ and $p \circ j$ to be zero. To see this it suffices to show that $j \circ q \circ i = 0$, because $j$ is monic.
	This holds because $jqi = (\id_Z - ip)i = i - ipi = i - i = 0$.
\end{rmk}

\begin{proof}(Of the lemma)
	\begin{enumerate}
    	\item If $x$ is initial, then $\Hom_{\cat C}(X, X) = 0$ so $\id_X = 0$.
		Conversely, if $\id_x = 0$ then every map $f: X \to Y$ i n $\cat C$ satisfies $f = f \circ \id_X = f \circ 0 = 0$. So $X$ is initial. This proves (i) is equivalent to (ii) and that (ii) is equivalent to (iii) follows dually.
		\item If \[X \xrightleftarrows{i}[p] Z \xleftrightarrows{j}[q] Y\] is a biproduct, we saw $qi = 0$ and $pj = 0$. Then $X \xrightarrow{i} Z$ and $Y \xrightarrow{j} Z$ is a coproduct. If \[X \xrightarrow{f} W \xleftarrow{g} Y\] is any cocone, set $h = fp + gq: Z \to W$. Then $hi = (fp + gq)i = fpi + gqi = f \circ \id_X + g \circ 0 = f$, and likewise $hj = g$. If $h': Z \to W$ satisfies $h'i = f$ and $h'j = g$ then $h' = h' \circ \id_Z = h' \circ (ip + jq) = h'ip + h'jq = fp + gq = h$. So \[X \xrightarrow{i} Z \xleftarrow{j} Y\] form a coproduct. Conversely, if these maps form a coproduct, then the cocones 
		\begin{align*}
        	&X \xrightarrow{\id} X \xleftarrow{0} Y\\
			&X \xrightarrow{0} Y \xleftarrow{\id} Y			
        \end{align*}
		 define maps $p: Z \to X$, $q: Z \to Y$ with $pi = \id_X, pj = 0, qi = 0, qj = \id_Y$. Then $ip + jq: Z \to Z$ satisfies $(ip + jq)i = ipi + jqi = i \circ \id_X + j \circ 0 = i$, and likewise $(ip + jq)\circ j = \cdots = j$. So $ip + jq = \id_Z$ by the universal property of the coproduct. This proves $(i) \Leftrightarrow (iii)$ and $(ii) \Leftrightarrow$ follows dually.
    \end{enumerate}
\end{proof}
\begin{defn}
	An \emph{additive} category is a pre-additive category with finite products.
\end{defn}
\begin{rmk}
	By the lemma \cref{lem:pre-additive-categories-are-nice} if $\cat C$ is additive then
	\begin{itemize}
    	\item the terminal object is also initial, hence a zero object $0$.
		\item IF $X, Y \in \cat C$ then the biproduct \[X \xrightleftarrows{i}[p] X \oplus Y \xleftrightarrows{j}[q] Y\] exists, so $\cat C$ has finite coproducts.
		\item If $X_1, \cdots, X_n \in \cat C$ then the map $X_1 \coprod \cdots \coprod X_n \to X_1 \times \cdots \times X_n$ such that $X_i \to X_1 \coprod \cdots \coprod X_n \to X_1 \times \cdots \times X_n \to X_j$ is $\id_X$ if $i = j$ and $0$ if $i \neq j$ is an isomorphism. We write $X_1 \oplus \cdots \oplus X_n$ for the n-ary biproduct.
    \end{itemize}

\end{rmk}
% \begin{exc}
% 	\todo{something about turning map into functor, additional exercises}.
% \end{exc}

\begin{defn}
	A \emph{semi-additive} category is a category $\cat C$ with finite products and coproducts such that
	\begin{enumerate}
    	\item the natural map $ \emptyset \to *$ is an isomorphism (so $\cat C$ has a zero object $0$),
		\item for $X_1, ..., X_n$ in $\cat C$ the map $X_1 \coprod \cdots \coprod X_n \to X_1 \times \cdots \times X_n$ is an isomorphism.
    \end{enumerate}
	(Note that we use (i) to define the map in (ii)).
\end{defn}

\begin{rmk}
	While (pre)-additive is a \emph{structure}, being semi-additive is a \emph{property} of a category $\cat C$.
\end{rmk}

\begin{lem}
	If $\cat C$ is semi-additive, then it is canonically enriched in commutative monoids. If $\cat C$ was additive, then this agrees with the given enrichment in abelian groups (under the inclusion $\catAbelianGroup \hookrightarrow \catCommutativeMonoid$).
\end{lem}
\begin{proof}(Sketch, in which the commutativity of the many diagrams is left to the reader,)
	For $f, g: X \rightrightarrows Y$ in $\cat C$, define $f + g$ to be
	\[
		X \xrightarrow{\Delta} X \times X = X \oplus X \xrightarrow{f \oplus g} Y \oplus Y = Y \coprod Y \xrightarrow{\triangledown} Y
	\] where $\Delta$ is the diagonal map and $\triangledown$ is the codiagonal.
	For $f, g, h$ maps from $X$ to $Y$, commutativity of 
\[\begin{tikzcd}
	& {X\oplus X} && {Y\oplus Y} \\
	X & {X\oplus X\oplus X} && {Y\oplus Y\oplus Y} & Y \\
	& {X\oplus X} && {Y\oplus Y}
	\arrow["\Delta", from=2-1, to=1-2]
	\arrow["{\id_X \oplus \Delta}"', from=3-2, to=2-2]
	\arrow["{\Delta \oplus \id_X}", from=1-2, to=2-2]
	\arrow["\Delta", from=2-1, to=2-2]
	\arrow["\Delta"', from=2-1, to=3-2]
	\arrow["{f \oplus(g+h)}"', from=3-2, to=3-4]
	\arrow["{(f+g)\oplus h}", from=1-2, to=1-4]
	\arrow["{?}", from=2-4, to=1-4]
	\arrow["{\id_Y \oplus \triangledown}", from=3-4, to=2-4]
	\arrow["\triangledown"', from=3-4, to=2-5]
	\arrow["\Delta", from=1-4, to=2-5]
	\arrow["{f\oplus g\oplus h}"{description}, from=2-2, to=2-4]
\end{tikzcd}\]	\todo{map out of triple Y oplus}

	shows that $(f+g) + h = f + (g + h)$. For $f: X \to Y$ commutativity of 
\[\begin{tikzcd}
	& {X\oplus X} && {Y\oplus Y} \\
	X & {X \oplus 0} && {Y\oplus 0} & Y
	\arrow[from=2-1, to=2-2]
	\arrow["\id\oplus0", from=1-2, to=2-2]
	\arrow["\Delta", from=2-1, to=1-2]
	\arrow["{f\oplus \id}"', from=2-2, to=2-4]
	\arrow["{f\oplus 0}", from=1-2, to=1-4]
	\arrow["\triangledown", from=1-4, to=2-5]
	\arrow[from=2-4, to=2-5]
	\arrow["{\id \oplus 0}", from=2-4, to=1-4]
\end{tikzcd}\]	
	 shows $f + 0 = f$ and likewise $ 0 + f = f$.
	 For $f,g: X \rightrightarrows Y$ commutativity of 
	
\[\begin{tikzcd}
	& {X\oplus X} && {Y\oplus Y} \\
	X &&&& Y \\
	& {X\oplus X} && {Y\oplus Y}
	\arrow["\triangledown", from=1-4, to=2-5]
	\arrow["\triangledown"', from=3-4, to=2-5]
	\arrow["{g \oplus f}"', from=3-2, to=3-4]
	\arrow["{f\oplus g}", from=1-2, to=1-4]
	\arrow["\Delta", from=2-1, to=1-2]
	\arrow["\Delta"', from=2-1, to=3-2]
	\arrow["{\text{swap}}"{description}, from=1-2, to=3-2]
	\arrow["{\text{swap}}"{description}, from=1-4, to=3-4]
\end{tikzcd}\]	 gives $f + g = g + f$ so have $\Hom_{\cat C}(X,Y)$ is a commutative monoid. If $f,f': X \rightrightarrows Y$ and $g,g':Y  \rightrightarrows Z$ then commutativity of 

\[\begin{tikzcd}
	X & {X\oplus X} & {Y\oplus Y} & Y \\
	&& {Z\oplus Z} & Z
	\arrow["{gf\oplus gf'}"', from=1-2, to=2-3]
	\arrow[from=1-3, to=2-3]
	\arrow["\triangledown", from=1-3, to=1-4]
	\arrow["g", from=1-4, to=2-4]
	\arrow["\triangledown", from=2-3, to=2-4]
	\arrow["{f\oplus f'}", from=1-2, to=1-3]
	\arrow["\Delta", from=1-1, to=1-2]
\end{tikzcd}\]
shows that $g(f+ f') = gf + gf'$ and likewise $(g + g')f = gf + g'f$, proving the first statement.
	If $\cat C$ is additive and $f,g: X \rightrightarrows Y$ then let
	\[
    	X \xrightleftarrows{i_1}[p_1] X \oplus X \xleftrightarrows{i_2}[p_2] X
    \] be the biproduct, and likewise for $Y$. Then commutativity of 
	
\[\begin{tikzcd}
	X & {X\oplus X} & X & X & {X\oplus X} & X \\
	Y & {Y\oplus Y} & Y & Y & {Y\oplus Y} & Y
	\arrow["f", from=1-1, to=2-1]
	\arrow["{p_1}", from=1-1, to=1-2]
	\arrow["{p_2}", from=1-2, to=1-3]
	\arrow["g", from=1-3, to=2-3]
	\arrow["{p_2}"', from=2-2, to=2-3]
	\arrow[from=2-2, to=2-1]
	\arrow["{f\oplus g}"{description}, from=1-2, to=2-2]
	\arrow["f"', from=1-4, to=2-4]
	\arrow["{i_1}"', from=2-4, to=2-5]
	\arrow["{i_1}", from=1-4, to=1-5]
	\arrow["{i_2}"', from=1-6, to=1-5]
	\arrow["{i_2}", from=2-6, to=2-5]
	\arrow["g", from=1-6, to=2-6]
	\arrow["{f\oplus g}", from=1-5, to=2-5]
\end{tikzcd}\]
gives $f \oplus g = (f \oplus g) \circ \id_{X \oplus X} = (f \oplus g) \circ (i_1 p_1 + i_2 p_2) = i_1 f p_1 + i_2 g p_2$, and likewise $f \oplus 0 = i_1 f p_1$, $0 \oplus g = i_2 g p_2$, so \[
    	f \oplus g = (f \oplus 0) + (0 \oplus g).
    \] Then bilinearity of composition shows that the composite
	\[
    	X \xrightarrow{\Delta} X \oplus X \xrightarrow{f \oplus g} Y \oplus Y \triangledown Y
    \] is $f + g$ (in the enriched sense).
\end{proof}
% What we have learnt from this long proof is that being an additive category is a \emph{property}.
\begin{cor}
	Being additive (i.e. admitting an additive structure) is a property.
\end{cor}
\begin{proof}
	The category $\cat C$ is additive if and only if if it is semi-additive and all $\Hom$ monoids are groups.
\end{proof}

\begin{exmp}
	$\catCommutativeMonoid$ is semi-additive (omitted), but not additive ($\Hom(\bb N, \bb N) = \bb N$ does not have inverses).
\end{exmp}

\section{Abelian categories}
\begin{exc}
	If $\cat C$ is pre-additive, show that \[
    	\eq(X \rightrightarrows{f}[g] = \eq(X \rightrightarrows{(f-g)}[0] Y)) =: \ker(f-g)
    \] (if one of them exists).
\end{exc}
So an additive category $\cat C$ has finite limits (respectively colimits) if and only if it has kernels (respectively cokernels).
\begin{defn}
	A \emph{pre-abelian} category is an additive category with kernels and cokernels.
\end{defn}
\begin{defn}
	Let $\cat C$ be a pre-abelian category and $f: X \to Y$ in $\cat C$. The \emph{image} of $f$ is $\im(f) = \ker(Y \to \coker(f))$. The \emph{coimage} of $f$ is $\coim(f) = \coker(\ker(f) \to X)$.
\end{defn}
Warning: the zoo of names will get worse.
\begin{lem}
	Any $f: X \to Y$ in a pre-abelian category factors uniquely via \[
    	X \to \coim(f) \to \im(f) \to Y.
    \]
\end{lem}
\begin{proof}
	The composition $\ker(f) \to X \to Y$ is $0$. So there is a unique factorisation \[
    	X \to \coim(f) \to Y
    \] of $f$. The composition $\coim(f) \to Y \to \coker(f)$ is $0$ (since this holds after precomposition with $X \to \coim(f)$). So there is a unique factorisation \[
    	\coim(f) \to \im(f) \to Y.
    \]
\end{proof}
And now the moment we have all been waiting for: the definition of an abelian category.

\begin{defn}
	An \emph{abelian} category is a pre-abelian category $A$ such that the $\coim(f) \to \im(f)$ is an isomorphism for all $f: X \to Y$ in $\cat A$.
\end{defn}
\begin{exmp}
	In $\catAbelianGroup$, $\catModule_R$ this is the first isomorphism theorem: $X/\ker(f) \xrightarrow{\sim} \im(f)$.
\end{exmp}
\begin{lem}
	If $\cat A$ is abelian and $\cat C$ is small, then $\catFunctor(\cat C\opp, \cat A)$ is abelian.
\end{lem}
This lemma implies that the category of presheaves of abelian groups on a topological space is an abelian category.

\begin{proof}
	If $\alpha, \beta: F \implies G$ is a natural transformation, then define the sum $\alpha + \beta: F \implies G$ by $(\alpha + \beta)_U = \alpha_U + \beta_U$. This is natural since the maps $F(V) \to F(U)$ and $G(V) \to G(U)$ are group homorphisms for any map $U \to V$ in $\cat C$. This turns $\catFunctor(\cat C\opp, \cat A)$ into a pre-additive category. All the other questions (existence of finite, limits and colimits, the first isomorphism theroem) are checked objectwise.
\end{proof}
\end{document}

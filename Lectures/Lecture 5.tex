\chapter{Pullback and pushforward}

\remyquote{There is a risk you might learn something -- beware!}
\noindent
The goal of this lecture is to define a pair of adjoint functors
\begin{equation*}
  \begin{tikzcd}
    \catSheaf(Y) \ar[r, "f_*"'{name=rightadj}, shift right=2] &
    \catSheaf(X) \ar[l, "f^*"'{name=leftadj}, shift right=2]
    \ar[from=leftadj, to=rightadj, phantom, "\leftadj" rotate=-90]
  \end{tikzcd}
\end{equation*}
for a continuous map \(f\colon Y\to X\), called \indexemph{pullback} and \indexemph{pushforward}.
The strategy will be to first define these operations for presheaves, and then restrict to sheaves.
One of the functors will already send sheaves to sheaves, for the other we will postcompose the functor on the level of presheaves with sheafification.

\section{Pushforward}

A continuous map \(f\colon Y\to X\) induces a functor \(f\inv\colon\open(X)\to\open(Y)\) which sends an open set \(U\subseteq X\) to the open set \(f\inv(U)\subseteq Y\).
If \(U\subseteq V\subseteq X\) are open sets, then \(f\inv(U)\subseteq f\inv(V)\subseteq Y\), so this is indeed functorial.

\begin{defn}
The \indexdefnemph{pushforward} of a presheaf \(G\) on \(Y\) along \(f\) is the composite presheaf
\begin{equation*}
  f_*G \colon \open(X)\opp \xrightarrow{f\inv} \open(Y)\opp \xrightarrow{G} \catSet
\end{equation*}
on \(X\).
Explicitly, the value of the pushforward \(f_*G\) on an open \(U\subseteq X\) is \((f_*G)(U) = G(f\inv(U))\).
\end{defn}

The pushforward functor \(f_*\colon\catPresheaf(Y)\to\catPresheaf(X)\) is given by precomposition by the functor \(f\inv\colon\open(X)\opp\to\open(Y)\opp\), and so is indeed seen to be functorial.

\begin{lem}
If \(\mathcal G\) is a sheaf on \(Y\), then the pushforward \(f_*\mathcal G\) of \(\mathcal G\) along \(f\) is a sheaf on \(X\).
In particular, pushforward restricts to a functor \(f_*\colon\catSheaf(Y)\to\catSheaf(X)\) on the level of sheaves.
\end{lem}
\begin{proof}
Let \(U=\bigcup_{i\in I}U_i\) be an open cover in \(X\).
Then \(f\inv(U) = \bigcup_{i\in I}f\inv(U_i)\) is an open cover in \(Y\).
Applied to this cover, the sheaf condition for \(\mathcal G\) gives us an equaliser diagram
\begin{equation*}
  \begin{tikzcd}
    \mathcal G(f\inv(U)) \ar[r] & \prod_{i\in I}\mathcal G(f\inv(U_i)) \ar[r, shift left] \ar[r, shift right] & \prod_{i,j\in I}\mathcal G(f\inv(U_i\cap U_j))
  \end{tikzcd}
\end{equation*}
(where we have rewritten \(f\inv(U_i)\cap f\inv(U_j)=f\inv(U_i\cap U_j)\)), giving the sheaf condition for the pushforward \(f_*\mathcal G\).
\end{proof}

\section{Pullback}
\remyquote{Just sheafify the hell out of everything.}
\noindent
The \indexemph{pullback} of a sheaf on \(X\) along \(f\) should be a sheaf on \(Y\).
One way to approach the problem of constructing a presheaf on \(Y\) from a presheaf \(\mathcal F\) on \(X\), would be to try extending it along \(f\inv\) (and this is what we will attempt):
\begin{equation*}
  \begin{tikzcd}[column sep=tiny]
    \open(X)\opp \ar[rr, "\mathcal F"] \ar[rd, "f\inv"'] & & \catSet \\
    & \open(Y)\opp \ar[ru, dashed, "?"']
  \end{tikzcd}
\end{equation*}
In general, there might not be such an on-the-nose extension, but we can approximate an extension by considering extensions \emph{up to} a natural transformation.
There could be many such approximations, so we want to consider the `best possible approximation' in some suitable sense.

In category theory, the general problem of approximating an extension of a functor along another functor is studied using \indexemph{Kan extensions}; we refer to~\cite[Chapter~6]{riehlCategoryTheoryContext2016} for an introduction of Kan extensions, whose definition we actually do not need to know for our current purposes.
Here we will apply the general theory to our case to define the pullback of sheaves.\footnote{One can also construct and prove everything by hand, as was done in class, but this is rather tedious. We wish to illustrate with the following exposition that all the arguments will be completely formal.}

\begin{prop}\label{prop:pullback-Kan-extension}
Left Kan extensions\index{Kan extension} of presheaves on \(X\) (functors \(\open(X)\opp\to\catSet\)) along the functor \(f\inv\colon\open(X)\opp\to\open(Y)\opp\) exist, and assemble into a functor
\[\Lan*_{f\inv}\colon\catPresheaf(X)\to\catPresheaf(Y)\]
which is left adjoint to the \indexterm{pushforward} functor \(f_*\):
\begin{equation*}
  \begin{tikzcd}
    \catPresheaf(Y) \ar[r, "f_*"'{name=rightadj}, shift right=2] &
    \catPresheaf(X) \ar[l, "\Lan*_{f\inv}"'{name=leftadj}, shift right=2]
    \ar[from=leftadj, to=rightadj, phantom, "\leftadj" rotate=-90]
  \end{tikzcd}
\end{equation*}
Moreover, the left Kan extension along \(f\inv\) of a presheaf \(F\) on \(X\) is given on opens \(U\subseteq Y\) by
\begin{equation} \label{eq:left-Kan-extension-f-inv-colimit}
  \Lan_{f\inv} F(U) = \colim\displaylimits_{f\inv(W)\supseteq U}F(W)
\end{equation}
where \(W\) ranges over opens of \(X\) (more precisely, the colimit diagram is indexed by the full subcategory of \(\open(X)\) on those \(W\) such that \(f\inv(W)\supseteq U\), or equivalently \(W\supseteq f(U)\)).
\end{prop}
\begin{proof}
This is a special case of~\cite[Corollary~6.2.6]{riehlCategoryTheoryContext2016}; the only nontrivial step in applying the general result to this special case is recognising that the comma category \(f\inv\downarrow U\) for \(U\in\open(Y)\opp\) is the described index category, but this verification is elementary enough to be left to the reader.
\end{proof}

We would also like a description of what \(\Lan*_{f\inv}\) does on maps.
By a computation of the colimit in \cref{eq:left-Kan-extension-f-inv-colimit}, an element of \(\Lan*_{f\inv} F(U)\) is given by \([s]_W\) for some section \({s\in F(W)}\) for some \(W\supseteq f(U)\), where \([s]_W = [s']_{W'}\) if and only if there exists an open \(W''\subseteq W\cap W'\) containing \(f(U)\) such that \(s|_{W''} = s'|_{W''}\).
Unravelling the construction in \cite[Theorem~6.2.1]{riehlCategoryTheoryContext2016}, we see that the map induced by opens \(U\subseteq V\) in \(Y\) is given by
\[ \colim\displaylimits_{f\inv(W)\supseteq V}F(W) \to \colim\displaylimits_{f\inv(W)\supseteq U}F(W)\text{,} \quad [s]_W \mapsto [s]_W\text{.} \]

We may now define the pullback as follows:
\begin{defn}
The \emph{pullback}\index{pullback!of presheaves} \(f^{\circledast}F\) of a presheaf \(F\) on \(X\) along \(f\) is the left Kan extension of \(F\) along \(f\inv\).
The pullback functor is denoted \(f^{\circledast}\coloneq\Lan*_{f\inv}\colon\catPresheaf(X)\to\catPresheaf(Y)\).
\end{defn}

We have defined the pullback functor in such a way that it is left adjoint to the pushforward of presheaves.

We use the circled asterisk \(\circledast\) in the notation for the pullback of presheaves to distinguish it from the pullback of sheaves, which we will define momentarily, and for which we use a normal asterisk.\footnote{In class, we used the notation $f^{\textup{pre,*}}$ for $f^\circledast$, which we do not find very elegant.}
Unlike the pushforward, namely, the pullback of presheaves does not directly restrict to sheaves; that is to say, there are sheaves which are sent by \(f^{\circledast}\) to a presheaf which does not satisfy the sheaf condition, as witnessed by the following counterexample:

\begin{exmp}
Let $Y$ be the two-point space with the discrete topology, and let $X$ be the one-point space, with the unique map $f: Y \to X$.
We can easily put a sheaf on $X$, for example by defining $\mathcal{F}(\emptyset) = *$ and $\mathcal{F}(X) = \{42, 43, 44\}$.
The pullback in presheaves at any point $\{*\} \subset Y$ is $f^{\circledast}(\{*\}) = \colim_{f^{-1}(W) \supseteq *}\mathcal{F}(W) = \mathcal{F}(X) = \{42, 43, 44\}$. But the pullback in presheaves at $Y$ itself is also $\mathcal{F}(X) = \{42, 43, 44\}$. Check that the sheaf condition doesn't hold for the cover of $Y$ given by two opens, one containing precisely each point: the gluing condition on $\{42,43,44\} \times \{42, 43, 44\}$ is void because the points do not intersect.
\end{exmp}

\begin{defn}
The \emph{pullback}\index{pullback!of sheaves} \(f^*\mathcal F\) of a sheaf \(\mathcal F\) on \(X\) along a continuous map \(f\colon Y\to X\) is the sheaf
\[ f^*\mathcal F \coloneq (f^{\circledast}\mathcal F)^\sharp \]
on \(Y\), the sheafification of the presheaf pullback of \(\mathcal F\).
The pullback functor of sheaves is thus the composite
\[ f^* \colon \catSheaf(X) \hookrightarrow \catPresheaf(X) \xrightarrow{f^{\circledast}} \catPresheaf(Y) \xrightarrow{(-)^\sharp} \catSheaf(Y)\text{.} \]
\end{defn}

Note that the definition of the pullback also makes sense for presheaves; we also denote the functor \(\catPresheaf(X)\to\catSheaf(Y)\) sending a presheaf \(F\) to the pullback \(f^*F = (f^\circledast)^\sharp\) by \(f^*\).

\begin{prop}\label{lem:pushforward-pullback-adjunction-sheaves}
Pushforward and pullback of sheaves along \(f\) define an adjunction
\begin{equation*}
  \begin{tikzcd}
    \catSheaf(Y) \ar[r, "f_*"'{name=rightadj}, shift right=2] &
    \catSheaf(X) \ar[l, "f^*"'{name=leftadj}, shift right=2]
    \ar[from=leftadj, to=rightadj, phantom, "\leftadj" rotate=-90]
  \end{tikzcd}
\end{equation*}
\end{prop}
\begin{proof}
Compose the adjunctions
\begin{equation*}
  \begin{tikzcd}
    \catSheaf(Y) \ar[r, ""'{name=rightadj1}, inclusion, shift right=2] \ar[rrd, "f_*"', bend right=15] &
    \catPresheaf(Y) \ar[l, "(-)^\sharp"'{name=leftadj1}, shift right=2] \ar[r, "f_*"'{name=rightadj2}, shift right=2] &
    \catPresheaf(X) \ar[l, "f^{\circledast}"'{name=leftadj2}, shift right=2] \ar[ll, "f^*"', bend right=15, shift right=5] \\
    & & \catSheaf(X) \ar[u, inclusion]
    \ar[from=leftadj1, to=rightadj1, phantom, "\leftadj" rotate=-90]
    \ar[from=leftadj2, to=rightadj2, phantom, "\leftadj" rotate=-90]
  \end{tikzcd}
\end{equation*}
of \cref{cor:sheafification-left-adjoint-to-inclusion} and \cref{prop:pullback-Kan-extension}.
\end{proof}

Between the sets $\Hom[\catSheaf(Y)](f^*\mathcal F,\mathcal G)$ and $\Hom[\catSheaf(X)](\mathcal F,f_*\mathcal G)$ which are naturally isomorphic by the adjunction, there is also an `intermediate' set of maps, called \emph{$f$-maps}; see~\cite[\href{https://stacks.math.columbia.edu/tag/008K}{Lemma 008K}]{stacks-project}.

From the fact that the pushforward already sends sheaves to sheaves (so its definition `doesn't need sheafification'), we obtain the following corollary, showing that pulling back the sheafification of a presheaf is the same as pulling back the presheaf.

\begin{cor}\label{cor:pullback-sheafification}
Let \(f\colon Y\hookrightarrow X\) be a continuous map.
Then the functors
\[ \catPresheaf(X) \xrightarrow{f^*} \catSheaf(Y) \qquad \text{and} \qquad \catPresheaf(X) \xrightarrow{(-)^\sharp} \catSheaf(X) \xrightarrow{f^*} \catSheaf(Y) \]
are naturally isomorphic.
\end{cor}
\begin{proof}
  % and write \(\iota\colon\catSheaf(X)\hookrightarrow\catPresheaf(X)\) for the full subcategory inclusion
By definition, the former functor is the composite
\begin{equation*}
  \begin{tikzcd}
    \catPresheaf(X) \ar[r, "f^\circledast"{name=leftadj1}, shift left=2] &
    \catPresheaf(*) \ar[l, "f_*"{name=rightadj1}, shift left=2] \ar[r, "(-)^\sharp"{name=leftadj2}, shift left=2] &
    \catSheaf(*) \ar[l, ""{name=rightadj2}, shift left=2, inclusion]
    \ar[from=leftadj1, to=rightadj1, phantom, "\leftadj" rotate=-90]
    \ar[from=leftadj2, to=rightadj2, phantom, "\leftadj" rotate=-90]
  \end{tikzcd}
\end{equation*}
of the presheaf pullback along \(f\) and sheafification, which have right adjoints given by respectively the pushforward along \(f\) and the full subcategory inclusion by \cref{prop:pullback-Kan-extension} and \cref{cor:sheafification-left-adjoint-to-inclusion}.
The latter functor has a right adjoint
\begin{equation*}
  \begin{tikzcd}
    \catPresheaf(X) \ar[r, "(-)^\sharp"{name=leftadj1}, shift left=2] &
    \catSheaf(X) \ar[l, ""{name=rightadj1}, shift left=2, inclusion] \ar[r, "f^*"{name=leftadj2}, shift left=2] &
    \catSheaf(*) \ar[l, "f_*"{name=rightadj2}, shift left=2]
    \ar[from=leftadj1, to=rightadj1, phantom, "\leftadj" rotate=-90]
    \ar[from=leftadj2, to=rightadj2, phantom, "\leftadj" rotate=-90]
  \end{tikzcd}
\end{equation*}
given by pushforward along \(f\) followed by the full subcategory inclusion by \cref{cor:sheafification-left-adjoint-to-inclusion} and \cref{lem:pushforward-pullback-adjunction-sheaves}.
These two right adjoint are equal by definition, so we find the required natural isomorphism.
\end{proof}

The following corollary roughly says that pushforward and pullback are `functorial in the map' (respectively co- and contravariantly).\footnote{This statement can probably be made precise in terms of \(2\)-categories.}

\begin{cor}\label{cor:pushforward-pullback-functorial-in-map}
For any space \(X\), we have \((\id_X)_*=\id_{\catSheaf(X)}\cong(\id_X)^*\), and for any maps \(f\colon Z\to Y\) and \(g\colon Y\to X\) we have \((gf)_*=g_*f_*\colon\catSheaf(Z)\to\catSheaf(X)\) and \((gf)^*\cong f^*g^*\colon\catSheaf(X)\to\catSheaf(Z)\).
\end{cor}
Note that the identities for the pushforward hold on-the-nose, whereas we only prove the identities for the pullback up to natural isomorphism.
\begin{proof}
The identities for the pushforward follow directly from the definition since \((gf)\inv(U)=f\inv(g\inv(U))\).
The identities for the pullback follow from those for the pushforward and the adjunction of \cref{lem:pushforward-pullback-adjunction-sheaves}: we have adjunctions \((gf)^*\leftadj(gf)_*\) and \(f^*g^*\leftadj g_*f_*\) and the right adjoints in these adjunctions are equal.
\end{proof}

\section{Stalks, germs, skyscrapers}

\begin{defn}\label{defn:stalk}
Let $i_x\colon \{x\} \hookrightarrow X$ be the inclusion of a point in a space.
Then $i_x^* \mathcal{F}$ is the \indexdefnemph{stalk} $\mathcal{F}_x$ of $\mathcal{F}$ at $x$; this is a sheaf on a point, so equivalently a set by \cref{exmp:sheaf-on-point-set} (the value of the sheaf on the whole space).
Unravelling definitions, we have \[
    \mathcal{F}_x = \colim_{U \ni x} \mathcal{F}(U)\text{.}
\]
An element of the stalk at $x$ is represented by a \indexdefnemph{germ} $[s]_U$ for $U \ni x$ and $s \in \mathcal{F}(U)$, where \([s]_U = [t]_V\) if and only if there exists an open neighbourhood \(W\subseteq U\cap V\) of \(x\) such that \(s|_W=t|_W\).
\end{defn}
Note that the stalk $\mathcal{F}_x$ is the fibre of $\etalespace(X) \to X$ over $x$.

\begin{rmk}
\Cref{cor:pullback-sheafification} tells us that we can compute the stalks of the sheafification of a presheaf by computing the stalks of the presheaf itself.
\end{rmk}

\begin{exmp}[orientation sheaf of a \indexterm{manifold}]
Let \(M\) be an \(n\)-dimensional topological manifold, by which we mean a Hausdorff space that is locally homeomorphic to \(\bb R^n\) (i.e., every point of \(M\) admits a neighbourhood homeomorphic to \(\bb R^n\)).
For a subset \(K\subseteq M\), we write \[H_k(M\mid K;R) \coloneq H_k(M,M\setminus K;R)\] for the \(k\)th homology of \(M\) relative to \(M\setminus K\) with coefficients in a commutative ring \(R\).

A \emph{local \(R\)-orientation} at a point \(x\in M\) is a generator \(\mu_x\) of \(H_n(M\mid x;R)\cong R\) (note that we take homology in degree \(n\), the dimension of the manifold \(M\)).
That \(H_n(M\mid x;R)\) is isomorphic to \(R\) (as \(R\)-modules) follows from excision; choosing a generator corresponds exactly to choosing such an isomorphism.
An \emph{\(R\)-orientation}\index{manifold!orientation} of \(M\) is a family of local orientations \((\mu_x)_{x\in M}\) with the property that every point \(y\in M\) admits a compact neighbourhood \(K\) and an element \(\mu_K\in H_n(M\mid K;R)\) such that \(\mu_K\) restricts to \(\mu_x\) for any \(x\in K\) under the map \(H_n(M\mid K;R)\to H_n(M\mid x;R)\) induced by the inclusion \(M\setminus K\to M\setminus\{x\}\).

There is a sheaf \(\mathcal O_M\) of \(R\)-modules, the \emph{orientation sheaf}\index{manifold!orientation sheaf} of \(M\), defined by
\[\mathcal O_M(U)\coloneq H_n(M\mid U;R)\]
for \(U\subseteq M\) open.
The restriction map for opens \(U\subseteq V\subseteq M\) is induced by the inclusion of pairs \((M,M\setminus V)\hookrightarrow(M,M\setminus U)\).
The stalk of the orientation sheaf \(\mathcal O_M\) at \(x\in M\) is the \(R\)-module \(H_n(M\mid x;R)\).
\end{exmp}

\begin{rmk}
One can show (as a generalisation of an exercise in Homework 3) that the étalé space of the pullback can be obtained as a \indexterm{fibre product}. More precisely, given $f\colon Y \to X$, the diagram
\[\begin{tikzcd}
	{\etalespace(f^*\mathcal{F})} & {\etalespace(\mathcal{F})} \\
	Y & X
	\arrow[from=1-1, to=2-1]
	\arrow[from=2-1, to=2-2]
	\arrow[from=1-2, to=2-2]
	\arrow[from=1-1, to=1-2]
	\arrow[pullback, from=1-1, to=2-2]
\end{tikzcd}\]
is a pullback square.
This is used in \cite[\S~\RN{2}.9]{MacLaneMoerdijkSheavesGeometryLogic} to define $f^*$.
\end{rmk}


Given a map $\alpha\colon \mathcal{F} \to \mathcal{G}$ between sheaves on a space $X$ and a point $x \in X$, there is an induced map on stalks $\alpha_x\colon \mathcal{F}_x \to \mathcal{G}_x$, because $\mathcal{G}_x$ is a cocone over
\[\setpred{\mathcal{F}(U)}{U \in \open(X),x \in U}\text{,}\]
of which $\mathcal{F}_x$ is a colimit.

\begin{lem}\label{lem:stalkwise-check-isomorphism}
    Let $\alpha\colon \mathcal{F} \to \mathcal{G}$ be a map in $\catSheaf(X)$.
    Then $\alpha$ is an isomorphism of sheaves if and only if $\alpha_x\colon \mathcal{F}_x \to \mathcal{G}_x$ is a bijection for all $x \in X$.
\end{lem}
\begin{proof}
    The map $\etalespace(\alpha)\colon \etalespace(\mathcal{F}) \to \etalespace(\mathcal{G})$ is a local homeomorphism over $X$, because the maps $\etalespace(\mathcal{F}) \to X$ and $\etalespace(\mathcal{G}) \to X$ are local homeomorphisms. In particular, $\etalespace(\alpha)$ is an open map. So it is invertible if and only if it is a bijection. Check that this is equivalent to requiring that the map on the fibres over $x$ is a bijection for all $x \in X$. Since the stalks $\mathcal{F}_x$ and $\mathcal{G}_x$ are exactly these fibres, we are done.
\end{proof}

The next lemma shows that the pullback of a constant sheaf along any map is again constant.

\begin{lem}
	Let $f: Y \to X$ be a continuous map and let $S$ be a set. Then $f^*\underline{S}_X \cong \underline{S}_Y$.
\end{lem}
\begin{proof}
  It suffices to show the lemma holds for the special case of the maps $p_W: W \to *$ from a space to a point. The general case will then follow. Indeed, then \[f^*\underline{S}_X \cong f^* p_X^* \underline{S}_* \cong (f \circ p_X)^* \underline{S}_* = p_Y^* \underline{S}_* \cong \underline{S}_Y\]
  by \cref{cor:pushforward-pullback-functorial-in-map}.
  For the special case, the presheaf pullback on a nonempty open becomes constant: \[p_W^\circledast \underline{S}_*(U) = \colim_{V \supseteq p_W(U)}\underline{S}_*(V) \cong \underline{S}_*(*) = S\text{.}\] On the empty set, we have $p_W^\circledast(\emptyset) = *$.
  Its sheafification $p_W^*\underline{S}_*$ is then $\underline{S}_W$. (Verify that sheafification is unaffected by the value of a presheaf on the empty set.)
  \end{proof}

\begin{defn}
    Let $\{x\} \to X$ be the inclusion of a point $x \in X$ and let \(S\) be a set. The \indexemph{skyscraper sheaf} at $x$ with value $S$ is $i_{x,*}\underline{S}$, the pushforward of the constant sheaf. We denote it $i_{*, S}$.
\end{defn}

\begin{exmp}
    Let $\{0\} \to \bb R$ be the inclusion of zero into the real numbers. Let $S = \{a,b\}$. Then the skyscraper sheaf is
    \[
        i_{*,S}(U) = S(U \cap 0) =     \begin{cases}
        \{a,b\} & \text{if } 0 \in U\\
        * & \text{if } x \notin U.
    \end{cases}
    \]
    In Homework 2, we saw that the étalé space of this sheaf is the line with two origins.
\end{exmp}

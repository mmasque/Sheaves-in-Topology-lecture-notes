\chapter{Review and outlook}

\section{Comparison between cohomology theories}
\remyquote{I should say something at some point, let me not do that yet.}
\noindent
In this section, we compare different definitions of cohomology theories for topological spaces.
Specifically, we will compare sheaf cohomology, Čech cohomology, de Rham cohomology, singular cohomology and the `Eilenberg--MacLane approach' to cohomology (also called the `homotopy construction of cohomology').
The conclusion will be the picture in \cref{fig:comparison-cohomology-theories}.
The arrows' labels point to the comparison results.
\begin{figure}
  \centering
  \begin{tikzpicture}
    \node[draw] (sheaf) at (0,0) {Sheaf cohomology \(H^i(-,\mathcal F)\)};
    \node[draw] (deRham) at (4,-1.5) {De Rham cohomology \(H^i_{\smash{\textup{dR}}}(-)\)};
    \node[draw] (Čech) at (0,-3) {Čech cohomology \(\check{H}^i(-,\mathcal F)\)};
    \node[draw] (singular) at (-4,-1.5) {Singular cohomology \(H^i_{\smash{\textup{sing}}}(-,A)\)};
    \node[draw] (EilenbergMacLane) at (-4,1.5) {Eilenberg--MacLane \([-,K(A,i)]\)};
    \draw[<->] ($(deRham.north)!0.5!(deRham.north west)$) -- ($(sheaf.south east)!0.5!(sheaf.south)$) node[pos=0.5, above right] {\labelcref{prop:comparison-de-Rham-cohomology-sheaf-cohomology}};
    \draw[<->] (sheaf.south) -- (Čech.north) node[pos=0.5, right] {\labelcref{prop:comparison-sheaf-cohomology-Čech-cohomology}};
    \draw[<->] ($(deRham.south)!0.5!(deRham.south west)$) -- ($(Čech.north east)!0.5!(Čech.north)$) node[pos=0.5,below right] {\labelcref{prop:comparison-de-Rham-cohomology-Čech-chomology}};
    \draw[<->] ($(singular.south)!0.5!(singular.south east)$) -- ($(Čech.north west)!0.5!(Čech.north)$) node[pos=0.5, below left] {\labelcref{prop:comparison-singular-cohomology-Čech-cohomology}};
    \draw[<->] (singular.north) -- (EilenbergMacLane.south) node[pos=0.5, left] {\labelcref{prop:representability-of-singular-cohomology}};
    \draw[<->] ($(EilenbergMacLane.south)!0.5!(EilenbergMacLane.south east)$) -- ($(sheaf.north west)!0.5!(sheaf.north)$) node[pos=0.5, above right] {\labelcref{prop:comparison-sheaf-cohomology-Eilenberg-MacLane}};
    \draw[<->] ($(singular.north)!0.5!(singular.north east)$) -- ($(sheaf.south west)!0.5!(sheaf.south)$) node[pos=0.5,above left] {\labelcref{prop:comparison-singular-cohomology-sheaf-cohomology}};
  \end{tikzpicture}
  \caption{Comparison of cohomology theories}
  \label{fig:comparison-cohomology-theories}
\end{figure}

\begin{prop}\label{prop:comparison-sheaf-cohomology-Čech-cohomology}
If \(X\) is paracompact Hausdorff, then \(\check{H}^i(X,\mathcal F)\cong H^i(X,\mathcal F)\) for all abelian sheaves \(\mathcal F\) on \(X\).
Moreover, if \(\mathcal U\) is an open cover of \(X\) such that each intersection \(U_{i_0}\cap\dots\cap U_{i_k}\) is a disjoint union of contractible spaces, then \(\check{H}^i(\mathcal U,\underline{A})\cong H^i(X,\underline{A})\) for every abelian group \(A\).
\end{prop}
\todo{ref proof}

\begin{prop}\label{prop:comparison-de-Rham-cohomology-sheaf-cohomology}
If \(X\) is a \(C^\infty\)-manifold (assumed to be paracompact), then \(H^i_{\smash{\textup{dR}}}(X)\cong H^i(X,\underline{\bb R})\).
\end{prop}
\begin{proof}
The de Rham complex
\[ 0\to\underline{\bb R}\to \Omega^0_X \to \Omega^1_X \to \dots \]
is an exact sequence of sheaves: locally, every closed \(i\)-from is exact (Poincaré lemma).
Since \(\Omega^i_X\) is a sheaf of \(C^\infty(-,\bb R)\)-modules, it is soft, so the above is an acyclic resolution of \(\underline{\bb R}\).
\end{proof}

\begin{prop}\label{prop:comparison-de-Rham-cohomology-Čech-chomology}
If \(X\) is a manifold, then \(H^i_{\smash{\textup{dR}}}(X)\cong\check{H}^i(\mathcal U,\underline{\bb R})\) for any cover \(\mathcal U\) such that each intersection \(U_{i_0}\cap \dots\cap U_{i_k}\) is a disjoint union of contractible spaces.
\end{prop}
\todo{ref Bott-Tu}

\begin{prop}\label{prop:comparison-singular-cohomology-sheaf-cohomology}
If \(X\) is locally contractible, then \(H^i_{\smash{\textup{sing}}}(X,A) \cong H^i(X,\underline{A})\) for every abelian group \(A\).
\end{prop}
\begin{proof}
\todo{write}
\end{proof}

\begin{prop}\label{prop:comparison-singular-cohomology-Čech-cohomology}
If \(X\) admits a cover \(\mathcal U\) such that each intersection \(U_{i_0}\cap\dots\cap U_{i_k}\) is a disjoint union of contractible spaces, then \(H^i_{\smash{\textup{sing}}}(X,A)\cong\check{H}^i(X,\underline{A})\) for every abelian group \(A\).
\end{prop}

To state the comparison between singular cohomology and the `Eilenberg--MacLane approach' (what \cite{hatcherAlgebraicTopology2002} calls the `homotopy construction of cohomology'), we have to introduce some notions.

\begin{defn}\label{defn:Eilenberg-MacLane-space}
Let \((G,n)\) be a pair of a natural number \(n\) and a group \(G\) (which should be abelian if \(n\geq 2\)).
Then an \emph{Eilenberg--MacLane space} \(K(G,n)\) is a pointed space such that
\[ \homotopy[i](K(G,n)) \cong
  \begin{cases}
    G & \text{if } i = n\text{,} \\
    0 & \text{otherwise.}
  \end{cases}
\]
\end{defn}

\begin{exmp}
The circle \(\sphere[1]\) is a \(K(\bb Z,1)\) (as can be seen from the long exact sequence of the fibration \(\bb Z\to\bb R\to\sphere[1]\)).
\end{exmp}

\begin{thm}[name={\cite[Proposition~4.30]{hatcherAlgebraicTopology2002}}]
For every pair \((G,n)\) as in \cref{defn:Eilenberg-MacLane-space}, there exists a \(K(G,n)\), and any two \(K(G,n)\)'s are weakly homotopy equivalent.
\end{thm}

\begin{prop}[name={representability of singular cohomology~\cite[Theorem 4.57]{hatcherAlgebraicTopology2002}}]\label{prop:representability-of-singular-cohomology}
If \(X\) is a CW-complex, then \([X,K(A,i)]\cong H^i_{\smash{\textup{sing}}}(X,A)\) for every abelian group \(A\).
\end{prop}

\begin{rmk}
Since singular homology \(H^i(-,A)\colon\catTopologicalSpace\to\catAbelianGroup\) is homotopy invariant, it can be seen as a functor \(H^i(-,A)\colon\catHomotopy(\catTopologicalSpace)\to\catAbelianGroup\) on the homotopy category of topological spaces.
\Cref{prop:representability-of-singular-cohomology} says that this latter functor is represented by the Eilenberg--MacLane space \(K(A,i)\).
\end{rmk}

\begin{prop}\label{prop:comparison-sheaf-cohomology-Eilenberg-MacLane}
If \(X\) is paracompact Hausdorff, then \([X,K(A,i)]\cong H^i(X,\underline{A})\) for every abelian group \(A\).
\end{prop}
We saw this for \(K(\bb Z,1)\simeq\sphere[1]\) in Homework~8, Exercise~2.

\todo{Warsaw circle}
% \begin{center}
%   \begin{tikzpicture}
%     \draw[thick] plot[domain=0.01:2,samples=5000] (\x,{sin(1/\x r)});
%   \end{tikzpicture}
% \end{center}

We conclude that sheaf cohomology gives the `best answer':
\begin{itemize}
\item The Eilenberg--MacLane approach works for spaces with many maps out of it (paracompact Hausdorff).
\item Singular cohomology works for spaces with many maps into it (CW-complexes).
\item De Rham cohomology works for manifolds (what we might call locally trivial spaces?).
\end{itemize}

The introduction of \citeauthor{lurieInfinityTopoi2003}'s preprint \citetitle{lurieInfinityTopoi2003}~\cite{lurieInfinityTopoi2003} also discusses these various notions of cohomology; we recommend taking a look.
The next section discusses the novel ideas from Lurie's work.

\section{Stacks and higher topoi}
The goal of \citeauthor{lurieInfinityTopoi2003}'s preprint \citetitle{lurieInfinityTopoi2003}~\cite{lurieInfinityTopoi2003} was to generalise the set \([X,K(G,n)]\) of homotopy classes of maps into an Eilenberg--MacLane space to the set \([X,Y]\) of homotopy classes of maps into an arbitrary space/homotopy type/Kan complex/\oo-groupoid/anima\footnote{These words all mean the same thing: they are presentations of topological spaces up to weak homotopy equivalence. When we say `space' in this section, this is what we mean, and we always use the adjective `topological' when talking about topological spaces. If \(Y\) is a Kan complex, then \([X,Y]\) should be interpreted as the set \([X,\abs{Y}]\) of homotopy classes of maps into the geometric realisaton of \(Y\).} \(Y\), using an `internal' definition in terms of sheaves on \(X\).

\begin{defn}
A \emph{sheaf of spaces} (also called an \emph{\oo-stack}) on a topological space \(X\) is a functor
\[ \mathcal F\colon\open(X)\opp\to\catSpace \]
into the \oo-category of spaces (often also denoted \(\mathcal S\)) such that for every open cover \(U=\bigcup_{i\in I}U_i\), the diagram
\begin{equation*}
  \begin{tikzcd}[column sep=small]
    \mathcal F(U) \ar{r} & \prod\limits_{i_0\in I}\mathcal F(U_{i_0}) \ar[shift left]{r} \ar[shift right]{r} & \prod\limits_{i_0,i_1\in I}\mathcal F(U_{i_0}\cap U_{i_1}) \ar[shift left=2]{r} \ar{r} \ar[shift right=2]{r} & \prod\limits_{i_0,i_1,i_2\in I}\mathcal F(U_{i_0}\cap U_{i_1}\cap U_{i_2}) \ar[shift left=3]{r} \ar[shift left]{r} \ar[shift right]{r} \ar[shift right=3]{r} & \ldots
  \end{tikzcd}
\end{equation*}
realises \(\mathcal F(U)\) as the homotopy limit (the limit in the \(\infty\)-categorical sense~\cite[Definition 1.2.13.4]{lurieHigherToposTheory2009}) of the rest of the diagram.
\end{defn}

If we work with the \oo-category \(\catSpace_{\leq n}\) of \(n\)-truncated spaces (that is, spaces with vanishing homotopy groups above degree \(n\)), we only need to consider the diagram up to the \((n+1)\)-fold product.
For instance, for \(n=0\) we have \(\catSpace_{\leq 0}\simeq\catSet\), and we only need to look at the part
\begin{equation*}
  \begin{tikzcd}[column sep=small]
    \mathcal F(U) \ar{r} & \prod\limits_{i_0\in I}\mathcal F(U_{i_0}) \ar[shift left]{r} \ar[shift right]{r} & \prod\limits_{i_0,i_1\in I}\mathcal F(U_{i_0}\cap U_{i_1})
  \end{tikzcd}
\end{equation*}
as we indeed did before for sheaves of sets.
(This is related to the fact that \((-)^\sharp=\check{H}^0(X,\check{H}^0(X,-))\): after one iteration of \(\check{H}^0(X,-)\), the map from \(\mathcal F(U)\) to the equaliser of \(\prod_i\mathcal F(U_i)\rightrightarrows \prod_{i,j}\mathcal F(U_i\cap U_j)\) is \((-1)\)-truncated (that is, injective).
After the second iteration, it is \((-2)\)-truncated, so an isomorphism.)

The \oo-category of \(n\)-truncated spaces can alternatively be presented as the \oo-category of \(n\)-groupoids (for \(n\geq -2\)).
In the low degrees, the \oo-category of \((-2)\)-groupoids is equivalent to the terminal category; the \oo-category of \((-1)\)-groupoids is equivalent to the walking arrow category \(\initial\to\terminal\); and the \oo-category is equivalent o the category of sets.

\begin{exmp}
A sheaf of \(1\)-truncated spaces is a \emph{stack} (in \(1\)-groupoids): it is a functor \(\open(X)\opp\to\catGroupoid\) (as \oo-categories) with a gluing condition.
\end{exmp}

\begin{exmp}
The association
\[ U \mapsto \{\text{rank } n \text{ locally constant sheaves on } U\}^\cong \]
sending an open \(U\) to the groupoid core of the full subcategory of rank \(n\) locally constant sheaves is a stack.
It is the constant sheaf \(\deloop\GL_n(\underline{\bb Z})\).
The sheaf condition says: if we have a family \((\mathcal G_i)_{i\in I}\) of locally constant sheaves and isomorphisms \((\phi_{ij}\colon\mathcal G_j|_{U_i\cap U_j}\cong\mathcal G_i|_{U_i\cap U_j})_{i,j\in I}\) such that \(\phi_{ij}|_{U_i\cap U_j\cap U_k}\circ\phi_{jk}|_{U_i\cap U_j\cap U_k}=\phi_{ik}|_{U_i\cap U_j\cap U_k}\), then we can glue the \(\mathcal G_i\) to a sheaf \(\mathcal G\) on \(U\) -- this is just `gluing sheaves'!
\end{exmp}

\begin{rmk}
The classical literature about (\(1\)-)stacks translates everything in concrete statements about \(1\)-categories.
\end{rmk}

\begin{thm}[name={\cite[Theorem~7.1.0.1]{lurieHigherToposTheory2009}}]
If \(X\) is a paracompact topological space and \(K\) is a space, then the functor
\[ \rightderived\Gamma(X,-)\colon\catSheaf(X,\catSpace)\to\catSheaf(*,\catSpace) \simeq\catSpace \]
induced by \(X\to *\) satisfies
\[ \pi_0\rightderived\Gamma(X,\underline{K}) \cong [X,K]\text{.} \]
\end{thm}
The proof of this theorem is finished on page~705 of \cite{lurieHigherToposTheory2009}.

\section{Relative Poincaré duality}
\todo{write}

\remyquote{Should we go back to earth?}

%%% Local Variables:
%%% mode: latex
%%% TeX-master: "../main"
%%% End:

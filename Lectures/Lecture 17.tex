\chapter{Čech cohomology, comparison with sheaf cohomology}
\label{lecture:17}

\section{Čech cohomology}
\remyquote{Since I don't know what I'm doing, let me take a break.}
\noindent
Recall from the last lecture that we can define the cochains in degree \(n\) of the Čech complex of an abelian presheaf \(F\) on a space \(X\) with respect to an open cover \(\mathcal U=\{U_i\hookrightarrow X\}_{i\in I}\) as
\[ \check{C}^n(\mathcal U,F) \cong \prod_{i_0<\ldots<i_n}\mathcal F(U_{i_0,\ldots,i_n}) \]
if the indexing set \(I\) is totally ordered.
Throughout this lecture, we work with this definition, so we assume \(I\) is totally ordered; we can also do without this assumption, but prefer the notational simplicity we obtain with this assumption.

The Čech cohomology, which is the cohomology of the Čech complex, is defined for all presheaves, not just sheaves.
Today, we will discuss some results about Čech cohomology which do hold on the category of presheaves, but not on the full subcategory of sheaves.
In the following lemma we already see why this is important.

\begin{lem}
The functor \(\check{H}^i(\mathcal U,-)\colon\catAbelianPresheaf(X)\to\catAbelianGroup\) define a \(\delta\)-functor.\todo{define}
\end{lem}
\begin{proof}
If \(0\to F\to G\to H\to 0\) is a short exact sequence of presheaves, then for all \(i_0<\ldots<i_n\), the sequence
\[ 0 \to F(U_{i_0,\ldots,i_n}) \to G(U_{i_0,\ldots,i_n}) \to H(U_{i_0,\ldots,i_n}) \to 0 \]
is exact (but this is not true in the category of sheaves!).
Thus, the sequence
\[ 0 \to \check{C}^\bullet(\mathcal U,F) \to \check{C}^\bullet(\mathcal U,G) \to \check{C}^\bullet(\mathcal U,H)\to 0 \]
of complexes is exact, giving the required long exact sequence.
\end{proof}

\begin{thm}\label{thm:Čech-chomology-vanishes-injective-presheaf}
If \(I\) is an injective abelian presheaf on a space \(X\), then the Čech cohomology \(\check{H}^i(\mathcal U,I)\) vanishes for all \(i>0\).
\end{thm}

To prove this theorem, we introduce some notation and prove a lemma.

\begin{notn}
Let \(\cochaincomplex{K}\) be the cochain complex of presheaves
\[ \ldots \to K^{-2}\to K^{-1} \to K^0\to 0\to \ldots \]
given by
\[ K^{-n}\coloneq\bigoplus_{i_0<\ldots<i_n}\underline{\bb Z}^{\smash{\textup{pre}}}_{U_{i_0,\ldots,i_n}} \]
with differential \(d^{-n}\colon K^{-n}\to K^{-n+1}\) given on the factor \(\underline{\bb Z}^{\smash{\textup{pre}}}_{U_{i_0,\ldots,i_n}}\) by
\begin{align*}
  \sum_{k=0}^n(-1)^k\bigl(\underline{\bb Z}^{\smash{\textup{pre}}}_{U_{i_0,\ldots,i_n}}\to\underline{\bb Z}^{\smash{\textup{pre}}}_{U_{i_0,\ldots,i_{k-1},i_{k+1},\ldots,i_n}}\bigr) \colon \underline{\bb Z}^{\smash{\textup{pre}}}_{U_{i_0,\ldots,i_n}} \to K^{-(n-1)}=\bigoplus_{j_0<\ldots<j_{n-1}}\underline{\bb Z}^{\smash{\textup{pre}}}_{U_{j_0,\ldots,j_{n-1}}}
\end{align*}
of the maps induced by the inclusions \(U_{i_0,\ldots,i_n}\hookrightarrow U_{i_0,\ldots,i_{k-1},i_{k+1},\ldots,i_n}\).
Then we have \(\Hom[\catAbelianPresheaf(X)](K^{-\bullet}, F) \cong \check{C}^\bullet(\mathcal U,F)\), more or less by definition.
\end{notn}

\begin{lem}
Let \(\underline{\bb Z}^{\smash{\textup{pre}}}_{\mathcal U}\subseteq\underline{\bb Z}^{\smash{\textup{pre}}}_X\) be the image of \(K^0=\bigoplus_{i\in I}\underline{\bb Z}^{\smash{\textup{pre}}}_{U_i}\to\underline{\bb Z}^{\smash{\textup{pre}}}_X\), whose value on an open \(V\) is \(\bb Z\) if \(V\subseteq U_i\) for some \(i\), and zero otherwise.
Then the sequence
\[ K^\bullet_+ = (\ldots\to K^{-2}\to K^{-1}\to K^0\to\underline{\bb Z}^{\smash{\textup{pre}}}_{\mathcal U}\to 0\to\ldots) \]
is exact in \(\catAbelianPresheaf(X)\).
In other words, the cohomology of \(K^\bullet\) is
\[ H^i(K^\bullet) =
  \begin{cases}
    \underline{\bb Z}^{\smash{\textup{pre}}}_{\mathcal U} & \text{if } i = 0\text{,} \\
    0 & \text{if } i > 0\text{.}
  \end{cases}
\]
\end{lem}
\begin{proof}
\todo{write}
\end{proof}

\begin{proof}[name={of \cref{thm:Čech-chomology-vanishes-injective-presheaf}}]
We saw that \(\check{C}^\bullet_{\smash{\textup{ord}}}(\mathcal U,I) = \Hom[\catAbelianPresheaf(X)](K^\bullet,I)\), and \(K^\bullet_+\) is exact.
Since \(\Hom(-,I)\) is exact (because \(I\) is injective), we conclude that the sequence
\[ 0\to \Hom(\underline{\bb Z}^{\smash{\textup{pre}}}_{\mathcal U},I) \to \check{C}^0(\mathcal U,I) \to \check{C}^1(\mathcal U,I)\to\ldots \]
is exact, so \(\check{H}^i(\mathcal U,I)=0\) for \(i>0\).
\end{proof}

\begin{cor}
The \(\delta\)-functor \(\check{H}^i(\mathcal U,-)\colon\catAbelianPresheaf(X)\to\catAbelianGroup\) is isomorphic as a \(\delta\)-functor to \(\rightderived^i\check{H}^0(\mathcal U,-)\).
\end{cor}
The proof of this corollary is Additional exercise~16.2(c).

\begin{rmk}
The higher Čech cohomology functors are derived functors from the category of presheaves, not the category of sheaves.
In fact, the composite
\[ \catAbelianSheaf(X)\hookrightarrow\catAbelianPresheaf(X)\xrightarrow{\check{H}^0(\mathcal U,-)}\catAbelianGroup \]
is the global sections functor \(\Gamma(X,-)\).
\end{rmk}

\todo{write}

\begin{cor}\label{cor:Čech-cohomology-vanishing-intersections}
Let \(X\) be a topological space with an open cover \(\mathcal U=\{U_i\hookrightarrow X\}_{i\in I}\) and let \(\mathcal F\) be an abelian sheaf on \(X\).
If \(H^i(U_{i_0,\ldots,i_n},\mathcal F)=0\) for all \(i>0\), all \(n\geq 0\) and all \(i_0,\ldots,i_n\in I\), then the map
\[ \check{H}^i(\mathcal U,\mathcal F) \to H^i(X,\mathcal F) \]
is an isomorphism.
\end{cor}
\begin{proof}
\todo{write}
\end{proof}

\begin{exmp}\label{exmp:Čech-cohomology-intersections-disjoint-unions-of-contractibles}
If all intersections \(U_{i_0,\ldots,i_n}\) are disjoint unions of contractibles, then \(H^i(U_{i_0,\ldots,i_n},\underline{\bb Z})=0\) for all \(i>0\), so \(\check{H}^i(\mathcal U,\underline{\bb Z})=H^i(X,\underline{\bb Z})\).
\end{exmp}

\begin{exmp}
We will compute the cohomology \(H^i(\sphere[1],\underline{\bb Z})\) of the circle again.
Let \(x_1\) and \(x_2\) denote the `north pole' and `south pole' of the circle, and define \(U_1\coloneq\sphere[1]\setminus\{x_2\}\) and \(U_2\coloneq\sphere[1]\setminus\{x_1\}\); write \(\mathcal U\) for the open cover \(\{U_1\hookrightarrow\sphere[1],U_2\hookrightarrow\sphere[1]\}\).\todo{picture}
The cohomology of the circle can be computed with the Mayer--Vietoris sequence associated to this cover, but we can also use Čech cohomology: since \(U_1\cong U_2\cong\bb R\) and \(U_1\cap U_2\cong\bb R\cotimes\bb R\), we have \(H^i(U_{i_0,\ldots,i_n},\underline{\bb Z})=0\) for \(i>0\).
So we get \(H^i(\sphere[1],\underline{\bb Z}) = \check{H}^i(\mathcal U,\underline{\bb Z})\), the cohomology of the Čech complex which has the form
\begin{equation*}
  \begin{tikzcd}[row sep=tiny, column sep=small]
    \bb Z\oplus\bb Z \ar[r] & \bb Z\oplus\bb Z \ar[r] & 0 \ar[r] & \ldots \\
    {(a,b)} \ar[r, mapsto] & {(b-a,b-a)}
  \end{tikzcd}
\end{equation*}
Thus we obtain:
\[ H^i(\sphere[1],\underline{\bb Z}) =
  \begin{cases}
    \bb Z & \text{if } i = 0,1\text{,} \\
    0 & \text{if } i > 1\text{.}
  \end{cases}
\]
\end{exmp}

\section{Čech cohomology on paracompact Hausdorff spaces}
\remyquote{I'm basically mumbling right now.}

\begin{thm}\label{thm:Čech-cohomology-paracompact-Hausdorff-space}
If \(X\) is a paracompact Hausdorff space and \(\mathcal F\) is an abelian sheaf on \(X\), then there is an isomorphism
\[ \colim_{\mathcal U}\check{H}^i(\mathcal U,\mathcal F) \xrightarrow{\cong} H^i(X,\mathcal F)\text{.} \]
\end{thm}

\begin{defn}
Write \(\catOpenCover{X}\) for the category of open covers of \(X\), whose objects are the open covers \(\mathcal U=\{U_i\hookrightarrow X\}_{i\in I}\) and whose maps are \emph{refining maps}, that is, a map \(\mathcal V=\{V_j\hookrightarrow X\}_{j\in J}\to\mathcal U=\{U_i\hookrightarrow X\}_{i\in I}\) is a function \(\phi\colon J\to I\) such that \(V_j\subseteq U_{\phi(j}\) for all \(j\in J\).
\end{defn}

\begin{exc}[name={Additional exercise~17.3}]
The category \(\catOpenCover{X}\) is not cofiltered, but the homotopy \((1,0)\)-category \(h_0\catOpenCover{X}\) is.
The objects of this category are the same as \(\catOpenCover{X}\), but there is a unique arrow \(\mathcal V\to\mathcal U\) if there exists a refining map \(\mathcal V\to\mathcal U\), and no arrow otherwise.
\end{exc}

\begin{exc}[name={Additional exercise~17.4}]
Čech cohomology defines functors
\[ \check{H}^i \colon h_0\catOpenCover{X}\opp\times\catAbelianPresheaf(X)\to\catAbelianGroup\text{,} \quad (\mathcal U,F)\mapsto\check{H}^i(\mathcal U,F)\text{.} \]
\end{exc}

\begin{defn}\label{defn:Čech-cohomology-colimit}
Define \(\check{H}^i(X,F)\coloneq\colim_{\mathcal U\in h_0\catOpenCover{X}}\check{H}^i(\mathcal U,F)\).
\end{defn}

\begin{rmk}
Note that Čech cohomology is contravariant in the open cover; the colimit of \cref{defn:Čech-cohomology-colimit} can be thought of as the Čech cohomology of \(F\) with respect to the `finest' cover (such a cover might not exist).
\end{rmk}

\todo{write}

%%% Local Variables:
%%% mode: latex
%%% TeX-master: "../main"
%%% End:

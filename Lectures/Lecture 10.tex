\chapter{Exact functors, diagram lemmas}

\section{Exact functors}
\remyquote{For the five lemma, if you look at three different sources, you'll get three different lemmas. That's the three lemma.}
\noindent
\begin{defn}
Let \(\cat C\) and \(\cat D\) be categories and assume \(\cat C\) has finite limits (resp.~finite colimits).
Then a functor \(F\colon\cat C\to\cat D\) is \emph{left exact} (resp.~\emph{right exact}) if it preserves finite limits (resp.~finite colimits).
The functor \(F\) is \emph{exact} if it is both left and right exact.
\end{defn}

The goal of today is to show that this definition agrees with the definition in terms of short exact sequences.

\begin{exmp}\label{exmp:exact-functors}
\begin{itemize}
\item 
  If \(F\) has a right adjoint, then it is right exact.
  If \(F\) \emph{is} a right adjoint (so \emph{has} a left adjoint), then it is left exact.
\item
  We proved that \(f^*\colon\catSheaf(X)\to\catSheaf(Y)\) for \(f\colon Y\to X\) is exact in \cref{lem:pullback-preserves-colimits-and-finite-limits}.
\item
  On Homework~5, we show that sheafification \((-)^\sharp\colon\catPresheaf(X)\to\catSheaf(X)\) (or for abelian sheaves) is exact.
\item
  The forgetful functor \(\catTopologicalSpace\to\catSet\) is exact since it has adjoints on both sides, equipping a set with the discrete or indiscrete topology.
\item
  The forgetful functor \(\catAbelianGroup\to\catSet\) is left exact, but not right exact since \(A\oplus B\neq A\cotimes B\).
\end{itemize}
\end{exmp}

\begin{defn}
Let \(\cat C\) and \(\cat D\) be pre-additive categories.
Then a functor \(F\colon\cat C\to\cat D\) is \emph{additive} if the maps \(\Hom[\cat C](X,Y)\to\Hom[\cat D](FX,FY)\) are group homomorphisms.
\end{defn}

Most functors between abelian categories in nature are additive.

\begin{exmp}
\begin{itemize}
\item
  If \(\cat C\) is a pre-additive category with an object \(X\), then the representable functor \(\Hom[\cat C](-,X)\colon\cat C\opp\to\catAbelianGroup\) is additive: the map
  \[ \Hom[\cat C](Y,Z) \to \Hom[\catAbelianGroup](\Hom[\cat C](Z,X),\Hom[\cat C](Y,X))\text{,} \quad f \mapsto (g\mapsto g\circ f) \]
  is a group homomorphism since composition in a pre-additive category is bilinear.
  Likewise, the corepresentable functor \(\Hom[\cat C](X,-)\colon\cat C\to\catAbelianGroup\) is also additive.
\item
  The free--forgetful adjunction between abelian groups and sets gives a monad (in particular an endofunctor) on \(\catAbelianGroup\), which is not additive: it sends the zero object to \(\bb Z\) and the identity of \(0\), which is also the zero map, must be sent to the identity of \(\bb Z\) by functoriality and to the zero map of \(\bb Z\) by additivity.
\end{itemize}
\end{exmp}

The problem in the last non-example was that the functor did not preserve the zero object.

\begin{lem}\label{lem:characterisation-additive-functor}
Let \(\cat C\) and \(\cat D\) be additive categories and \(F\colon\cat C\to\cat D\) a functor.
Then the following are equivalent:
\begin{enumerate}
\item\label{lem:characterisation-additive-functor:additive} \(F\) is additive;
\item\label{lem:characterisation-additive-functor:products} \(F\) preserves finite products;
\item\label{lem:characterisation-additive-functor:coproducts} \(F\) preserves finite coproducts;
\item\label{lem:characterisation-additive-functor:biproducts} \(F\) preserves binary biproducts.
\end{enumerate}
\end{lem}
\begin{proof}
Note that \cref{lem:characterisation-additive-functor:biproducts} implies that \(F\) preserves the zero object: the biproduct of the zero object and itself is sent to
\begin{equation*}
  \begin{tikzcd}
    0 \ar[r, shift left, "i"] & 0\oplus 0 \ar[l, shift left, "p"] \ar[r, shift right, "q"'] & 0 \ar[l, shift right, "j"']
  \end{tikzcd}
  \qquad
  \overset{F}{\mapsto}
  \qquad
  \begin{tikzcd}
    F(0) \ar[r, shift left, "F(i)"] & F(0)\oplus F(0) \ar[l, shift left, "F(p)"] \ar[r, shift right, "F(q)"'] & F(0) \ar[l, shift right, "F(j)"']
  \end{tikzcd}
\end{equation*}
Then
\[ \id*_{F(0)} = F(\id*_0) = F(0\colon F(0)\to F(0)) = F(qi) = F(q)F(i) = 0\text{,} \]
so \(F(0)\) is a zero object.

Then \cref{lem:characterisation-additive-functor:products}, \cref{lem:characterisation-additive-functor:coproducts} and \cref{lem:characterisation-additive-functor:biproducts} are equivalent, since binary products, binary coproducts and binary biproducts agree by \cref{lem:pre-additive-categories-are-nice}.

Next, assume \(F\) is additive, and let
\begin{equation*}
  \begin{tikzcd}
    X \ar[r, shift left, "i"] & X\oplus Y \ar[l, shift left, "p"] \ar[r, shift right, "q"'] & Y \ar[l, shift right, "j"']
  \end{tikzcd}
\end{equation*}
be a biproduct in \(\cat C\), that is, \(pi=\id_X\), \(qj=\id_Y\) and \(ip+jq=\id_{X\oplus Y}\).
These equations are preserved by \(F\), showing that \cref{lem:characterisation-additive-functor:additive} implies \cref{lem:characterisation-additive-functor:biproducts}.

Conversely, assume \(F\) preserves binary biproducts, and let \(f,g\colon X\to Y\) be maps in \(\cat C\).
We saw in \cref{lem:enrichment-commutative-monoids} that \(f+g\) is the composition
\begin{equation*}
  \begin{tikzcd}
    X \ar[r, "\Delta"] & X\oplus X \ar[r, "f\oplus g"] & Y\oplus Y \ar[r, "\nabla"] & Y\text{.}
  \end{tikzcd}
\end{equation*}
Applying \(F\) gives the diagram
\begin{equation*}
  \begin{tikzcd}
    F(X) \ar[r, "\Delta"] & F(X)\oplus F(X) \ar[r, "F(f)\oplus F(g)"] & F(Y)\oplus F(Y) \ar[r, "\nabla"] & F(Y)\text{.}
  \end{tikzcd}
\end{equation*}
using the universal property of the product to show that \(F(\Delta)=\Delta\), the universal property of the coproduct to show that \(F(\nabla)=\nabla\), and the either universal property to show that \(F(f\oplus g)=F(f)\oplus F(g)\) since we already saw that preserving finite biproducts is equivalent to preserving finite products and finite coproducts.
This shows that \(F(f+g)=F(f)+F(g)\), showing that \cref{lem:characterisation-additive-functor:biproducts} implies \cref{lem:characterisation-additive-functor:additive}.
\end{proof}

\begin{cor}
Any left exact or right exact functor between pre-abelian categories is additive.
\end{cor}

\begin{lem}\label{lem:exactness-short-exact-sequence}
Let \(\cat A\) and \(\cat B\) be abelian categories and \(F\colon\cat A\to\cat B\) a functor.
Then:
\begin{enumerate}
\item \(F\) is left exact if and only if for every short exact sequence \(0\to A\to B\to C\to 0\) in \(\cat A\), the sequence \(0\to F(A)\to F(B)\to F(C)\) is exact in \(\cat B\).
\item \(F\) is right exact if and only if for every short exact sequence \(0\to A\to B\to C\to 0\) in \(\cat A\), the sequence \(F(A)\to F(B)\to F(C)\to 0\) is exact in \(\cat B\).
\item \(F\) is exact if and only if for every short exact sequence \(0\to A\to B\to C\to 0\) in \(\cat A\), the sequence \(0\to F(A)\to F(B)\to F(C)\to 0\) is exact in \(\cat B\).
\end{enumerate}
\end{lem}

\begin{rmk}
For any biproduct
\begin{equation*}
  \begin{tikzcd}
    A \ar[r, shift left, "i"] & A\oplus B \ar[l, shift left, "p"] \ar[r, shift right, "q"'] & B \ar[l, shift right, "j"']
  \end{tikzcd}
\end{equation*}
the sequence \(0\to A\to A\oplus B\to B\to 0\) is exact:
\begin{itemize}
\item the inclusion \(i\colon A\to A\oplus B\) is a (split) monomorphism;
\item the projection \(q\colon A\oplus B\to B\) is a (split) epimorphism;
\item if \(f\colon C\to A\oplus B\) is a map such that the composite \(qf\colon C\to A\oplus B\to B\) is zero, then
  \[ f = (ip+jq)f = ipf+jqf = ipf\text{,} \]
  so \(f\) factors (uniquely) through \(i\).
\end{itemize}
\end{rmk}

\begin{proof}[name={of \cref{lem:exactness-short-exact-sequence}}]
\begin{enumerate}
\item
  Note: if \(F\) takes short exact sequences to left exact sequences, then it is additive: we apply the five lemma to the diagram
  \begin{equation*}
    \begin{tikzcd}
      0 \ar[r] & F(A) \ar[r, "{\begin{pmatrix} \id \\ 0 \end{pmatrix}}"] \ar[d, equal] & F(A)\oplus F(B) \ar[d, "{(F(i),F(j))}"] \ar[r, "{(0,\id)}"] & F(B) \ar[d, equal] \ar[r] & 0 \ar[d] \\
      0 \ar[r] & F(A) \ar[r, "F(i)"'] & F(A\oplus B) \ar[r, "F(q)"'] & F(B) \ar[r] & \coker(F(q))
    \end{tikzcd}
  \end{equation*}
  with exact rows.
  The five lemma shows that \(F\) preserves binary coproducts, so \(F\) is additive.
  The result now follows since \(F\) preserves finite limits if and only if it preserves finite products and kernels, and \(\ker(f\colon B\to C)=\ker(B\twoheadrightarrow\im f)\) so left exactness is equivalent to preserving kernels.
\item Dually.
\item From the previous two items. \qedhere
\end{enumerate}
\end{proof}

\section{Diagram lemmas}
\remyquote{Let me call this proof a sketch, so I can bail out.}
\noindent
The exposition here follows~\cite{IversenCohomologyOfSheaves}.
The following lemma is a variant of the four lemma, a bit more general than what is usually called the four lemma:
\begin{lem}[four lemma]\label{lem:four-lemma}
Let \(\cat A\) be an abelian category and let
\begin{equation*}
  \begin{tikzcd}
    A \ar[r] \ar[d, "a"'] & B \ar[r] \ar[d, "b"] & C \ar[r] \ar[d, "c"] & D \ar[d, "d"] \\
    A' \ar[r] \ar[d] & B' \ar[r] & C' \ar[r] & D' \\
    0
  \end{tikzcd}
\end{equation*}
be a commutative diagram with exact rows and columns.
Then the sequence \(\ker b\to\ker c\to\ker d\) is exact.
\end{lem}

\begin{cor}[five lemma]\label{lem:five-lemma}
Let \(\cat A\) be an abelian category and let
\begin{equation*}
  \begin{tikzcd}
    A \ar[r] \ar[d, "a"'] & B \ar[r] \ar[d, "b"] & C \ar[r] \ar[d, "c"] & D \ar[r] \ar[d, "d"] & E \ar[d, "e"] \\
    A' \ar[r] & B' \ar[r] & C' \ar[r] & D' \ar[r] & E'
  \end{tikzcd}
\end{equation*}
be a commutative diagram with exact rows.
If \(a\) is epic, \(e\) is monic and \(b\) and \(d\) are isomorphisms, then \(c\) is an isomorphism.
\end{cor}
\begin{proof}
Apply the four lemma~\labelcref{lem:four-lemma} to \(a\), \(b\), \(c\) and \(d\) to get an exact sequence
\[0=\ker b\to\ker c\to\ker d=0\text{,} \] so \(\ker c=0\).
Dually, we see \(\coker c=0\), so \(c\) is an isomorphism by \cref{lem:mono-epi-abelian-category}.
\end{proof}

\begin{proof}[name={of \cref{lem:four-lemma}, sketch}]
Let \(E\) and \(E'\) be the images of respectively \(B\to C\) and \(B'\to C'\).
Their uiversal properties give commutative diagrams
\begin{align*}
  \begin{tikzcd}[ampersand replacement=\&]
    A \ar[r] \ar[d, "a"'] \& B \ar[r] \ar[d, "b"] \& E \ar[r] \ar[d, "e"] \& 0 \\
    A' \ar[r] \ar[d] \& B' \ar[r] \& E' \ar[r] \& 0 \\
    0
  \end{tikzcd}
  &&
  \begin{tikzcd}[ampersand replacement=\&]
    0 \ar[r] \& E \ar[r] \ar[d, "e"'] \& C \ar[r] \ar[d, "c"] \& D \ar[d, "d"] \\
    0 \ar[r] \& E' \ar[r] \& C' \ar[r] \& D'
  \end{tikzcd}
\end{align*}
with exact rows and columns.
We will check that the sequences
\[ 0 \to \ker e \to \ker c\to \ker d \]
and
\[ \ker b\to \ker e\to 0 \]
are exact.
Then \(\ker b\to\ker c\) has image \(\ker e=\ker(\ker c\to\ker d)\).

For the former, suppose we have a map \(f\) making the diagram:
\begin{equation*}
  \begin{tikzcd}
    & & X \ar[rd, "0"] \ar[d, "f"'] \\
    & \ker e \ar[r] \ar[d] & \ker c \ar[r] \ar[d] & \ker d \ar[d] \\
    0 \ar[r] & E \ar[r] \ar[d, "e"'] & C \ar[r] \ar[d, "c"] & D \ar[d, "d"] \\
    0 \ar[r] & E' \ar[r] & C' \ar[r] & D'
  \end{tikzcd}
\end{equation*}
commute.
Then \(X\to D\) is zero, so \(X\to C\) factors uniquely through \(E\).
Then \(X\to E'\to C'\) is zero, so \(X\to E'\) is zero, so \(X\to E\) factors uniquely through \(\ker e\).
Dually, the sequence
\[\coker a=0\to\coker b\to\coker e\to 0\]
is exact, so \(\coker b\to\coker e\) is an isomorphism.

We get a commutative diagram
\begin{equation*}
  \begin{tikzcd}
    0 \ar[r] & \im(A'\to B') \ar[r] \ar[d] & B' \ar[d] \ar[r] & E' \ar[r] \ar[d] & 0 \\
    0 \ar[r] & 0 \ar[r] & \coker b \ar[r, iso'] & \coker e \ar[r] & 0
  \end{tikzcd}
\end{equation*}
with exact rows, so by the first statement, we get an exact sequence
\begin{equation*}
  \begin{tikzcd}
    0 \ar[r] & \im(A'\to B') \ar[r] & \im b \ar[r] & \im e \ar[r] & 0 \\
    & A' \ar[u, epi] & B \ar[u, epi] \ar[r, epi] & E \ar[u, epi]
  \end{tikzcd}
\end{equation*}
so also an exact sequence
\[ A'\to \im b\to \im e\to 0\text{.} \]
Now we get
\begin{equation*}
  \begin{tikzcd}
    A \ar[r] \ar[d, "a'"', epi] & B \ar[r] \ar[d, "b'", epi] & E \ar[r] \ar[d, "e'", epi] & 0 \\
    A' \ar[r] & \im b \ar[r] & \im e \ar[r] & 0
  \end{tikzcd}
\end{equation*}
where \(\ker b'=\ker b\) and \(\ker e'=\ker e\).
Then we claim that \(\im e = \coker(\ker b'\to E)\): given a map \(g\) making the diagram
\begin{equation*}
  \begin{tikzcd}
    & \ker b' \ar[rrd, "0"] \ar[d] \\
    A \ar[r] \ar[d, "a'"', epi] & B \ar[r] \ar[d, "b'", epi] & E \ar[r, "g"'] \ar[d, "e'", epi] & Y \\
    A' \ar[r] & \im b \ar[r] & \im e \ar[r] & 0
  \end{tikzcd}
\end{equation*}
commute, the composite \(B\to Y\) factors uniquely through \(\im b\).
Then \(A\to Y\) is zero, so \(A'\to Y\) is zero as \(a\) is an epimorphism.
So \(\im b\to Y\) factors uniquely through \(\im e\).
Then
\[\ker b'\to E\xrightarrow{e'}\im e\to 0\]
is exact, so the map \(\ker b'\to\ker e'\) is an epimorphism.
\end{proof}

\begin{lem}[name=Snake lemma]
Given a commutative diagram 
\[\begin{tikzcd}
	& A & B & C & 0 \\
	0 & {A'} & {B'} & {C'}
	\arrow[from=1-2, to=1-3]
	\arrow["a"', from=1-2, to=2-2]
	\arrow[from=1-3, to=1-4]
	\arrow["b"', from=1-3, to=2-3]
	\arrow[from=1-4, to=1-5]
	\arrow["c"', from=1-4, to=2-4]
	\arrow[from=2-1, to=2-2]
	\arrow[from=2-2, to=2-3]
	\arrow[from=2-3, to=2-4]
\end{tikzcd}\] in an abelian category, if the rows are exact then there is an exact sequence \[
	\ker (a) \to \ker (b) \to \ker (c) \xrightarrow{\partial} \coker (a) \to \coker (b) \to \coker (c).
\] The map $\partial$ is often called the \emph{connecting homomorphism}. 
\end{lem}

%%% Local Variables:
%%% mode: latex
%%% TeX-master: "../main"
%%% End:

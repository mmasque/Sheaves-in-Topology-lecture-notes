\chapter*{Introduction}

These notes are based on lectures given by Remy van Dobben de Bruyn for the Master's course \emph{Sheaves in Topology}, taught at Utrecht University in the spring semester of 2023--2024.\footnote{See~\url{https://cursusplanner.uu.nl/course/WISM501/2023/SEM2} for the course description.}

The prerequisites for this course are a solid understanding of point-set topology, basic knowledge of fundamental groups and covering spaces, familiarity with the language of categories, and a working knowledge of modules over rings. 

\section*{Recommended literature}
Standard works:
\NewDocumentCommand\citeauthortitlecite{om}{\citeauthor{#2}, \citetitle{#2}~\IfNoValueTF{#1}{\cite{#2}}{\cite[#1]{#2}}}
\begin{itemize}
\item \citeauthortitlecite{IversenCohomologyOfSheaves}
\item \citeauthortitlecite{BredonSheafTheory}
\item \citeauthortitlecite{TennisonSheafTheory}
\item \citeauthortitlecite{KashiwaraSchapiraSheavesOnManifolds}
\item \citeauthortitlecite[\href{https://stacks.math.columbia.edu/tag/006A}{Chapter~006A}]{stacks-project} (chapter on sheaves)
\end{itemize}
\noindent
More advanced texts:
\begin{itemize}
\item \citeauthortitlecite{DimcaSheavesInTopology}
\item \citeauthortitlecite{MacLaneMoerdijkSheavesGeometryLogic}
\end{itemize}
\noindent
Exodromy correspondence (research papers):
\begin{itemize}
\item \citeauthortitlecite{TreumannExitPathsConstructibleStacks}
\item \citeauthortitlecite{CurryPatelClassificationConstructibleCosheaves}
\end{itemize}
\section*{Course content}
The first four lectures introduce presheaves and sheaves on a topological space $X$ and describe an equivalence of categories between local homeomorphisms over $X$ and sheaves on $X$. For the special case of locally constant sheaves there is an equivalence to the category of covering spaces of $X$. 

Some categorical properties of sheaves, and constructions such as the pushforward and the pullback, are discussed next. After an introduction to homological algebra (which is independent of the content on sheaves), the notion of sheaf cohomology is treated. This takes up \cref{lecture:8}, \cref{lecture:9}, \cref{lecture:10}, \cref{lecture:11}, and \cref{lecture:12}. 

The real fun begins when the homological algebra is applied to sheaves of abelian groups. One of the main results of the course, the proper base change theorem, is proven in \cref{lecture:15} for paracompact Hausdroff and locally compact Hausdorff spaces. The treatment of sheaf cohomology ends with a discussion of Čech cohomology. 

The last three weeks (\cref{lecture:18}, \cref{lecture:19}, \cref{lecture:20}) are reserved entirely to having fun, and as such were not examinable material in the 2024 version of the course. \todo{write after we write \cref{lecture:19}.} 

\todo{course content}


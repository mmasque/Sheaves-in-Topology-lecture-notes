\chapter{Acyclic resolutions, supports}
\label{lecture:13}

\section{Acyclic resolutions}
\remyquote{``For the benefit of those millenials who believe that the Godement resolution is one of the founding documents of the United Nations, here is a translation of the above construction into contemporary language.''~\cite[Remark~7.1]{HarrisVenkateshDerivedHeckeAlgebras}}
\noindent
In this section, let \(\cat A\) and \(\cat B\) be abelian categories and assume \(\cat A\) has enough injectives.

\begin{defn}
Let \(F\colon\cat A\to\cat B\) be a left exact functor.
Then an object \(A\) of \(\cat A\) is \emph{\(F\)-acyclic} if \(\rightderived^iF(A)=0\) for \(i>0\).
\end{defn}

\begin{exmp}
If \(I\) is an injective object of \(\cat A\), then it is \(F\)-acyclic for any \(F\).
Indeed, take the injective resolution
\[ 0 \to I \to \underbracket{I \to 0 \to 0 \to \ldots} \]
of \(I\).
Then
\[ \rightderived^iF(I) = H^i(F(I)\to 0\to 0\to\ldots) =
  \begin{cases}
    F(I) & \text{if } i=0\text{,} \\
    0 & \text{if } i>0\text{.}
  \end{cases}
\]
\end{exmp}

The following lemma says that the right-derived functors of a left exact functor \(F\) can also be computed using \(F\)-acyclic resolutions rather than injective resolutions.

\begin{lem}\label{lem:compute-right-derived-functor-acyclic-resolution}
Let \(F\colon\cat A\to\cat B\) be a left exact functor, let \(A\) be an object of \(\cat A\) and let
\[0\to A\to C^0\to C^1\to\ldots\]
be a resolution of \(A\) with \(F\)-acyclic \(C^i\) for all \(i\).
Then \(\rightderived^iF(A)\cong H^i(F(C^\bullet))\) for all \(i\).
\end{lem}
\begin{proof}
We proceed by induction on \(i\).

For \(i=0\), left exactness gives
\[ H^0(F(C^\bullet)) \cong \ker(F(C^0)\to F(C^1)) \cong F(A) \cong \rightderived^0F(A)\text{.} \]
(We have not shown this yet, but it is not too hard to show that \(\rightderived^0F\cong F\).)

Let \(B\coloneq\im(C^0\to C^1)\cong\ker(C^1\to C^2)\).
By breaking of the resolution using \(B\), we get exact sequences
\begin{equation}
  \label{eq:compute-right-derived-functor-acyclic-resolution-ses}
  0\to A\to C^0\to B\to 0 
\end{equation}
and
\begin{equation}
  \label{eq:compute-right-derived-functor-acyclic-resolution-les}
  0\to B\to C^1\to C^2\to \ldots
\end{equation}
The short exact sequence~\cref{eq:compute-right-derived-functor-acyclic-resolution-ses} gives\todo{ref long exact sequence} a long exact sequence
\[ 0\to F(A)\to F(C^0)\to F(B)\to \rightderived^1F(A)\to \underbrace{\rightderived^1F(C^0)}_{=0}\to \rightderived^1F(B)\to \rightderived^2F(A)\to \underbrace{\rightderived^2F(C^0)}_{=0}\to \ldots \]
in which every third term is (eventually) zero since \(C^0\) is \(F\)-acyclic; hence we obtain isomorphisms \(\rightderived^iF(B)\cong \rightderived^{i+1}F(A)\) by exactness.
The long exact sequence \cref{eq:compute-right-derived-functor-acyclic-resolution-les} gives an exact sequence
\[ 0\to F(B)\to F(C^1)\to F(C^2) \]
by left exactness of \(F\).
Thus, we have
\[ \rightderived^1F(A) \cong \frac{F(B)}{F(C^0)} \cong \frac{\ker(F(C^1)\to F(C^2))}{\im(F(C^0)\to F(C^1))} \cong H^1(F(C^\bullet))\text{,} \]
proving the result for \(i=1\).

For \(i>1\), assume the result for \(i-1\) (and for all \(A\) and \(C^\bullet\)).
Applying the induction hypothesis to the \(F\)-acyclic resolution \cref{eq:compute-right-derived-functor-acyclic-resolution-les} of \(B\), we find
\[ \rightderived^iF(A) \cong \rightderived^{i-1}F(B) \cong H^{i-1}(F(C^1)\to F(C^2)\to\ldots) = H^i(F(C^0)\to F(C^1)\to\ldots)\text{.} \qedhere \]
\end{proof}

So far, this lemma does not say anything specific about sheaves.
The next lemma will be an important example for us.
Recall that a sheaf \(\mathcal F\) is \emph{flasque} if the restriction \(\mathcal F(V)\to\mathcal F(U)\) is surjective for opens \(U\subseteq V\).
We will use the result of the following exercise, which appears as Additional exercise~12.4.

\begin{exc}\label{exc:additional-exercise-12.4}
Let \(0\to\mathcal F\to\mathcal G\to\mathcal H\to 0\) be a short exact sequence of abelian sheaves on a space \(X\).
Show the following:
\begin{enumerate}
\item If \(\mathcal F\) is flasque and \(U\subseteq X\) is open, then \(0\to\mathcal F(U)\to\mathcal G(U)\to\mathcal H(U)\to 0\) is exact.
  (In other words, the functor \(\mathcal F\mapsto\mathcal F(U)\) is exact.)
\item If \(\mathcal F\) and \(\mathcal G\) are flasque, then so is \(\mathcal H\).
\end{enumerate}
\end{exc}

\begin{lem}\label{lem:flasque-sheaf-evaluation-acyclic}
Let \(\mathcal F\) be a flasque abelian sheaf on a space \(X\).
Then \(\mathcal F\) is \(H^0(U,-)\)-acyclic for any open \(U\subseteq X\), that is, \(H^i(U,\mathcal F)=0\) for \(i>1\).
\end{lem}
\begin{proof}
Choose an injection \(\mathcal F\hookrightarrow\mathcal I\) into an injective sheaf and let \(\mathcal H\) be the quotient, giving a short exact sequence
\[ 0\to\mathcal F\to\mathcal I\to\mathcal H\to 0\text{.}\]
Injective sheaves are flasque (\cref{exmp:injective-sheaves-are-flasque}), so by \cref{exc:additional-exercise-12.4}, \(\mathcal H\) is too and
\[ 0\to\mathcal F(U)\to\mathcal I(U)\to\mathcal H(U)\to 0 \]
is exact for any open \(U\subseteq X\).
Since injective sheaves are acyclic so \(H^i(U,\mathcal I)=0\), we get \(H^i(U,\mathcal F)=0\) for \(i>0\) and \(H^i(U,\mathcal F)\cong H^{i-1}(U,\mathcal H)\) for \(i>1\) from the long exact sequence.
Since \(\mathcal H\) is flasque, we conclude by induction on \(i\).
(This argument is similar to what we did in the proof of \cref{lem:compute-right-derived-functor-acyclic-resolution}.)
\end{proof}

\begin{exmp}
If \(X\) is discrete, then any sheaf on \(X\) is flasque, so \(H^i(X,\mathcal F)=0\) for all \(i>0\) and all abelian sheaves \(\mathcal F\) on \(X\).
\end{exmp}

\begin{exmp}
If \(f\colon Y\to X\) is continuous and \(\mathcal F\) is a flasque abelian sheaf on \(Y\), then \(f_*\mathcal F\) is flasque: if \(U\subseteq V\subseteq X\) are open, then the diagram
\begin{equation*}
  \begin{tikzcd}
    f_*\mathcal F(V) \ar[d, equal] \ar[r] & f_*\mathcal F(U) \ar[d, equal] \\
    \mathcal F(f\inv(V)) \ar[r, epi] & \mathcal F(f\inv(U))
  \end{tikzcd}
\end{equation*}
commutes, so surjectivity of the bottom map (by flasqueness of \(\mathcal F\)) implies surjectivity of the top map.
\end{exmp}

\begin{rmk}\label{rmk:Godement-resolution}
Recall that we constructed enough injectives in \(\catAbelianSheaf(X)\) as follows: take \(f\colon {X^{\disc}\hookrightarrow X}\) and choose an injection \(f^*\mathcal F\hookrightarrow\mathcal I\) into an injective in \(\catAbelianGroup(X^{\disc})\).
Then the composite \[\mathcal F\hookrightarrow f_*f^*\mathcal F\hookrightarrow f_*\mathcal I\] gives our required injection into an injective object.
But the map \(\mathcal F\hookrightarrow f_*f^*\mathcal F\) is already an injection into a flasque sheaf.
Iterating this construction gives a canonical flasque resolution
\[ 0\to\mathcal F\to f_*f^*\mathcal F\to f_*f^*(\coker(\mathcal F\to f_*f^*\mathcal F))\to\ldots\text{,} \]
which is called the \emph{Godement resolution} of \(\mathcal F\).
This resolution has some advantages: it is very explicit, functorial and additive.
\end{rmk}

\section{Supports}

\begin{defn}
Let \(\mathcal F\) be an abelian sheaf on \(X\) and let \(s\in\mathcal F(U)\) be a section.
Then the support of \(s\) is the set \(\supp(s) \coloneq \setpred{x\in U}{s_x\neq 0}\).
\end{defn}

\begin{lem}\label{lem:support-closed}
If \(s\in\mathcal F(U)\) is a section, then \(\supp(s)\subseteq U\) is closed.
\end{lem}
\begin{proof}
The locus where the two sections \(0,s\colon U\to\etalespace(\mathcal F)\) agree is open since the diagonal \[\Delta_{\etalespace(F)/X}\subseteq\etalespace(F)\pullback{X}\etalespace(\mathcal F)\] is open.\todo{why?}
\end{proof}

\begin{defn}
For a topological space \(X\), define a functor
\[ H^0_c(X,-)\colon\catAbelianSheaf(X)\to\catAbelianGroup\text{,}\quad\mathcal F\mapsto\setpred{s\in\mathcal F(X)}{\supp(s)\text{ is compact}}\text{,} \]
sending a sheaf to its global sections with compact support.
This functor is left exact and the derived functors \(H^i_c(X,-)\coloneq\rightderived^iH^0_c(X,-)\) are the \emph{sheaf cohomology with compact support}.
\end{defn}

The following lemma shows that ordinary sheaf cohomology and sheaf cohomology with compact support agree for all sheaves on compact spaces.

\begin{lem}\label{lem:compactly-supported-cohomology-is-ordinary-cohomology-on-compact-space}
Let \(\mathcal F\) be an abelian sheaf on a compact space \(X\).
Then \(H^i(X,\mathcal F)\cong H^i_c(X,\mathcal F)\) for all \(i\geq 0\).
\end{lem}
\begin{proof}
By \cref{lem:support-closed}, the support of any global section \(s\in\mathcal F(X)\) is closed in the compact space \(X\), so also compact.
Hence we have \(H^0_c(X,-) = \Gamma(X,-) = H^0(X,-)\), and then also their derived functors \(H^i_c(X,-)\) and \(H^i(X,-)\) agree for all \(i\geq 0\).
\end{proof}

We will derive properties of cohomology \(H^i\) and compactly supported cohomology \(H^i_c\) in parallel.
A more general unified strategy using `families of supports' is presented in \cite[\S~\RN{2}.9]{BredonSheafTheory}.

For a sheaf \(\mathcal F\) on a space \(X\) and any subspace \(i\colon Y\hookrightarrow X\), we write \(\mathcal F|_Y\) for the \emph{restriction} of \(\mathcal F\) to \(Y\), which is defined to be the pullback \(i^*\mathcal F\) of \(\mathcal F\) along \(i\).

\begin{defn}
An abelian sheaf \(\mathcal F\) on a space \(X\) is \emph{soft} (resp.~\emph{c-soft}) if the restriction \(\mathcal F(X)\to\mathcal F|_Z(Z)\) is surjective for every closed (resp.~compact closed) subset \(Z\subseteq X\).
\end{defn}

By abuse of notation, we write \(\mathcal F(W)=\mathcal F|_W(W)\) when \(W\subseteq X\) is locally closed.

\begin{rmk}
Some authors (such as~\cite{IversenCohomologyOfSheaves}) say `soft' for what we call `c-soft'.
\end{rmk}

We will show that soft (resp.~c-soft) sheaves are \(H^0(X,-)\)-acyclic (resp.~\(H^0_c(X,-)\)-acyclic) under suitable assumptions on the space \(X\).

\section{Paracompactness}
\remyquote{We're diving into point-set topology, it's happening: it's no longer a category theory course.}
% \remyquote{Recall that everything in~\cite{MunkresTopology} is trivial.}
\noindent
Recall that a space \(X\) is \emph{paracompact} if every open cover \(X=\bigcup_{i\in I}U_i\) has a refinement \(\bigcup_{j\in J}V_j=X\) (that is, there exists a function \(\phi\colon J\to I\) with \(V_j\subseteq U_{\phi(j)}\) for all \(j\in J\)) that is locally finite: every point \(x\in X\) has an open neighbourhood \(U\) meeting finitely many \(V_j\).

\begin{exmp}
All compact spaces, metrisable spaces and CW-complexes are paracompact.
Most authors \emph{assume} manifolds to be paracompact.
\end{exmp}

We will use the following point-set topological result about paracompact spaces.

\begin{lem}[name={\cite[Lemma~41.6]{MunkresTopology}}]\label{lem:paracompact-locally-finite-cover}
If \(X\) is a paracompact Hausdorff space and \(\bigcup_{i\in I}U_i=X\) is an open cover, then there exists a locally finite cover \(\bigcup_{i\in I}V_i=X\) with \(\overline{V_i}\subseteq U_i\) for all \(i\in I\).
\end{lem}

Recall also that a space $X$ is \emph{locally compact} if every point has a \emph{compact neighborhood}, i.e., if there exists for every $x \in X$ an open $U$ containing $x$ and a compact set $K \supset U$.

\begin{prop}\label{prop:restriction-map-isomorphism-paracompact}
Let \(X\) be a Hausdorff space, let \(i\colon Z\hookrightarrow X\) be a closed subspace and let \(\mathcal F\) be an abelian sheaf on \(X\).
Then the map
\begin{equation}
  \label{eq:restriction-map-isomorphism-paracompact}
  \colim_{Z\subseteq U\text{, }U\textup{ open}}\mathcal F(U) \xrightarrow{(-)|_Z} \mathcal F(Z)
\end{equation}
is an isomorphism if:
\begin{enumerate}
\item\label{prop:restriction-map-isomorphism-paracompact:paracompact} \(X\) is paracompact, or
\item\label{prop:restriction-map-isomorphism-paracompact:locally-compact} \(X\) is locally compact and \(Z\) is compact.
\end{enumerate}
\end{prop}
\begin{proof}[sketch]
Recall that \(\colim_{Z\subseteq U}\mathcal F(U)=(i^\circledast\mathcal F)(Z)\) (see \cref{prop:pullback-Kan-extension}) and we defined \(\mathcal F(Z)\coloneq(i^*\mathcal F)(Z)\).

The map \cref{eq:restriction-map-isomorphism-paracompact} is always injective: suppose \(s\in\mathcal F(U)\) is a section with \(s|_Z=0\).
Then the support of \(s\) is a closed subset (by \cref{lem:support-closed}) of the complement \(U\setminus Z\), so \(V\coloneq U\setminus\supp(s)\) is an open neighbourhood of \(Z\) such that \(s|_V=0\).

For surjectivity, we treat the cases separately:
\begin{enumerate}
\item
  If \(s\in\mathcal F(Z)\) is a section, then it \emph{locally} extends to an open neighbourhood: there exist opens \(U_i\subseteq X\) such that \(Z\subseteq\bigcup_{i\in I}U_i\) and sections \(t_i\in\mathcal F(U_i)\) with \(t|_{Z\cap U_i}=s|_{Z\cap U_i}\) for all \(i\in I\).
  Without loss of generality, we may assume that the open cover \((U_i)_{i\in I}\) is locally finite since closed subspaces of paracompact spaces are paracompact.
  (This is \cite[Theorem~41.2]{MunkresTopology}; the idea of the proof is to add \(X\setminus Z\) to the cover and restrict the refined cover by intersecting with \(Z\).)
  Using \cref{lem:paracompact-locally-finite-cover}, choose an open cover \(Z=\bigcup_{i\in I}V_i=X\) with \(\overline{V_i}\subseteq U_i\) for all \(i\in I\).
  Set
  \[ W\coloneq\setpred{x\in X}{x\in\overline{V_i}\cap\overline{V_j}\implies t_{i,x}=t_{j,x}}\text{,} \]
  which contains \(Z\) and write \(J(x)\coloneq\setpred{i\in I}{x\in\overline{V_i}}\) for a point \(x\in X\).
  Then for all \(x\), the set \(J(x)\) is finite and there is an open neighbourhood \(U\) of \(x\) such that \(J(y)\subseteq J(x)\) for all \(y\in U\): indeed, there exists an open neighbourhood \(V\) of \(x\) meeting finitely only many \(U_i\), and we can take
  \[ U\coloneq V\setminus\bigcup_{\substack{i\in I \\ U_i\cap V\neq\emptyset \\ x\notin V_i}}\overline{V_i}\text{.} \]
  If \(x\in W\), the sections \(t_{i,x}\) for \(i\in J(x)\) all agree, so the same holds in a neighbourhood of \(x\) since \(J(x)\) is finite.
  Hence \(W\) is open and the \(t_i\) glue to a well-defined section \(t\in\mathcal F(W)\).
\item
  Since \(X\) is locally compact, there exists a compact neighbourhood \(W\) of \(Z\) in \(X\).
  Replace \(X\) by \(W\) and use part~\cref{prop:restriction-map-isomorphism-paracompact:paracompact}.
  \qedhere
\end{enumerate}
\end{proof}

\begin{cor}
If \(\mathcal F\) is a flasque abelian sheaf on \(X\), then \(\mathcal F\) is:
\begin{enumerate}
\item soft if \(X\) is paracompact Hausdorff, and
\item c-soft if \(X\) is locally compact Hausdorff.
\end{enumerate}
\end{cor}

\begin{prop}\label{prop:global-sections-exact-soft}
Let \(0\to\mathcal F\to\mathcal G\to\mathcal H\to 0\) be a short exact sequence of abelian sheaves on a space \(X\).
\begin{enumerate}
\item\label{prop:global-sections-exact-soft:soft}
  If \(X\) is paracompact Hausdorff and \(\mathcal F\) is soft, then the short sequence
  \[ 0\to\mathcal F(X)\to\mathcal G(X)\to\mathcal H(X)\to 0 \]
  is exact.
\item\label{prop:global-sections-exact-soft:c-soft}
  If \(X\) is locally compact Hausdorff and \(\mathcal F\) is c-soft, then the short sequence
  \[ 0\to H^0_c(X,\mathcal F)\to H^0_c(X,\mathcal G)\to H^0_c(X,\mathcal H)\to 0 \]
  is exact.
\end{enumerate}
\end{prop}

Before proving the proposition, we note the following consequence.

\begin{cor}
A soft (resp.~c-soft) sheaf on a paracompact (resp.~locally compact) Hausdorff space \(X\) is \(H^0(X,-)\)-acyclic (resp.~\(H^0_c(X,-)\)-acyclic).
\end{cor}
\begin{proof}
Similar to the proof of \cref{lem:flasque-sheaf-evaluation-acyclic}, using a result similar to \cref{exc:additional-exercise-12.4} for softness or c-softness instead of flasqueness.
\end{proof}

\begin{proof}[name={of \cref{prop:global-sections-exact-soft}, sketch}]
\begin{enumerate}
\item
  Exactness on the left is general, so it remains to prove that \(\mathcal G(X)\to\mathcal H(X)\) is surjective.
  Let \(s\in\mathcal H(Z)\) be a section and let \(\bigcup_{i\in I}U_i=X\) be a locally finite cover with \(s_i\in\mathcal G(U_i)\) lifting \(s|_{U_i}\) for all \(i\in I\).
  By \cref{lem:paracompact-locally-finite-cover}, choose a locally finite cover \(\bigcup_{i\in I}V_i=X\) with \(\overline{V_i}\subseteq U_i\) for all \(i\in I\).
  \emph{Now we are going to do one of the most ugly tricks in the book}:
  Put all well-order on \(I\) and let \(I^\triangleright\coloneq I\cup\{\infty\}\) be the successor of \(I\), that is, with \(i<\infty\) for all \(i\in I\).
  For \(i\in I^\triangleright\) set \(W_i\coloneq\bigcup_{j<i}\overline{V_j}\); in particular, \(W_\infty=X\).
  Since the open cover \((U_i)_{i\in I}\) is locally finite, each \(W_i\) is closed.
  Using transfinite induction we define \(t_i\in\mathcal G(W_i)\) for all \(i\in I^\triangleright\) lifting \(s|_{W_i}\) such that \(t_i|_{W_j}=t_j\) if \(j\leq i\).

  For the smallest element \(i_0\) of \(I^\triangleright\), we have \(W_{i_0}=\emptyset\), so we must choose \(t_{i_0}=0\in\mathcal F(W_{i_0})=\mathcal F(\emptyset)=0\), and this clearly lifts \(s|_{W_i}=0\).

  In the successor case \(i=j+1\) for \(j\in I\), then \(t_j\) and \(s_i\) both lift \(s|_{W_j\cap\overline{V}_j}\), so their difference is in \(\mathcal F(W_j\cap\overline{V_j})\) (by exactness of pullback, \cref{lem:pullback-preserves-colimits-and-finite-limits}, restriction to a closed subspace is exact).
  This difference extends to some \(t_{ij}\in\mathcal F(X)\) since \(\mathcal F\) is soft, and then \(t_j\in\mathcal G(W_j)\) and \(s_i-t_{ij}|_{\overline{V_j}}\in\mathcal G(\overline{V_j})\) glue to some \(t_i\in\mathcal G(W_i)\).
  (One of the additional exercises shows you can do this.\todo{which one?})

  Finally, if \(i\) is a limit ordinal, then glue \(t_j\) for \(j<i\) to get \(t_i\).
\item
  If \(s\in H^0_c(X,\mathcal H)\) is a section with compact support, then apply part~\cref{prop:global-sections-exact-soft:soft} to a compact neighbourhood \(W\) of \(\supp(s)\) to get \(t\in\mathcal G(W)\) lifting \(s|_W\).
  On the boundary \(\partial W\) of \(W\), we have \(s|_{\partial W}=0\), so \(t|_{\partial W}\) is in \(\mathcal F(\partial W)\).
  Lift to a global section of \(\mathcal F\), subtract and extend by zero.
  \qedhere
\end{enumerate}
\end{proof}

The conclusion of all the work we did is the following picture for abelian sheaves on a Hausdorff space \(X\):
\begin{equation*}
  \begin{tikzcd}[arrows=Rightarrow, column sep=small, row sep=small]
    & & \textup{soft} \ar[r] & \textup{\(H^0(X,-)\)-acyclic} \\
    \textup{injective} \ar[r] & \textup{flasque} \ar[ru] \ar[rd] \\
    & & \textup{c-soft} \ar[r] & \textup{\(H^0_c(X,-)\)-acyclic}
  \end{tikzcd}
\end{equation*}
The top implications hold under the assumption that \(X\) is paracompact and the bottom implications under the assumption that \(X\) is locally compact.

%%% Local Variables:
%%% mode: latex
%%% TeX-master: "../main"
%%% End:

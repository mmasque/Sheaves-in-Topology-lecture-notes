\chapter{Derived functors, sheaf cohomology}
This week's lecturer was Dr. Soumya Sankar. 

\section{Derived functors}

Let $\cochaincomplex{C}$ be a cochain complex. 
	The kernels $Z^i:= \ker(d^i: C^i \to C^{i+1})$ are called cocycles, and the images $B^i := \im(d^{i-1}: C^{i-1} \to C^i)$ are called coboundaries. Check that the universal property of the kernel gives a map $B^i \to Z^i$. 
\begin{defn}
	The cokernel $\coker(B^i \to Z^i)$ is the $i$th cohomology of $\cochaincomplex{C}$. It is denoted by $\cohomology^i(\cochaincomplex{C})$. One checks that in abelian groups $\cohomology^i(\cochaincomplex{C})$ is the quotient $Z^i/B^i$. \todo{this is not so clear...}
	Given a morphism $\cochaincomplex{C} \to \cochaincomplex{D}$, taking the $i$th cohomology functorially induces a map $\cohomology^i{\cochaincomplex{C}} \to \cohomology^i{\cochaincomplex{D}}$.
\end{defn}
If $\cochaincomplex{C}$ is exact, $\cohomology^i(\cochaincomplex{C}) = 0$ for all $i$. 

\begin{exc}
	Let $\cochaincomplex{C} = 0 \to X \to I^0 \to I^1 \to ... $ be an injective resolution. Then $\cohomology^0(\cochaincomplex{C}) = X$ and $\cohomology^i(\cochaincomplex{C}) = 0$ for $i > 0$. 
\end{exc}

\begin{lem}
	Let $0 \to \cochaincomplex{A} \to \cochaincomplex{B} \to \cochaincomplex{C} \to 0$ be a short exact sequence of chain complexes. Then there is a canonical long exact sequence 
	\[
    	\cdots \to \cohomology^{i-1}(\cochaincomplex{A}) \to \cohomology^{i-1}(\cochaincomplex{B}) \to \cohomology^{i-1}(\cochaincomplex{C}) \to \cohomology^{i}(\cochaincomplex{A}) \to \cohomology^{i}(\cochaincomplex{B}) \to \cohomology^{i}(\cochaincomplex{C}) \to \cdots 
    \]
\end{lem}
\begin{proof}
	The proof from the lecture used abelian groups. \todo{rewrite in general abelian categories}. 
\end{proof}

\begin{defn}
Let \(F\colon\cat A\to\cat B\) be a left exact functor between abelian categories and suppose \(\cat A\) has enough injectives.
Define the \emph{\(i\)th right derived functor} \(\rightderived^iF\colon\cat A\to\cat B\) as follows: for an object \(X\) of \(\cat A\), choose an injective resolution
\[ 0\to X\to I^0\to I^1\to I^2\to\ldots\text{,} \]
apply \(F\) levelwise to the resolution to obtain the complex
\[ 0\to FI^0\to FI^1\to FI^2\to\ldots \]
and set \(\rightderived^iF(X)\coloneq H^i(F(I^\bullet))\) to be the \(i\)th cohomology of this complex.
\end{defn}

In the remainder of this lecture, we will verify that this definition makes sense and we apply it to introduce \emph{sheaf cohomology}.
Additional exercise~12.3 shows the functoriality of this construction.

\section{Homotopies and quasi-isomorphisms}
\begin{defn}
Let $\cochaincomplex{f}, \cochaincomplex{g}: \cochaincomplex{A} \to \cochaincomplex{B}$ be morphisms of cochain complexes. A \emph{homotopy} between $\cochaincomplex{f}$ and $\cochaincomplex{g}$ is a collection of maps $h^i: A^i \to B^{i-1}$ such that \[
	f^i - g^i = d^{i-1}_B h^i + h^{i+1}d^i_A.
\]
The maps $\cochaincomplex{f}$ and $\cochaincomplex{g}$ are \emph{homotopic}, we write $\cochaincomplex{f} \sim \cochaincomplex{g}$. It helps to keep the following `parallelogram diagram' in mind: 
\[\begin{tikzcd}
	{\cdots } & {A^{i-1}} && {A^i} && {A^{i+1}} & \cdots \\
	\cdots & {B^{i-1}} && {B^{i}} && {B^{i+1}} & \cdots
	\arrow[from=1-1, to=1-2]
	\arrow[from=1-2, to=1-4]
	\arrow[from=1-2, to=2-2]
	\arrow["{d^i_A}"', from=1-4, to=1-6]
	\arrow["{h^i}"{description}, dashed, from=1-4, to=2-2]
	\arrow["{f^i - g^i}"{description}, from=1-4, to=2-4]
	\arrow[from=1-6, to=1-7]
	\arrow["{h^{i+1}}"{description}, from=1-6, to=2-4]
	\arrow[from=1-6, to=2-6]
	\arrow[from=2-1, to=2-2]
	\arrow["{d^{i-1}_B}"', from=2-2, to=2-4]
	\arrow[from=2-4, to=2-6]
	\arrow[from=2-6, to=2-7]
\end{tikzcd}\]
If a morphism is homotopic to the zero morphism, it is called \emph{nullhomotopic}.
\end{defn}

\begin{defn}
	A map $\cochaincomplex{f}: \cochaincomplex{A} \to \cochaincomplex{B}$ is a \emph{homotopy equivalence} if there exists a map $\cochaincomplex{g}: \cochaincomplex{B} \to \cochaincomplex{A}$ such that $\cochaincomplex{g} \circ \cochaincomplex{f} \sim \id_{\cochaincomplex{g}}$ and $\cochaincomplex{g} \sim \id_{\cochaincomplex{f}}$. 
\end{defn}

\begin{defn}
	A map $\cochaincomplex{f}: \cochaincomplex{A} \to \cochaincomplex{B}$ is a \emph{quasi-isomorphism} if the induced map \[
    	\cohomology^i{\cochaincomplex{f}}: \cohomology^i{\cochaincomplex{A}} \to \cohomology^i{\cochaincomplex{B}}
    \] is an isomorphism for all $i \in \mathbb{Z}$. 
\end{defn}

\begin{exc}\label{exc:properties-of-homotopy}
	Let $\cochaincomplex{f}, \cochaincomplex{g}: \cochaincomplex{A} \to \cochaincomplex{B}$ be morphisms of cochain complexes. 
	\begin{enumerate}[label=(\alph*)]
    	\item If $\cochaincomplex{f} \sim \cochaincomplex{g}$ then $\cochaincomplex{f}$ and $\cochaincomplex{g}$ induce the same maps on homology. 
		\item If $\cochaincomplex{f}$ is a homotopy equivalence then it is a quasi isomorphism.
		\item If $F: \cat A \to \cat B$ is an additive functor between abelian categories and $h$ is a homotopy between $\cochaincomplex{f}$ and $\cochaincomplex{g}$ then $F(h)$ is a homotopy between $F(\cochaincomplex{f})$ and $F(\cochaincomplex{g})$\footnote{This is a slight abuse of notation. A homotopy is not a morphism of complexes, but a set of maps $h^i$ for $i \in \mathbb{Z}$. Here $F(h)$ means we apply $F$ to each $h^i$}.
		\item If $f$ is a homotopy equivalence and $F$ is additive then $F(f)$ is a quasi-isomorphism. 
    \end{enumerate}
\end{exc}

\begin{lem}\label{lem:maps-from-exact-to-injective-nullhomotopic}
	Let $\cochaincomplex{C}$ be an exact cocomplex and \[
    	\cochaincomplex{I}: 0 \to I^k \to I^{k+1} \to I^{k+2} \to \cdots 
    \] be a cocomplex of injective objects (we call a cocomplex which is zero for all $i$ smaller or larger than some $k \in \mathbb{Z}$ a \emph{bounded} cocomplex). 
	Then any map of complexes $\cochaincomplex{f}: \cochaincomplex{C} \to \cochaincomplex{I}$ is nullhomotopic. 
\end{lem}
\begin{proof}
	We want to construct maps $h^i: C^i \to I^{i-1}$ that satisfy the homotopy condition with respect to $\cochaincomplex{f}$ and $0$: \[
    	f^i - 0 = d_B^{i-1}h^i + h^{i+1}d_I^i
    \] for all $i \in \mathbb{Z}$. We proceed inductively. For $i < k$ we let $h^i = 0$. Assume now that $h^j$ is defined for all $j \leq i$ for some $i \in \mathbb{Z}$. 
	Some computation yields an insight. We have 
	\begin{align*}
    	&(f^i - d_I^{i-1}h^i) \circ d_C^{i-1} = f^i \circ d_C^{i-1} - d_I ^{i-1} \circ h^i \circ d_C^{i-1}\\
		& d_I^{i-1}f^{i-1} - d_I^{i-1}\circ h^i \circ d_C^{i-1} = d_I^{i-1}(f^{i-1} - h^i \circ d_C^{i-1})\\
		& d^{i-1}_I \circ d_I^{i-2} \circ h^{i-1} = 0
    \end{align*} the last step by the inductive hypothesis. 
	We see that $f^i - d_I^{i-1}$ factors via the cokernel of $d_C^{i-1}$ which by exactness is the image of .... \todo{finish}. 
\end{proof}

\begin{cor}\label{cor:extend-map-exact-complex-to-complex-of-injectives}
	Let $0 \to Y \to I^0 \to I^1 \to \cdots$ be a bounded cocomplex of injectives, and let $0 \to X \to C^0 \to C^1 \to ...$ be an exact cocomplex. Then any map $f: X \to Y$ extends to the cocomplexes uniquely up to homotopy. 
\end{cor}
\begin{proof}
Construct the following solid commutative diagram 	
\[\begin{tikzcd}
	X & {C^0} & {C^1} & {C^2} & \cdots \\
	Y \\
	{I^0} & {I^1} & {I^2} & {I^3} & \cdots
	\arrow[from=1-1, to=1-2]
	\arrow["f"', from=1-1, to=2-1]
	\arrow[from=1-2, to=1-3]
	\arrow["{f^0}"{pos=0.6}, dashed, from=1-2, to=3-1]
	\arrow["0"{pos=0.2}, from=1-2, to=3-2]
	\arrow[from=1-3, to=1-4]
	\arrow["{f^1}"{pos=0.6}, dashed, from=1-3, to=3-2]
	\arrow["0"{pos=0.2}, from=1-3, to=3-3]
	\arrow[from=1-4, to=1-5]
	\arrow["{f^2}"{pos=0.6}, from=1-4, to=3-3]
	\arrow["0"{pos=0.2}, from=1-4, to=3-4]
	\arrow[from=2-1, to=3-1]
	\arrow[from=3-1, to=3-2]
	\arrow[from=3-2, to=3-3]
	\arrow[from=3-3, to=3-4]
	\arrow[from=3-4, to=3-5]
\end{tikzcd}\]
giving a map of cocomplexes from $X \to C^0 \to C^1 \to C^2 \to \cdots$ to $I^0 \to I^1 \to I^2 \to \cdots$. 
By \cref{lem:maps-from-exact-to-injective-nullhomotopic} \todo{rest.}
\end{proof}

\begin{cor}
	Let $F: \cat A \to \cat B$ be a left exact functor between abelian categories, and suppose $\cat A$ has enough injectives. Then the derived functors $\rightderived^iF$ are well defined for all $i$. 
\end{cor}
\begin{proof}
	Suppose $0 \to X \to \cochaincomplex{I}$ and $0 \to X \to \cochaincomplex{J}$ are injective resolutions. Then the identity map $\id_X: X \to X$ extends to two maps $\cochaincomplex{f}: \cochaincomplex{I} \to \cochaincomplex{J}$ and $\cochaincomplex{g}: \cochaincomplex{J} \to \cochaincomplex{I}$. Since either composition extends the identity on $X$ uniquely up to homotopy, and the identity on the complexes extends the identity on $X$, the map $\cochaincomplex{f}$ is a homotopy equivalence, so $F(f)$ is a quasi-isomorphism by \cref{exc:properties-of-homotopy}, and thus $H^i(F(\cochaincomplex{I})) \cong H^i(F(\cochaincomplex{J}))$. \todo{prove functoriality of the derived functors}.
\end{proof}

\section{Sheaf cohomology}

The \emph{sections functor} \(\Gamma(U,-)\colon\catAbelianSheaf(X)\to\catAbelianGroup,\ \mathcal F\mapsto\mathcal F(U)\) is left exact for any open \(U\) of a space \(X\).
Hence the right derived functors \(\rightderived^i\Gamma(U,-)\) exist and are well-defined.

\begin{defn}
	Let $X$ be a topological space and let $U \subset X$ be an open subset. 
	The $i$th \emph{sheaf cohomology} \todo{of $U$?} is the \(i\)th right derived functor $\rightderived^i \Gamma(U, -)$. It is denoted $H^i(U, -)$.
\end{defn}

\begin{exmp}
In Homework~6, Exercise~2, you constructed a short exact sequence
\[ 0\to\underline{\bb Z}\xrightarrow{2\pi i}\mathcal O\xrightarrow{\exp}\mathcal O^\times\to 0 \]
of sheaves on \(\bb C^n\).
For some fixed open \(U\subseteq\bb C^n\), the sheaf cohomology \(H^i(U,-)\) is the \(i\)th right derived functor of \(\Gamma(U,-)\).
The short exact sequence gives rise to a long exact sequence
        \[\begin{tikzcd}[row sep=small]
            0 \ar[r] & H^0(U, \underline{\bb Z}) \ar[r] & H^0(U, \mathcal O) \ar[r] \ar[draw=none]{d}[name=X, anchor=center]{} & H^0(U, \mathcal O^\times)
            \ar[rounded corners,
            to path={ -- ([xshift=2ex]\tikztostart.east)
                      |- (X.center) \tikztonodes
                      -| ([xshift=-2ex]\tikztotarget.west)
                      -- (\tikztotarget)}]{dll}[at end]{} \\
            & H^1(U, \underline{\bb Z}) \ar[r] & H^1(U, \mathcal O) \ar[r] & H^1(U, \mathcal O^\times) \ar[r] & \ldots
        \end{tikzcd}\]
\end{exmp}

%%% Local Variables:
%%% mode: latex
%%% TeX-master: "../main"
%%% End:
